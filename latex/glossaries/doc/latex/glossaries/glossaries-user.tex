\documentclass[report,inlinetitle]{nlctdoc}

\usepackage[inner=0.5in,includemp]{geometry}

\usepackage{array}
\usepackage{alltt}
\usepackage{pifont}
\ifpdf
\usepackage{lmodern}
\usepackage{mathpazo}
\usepackage[scaled=.88]{helvet}
\usepackage{courier}
\else
 \newcommand{\grimace}{{\fontencoding {U}\fontfamily{futs}\selectfont \char77}}
\fi
\usepackage[colorlinks,
            bookmarks,
            hyperindex=false,
            pdfauthor={Nicola L.C. Talbot},
            pdftitle={User Manual for glossaries.sty},
            pdfkeywords={LaTeX,package,glossary,acronyms}]{hyperref}
\usepackage{xr-hyper}
\usepackage[
  xindy,
  nonumberlist,
  toc,
  nopostdot,
  nogroupskip,
  style=altlist
]{glossaries}
\usepackage{glossary-inline}

\pagestyle{headings}

\glsnoexpandfields

\renewcommand*{\glsseeformat}[3][\seename]{%
  (\xmakefirstuc{#1} \glsseelist{#2}.)%
}

\renewcommand*{\glossarypreamble}{%
\emph{This glossary style was setup using:}
\begin{ttfamily}
\begin{tabbing}
\cs{usepackage}[\=xindy,\\
\+\>nonumberlist,\\
  toc,\\
  nopostdot,\\
  style=altlist,\\
  nogroupskip]\{glossaries\}
\end{tabbing}
\end{ttfamily}
}

\ifpdf
\else
  % Need an extra line break in the html version
  % (Don't have time to fiddle with cfg files!)
  \renewcommand*{\glossentry}[2]{%
    \item[\glsentryitem{#1}\glstarget{#1}{\glossentryname{#1}}]\mbox{}\newline
      \glossentrydesc{#1}\glspostdescription\space #2\newline}%
\fi

\makeglossaries

\renewcommand*{\main}[1]{\hyperpage{#1}}
\newcommand*{\htextbf}[1]{\textbf{\hyperpage{#1}}}
\newcommand*{\itermdef}[1]{\index{#1|htextbf}}

\newcommand{\itempar}[1]{\item[{#1}]\mbox{}\par}

\newcommand{\glslike}{\hyperref[sec:gls-like]{\cs{gls}-like}}
\newcommand{\glstextlike}{\hyperref[sec:glstext-like]{\cs{glstext}-like}}

\IndexPrologue{\chapter*{\indexname}
 \markboth{\indexname}{\indexname}}

\longnewglossaryentry{indexingapp}%
{
  name={Indexing application},
  text={indexing application}
}%
{%
  An application (piece of software) separate from
  \TeX/\LaTeX\ that collates and sorts information that has an
  associated page reference. Generally the information is an index
  entry but in this case the information is a glossary entry.
  There are two main indexing applications that are used with \TeX:
  \gls{makeindex} and \gls{xindy}. These are both 
  \gls{cli} applications.
}

\longnewglossaryentry{cli}
{
  name={Command Line Interface (CLI)},
  first={command line interface (CLI)},
  text={CLI}
}
{%
  An application that doesn't have a graphical user
  interface. That is, an application that doesn't have any windows,
  buttons or menus and can be run in 
  \href{http://www.dickimaw-books.com/latex/novices/html/terminal.html}{a command
  prompt or terminal}.
}

\longnewglossaryentry{xindy}{%
  name={\appfmt{xindy}\index{xindy=\appfmt{xindy}|htextbf}},
  sort={xindy},
  text={\appfmt{xindy}\iapp{xindy}}
}%
{%
  A flexible \gls{indexingapp} with multilingual
  support written in Perl.
}

\newglossaryentry{makeindex}{%
name={\appfmt{makeindex}\index{makeindex=\appfmt{makeindex}|htextbf}},%
sort={makeindex},%
text={\appfmt{makeindex}\iapp{makeindex}},%
description={An \gls{indexingapp}.},
}

\newglossaryentry{makeglossaries}{%
name={\appfmt{makeglossaries}\index{makeglossaries=\appfmt{makeglossaries}|htextbf}},%
sort={makeglossaries},%
text={\appfmt{makeglossaries}\iapp{makeglossaries}},%
description={A custom designed Perl script interface 
to \gls{xindy} and \gls{makeindex} provided with the \styfmt{glossaries}
package.}
}

\longnewglossaryentry{makeglossariesgui}{%
name={\appfmt{makeglossariesgui}\index{makeglossariesgui=\appfmt{makeglossariesgui}|htextbf}},%
sort={makeglossariesgui},%
text={\appfmt{makeglossariesgui}\iapp{makeglossariesgui}}%
}%
{%
  A Java GUI alternative to \gls{makeglossaries} that
  also provides diagnostic tools. Available separately on CTAN.
}

\newglossaryentry{numberlist}{%
name={Number list\itermdef{number list}},%
sort={number list},%
text={number list\iterm{number list}},%
description={A list of \glslink{entrylocation}{entry locations} (also 
called a location list). The number list can be suppressed using the 
\pkgopt{nonumberlist} package option.}
}

\newglossaryentry{entrylocation}{%
name={Entry location\itermdef{entry location}},%
sort={entry location},%
text={entry location\iterm{entry location}},%
description={The location of the entry in the document. This
defaults to the page number on which the entry appears. An entry may
have multiple locations.}
}

\newglossaryentry{locationlist}{%
name={Location list},%
text={location list},
sort={location list},%
description={A list of \glslink{entrylocation}{entry locations}.
See \gls{numberlist}.}%
}

\newglossaryentry{linktext}{%
name={Link text\itermdef{link text}},
sort={link text},%
text={link text\iterm{link text}},
description={The text produced by commands such as \ics{gls}. It may
or may not be a hyperlink to the glossary.}
}

\let\glsd\glsuseri
\let\glsation\glsuserii

\longnewglossaryentry{firstuse}{%
name={First use\ifirstuse},
sort={first use},%
text={first use},%
user1={first used}}
{%
  The first time a glossary entry is used (from the start of the 
  document or after a reset) with one of the
  following commands: \ics{gls}, \ics{Gls}, \ics{GLS}, \ics{glspl},
  \ics{Glspl}, \ics{GLSpl} or \ics{glsdisp}.
  \glsseeformat{firstuseflag,firstusetext}{}
}

\longnewglossaryentry{firstuseflag}{%
name={First use flag\ifirstuseflag},
sort={first use flag},%
text={first use flag}%
}
{%
  A conditional that determines whether or not the entry
  has been used according to the rules of \gls{firstuse}. Commands
  to unset or reset this conditional are described in 
  \sectionref{sec:glsunset}.
}

\newglossaryentry{firstusetext}{%
name={First use text\ifirstusetext},
sort={first use text},%
text={first use text},%
description={The text that is displayed on \gls{firstuse}, which is
governed by the \gloskey{first} and \gloskey{firstplural} keys of
\ics{newglossaryentry}. (May be overridden by
\ics{glsdisp} or by \ics{defglsentry}.)}%
}

\longnewglossaryentry{sanitize}{%
name={Sanitize\itermdef{sanitize}},%
sort={sanitize},
text={sanitize\iterm{sanitize}},%
user1={sanitized\protect\iterm{sanitize}},%
user2={sanitization\protect\iterm{sanitize}}%
}%
{%
  Converts command names into character sequences. That is, a command called,
  say, \cs{foo}, is converted into the sequence of characters:
  \cs{}, \texttt{f}, \texttt{o}, \texttt{o}. Depending on the font,
  the backslash character may appear as a dash when used in the
  main document text, so \cs{foo} will appear as: ---foo.

  Earlier versions of \styfmt{glossaries} used this technique to write
  information to the files used by the indexing applications to
  prevent problems caused by fragile commands. Now, this is only used
  for the \gloskey{sort} key.
}

\newglossaryentry{latinchar}{%
  name={Latin Character\itermdef{Latin character}},
  text={Latin character\iterm{Latin character}},
  sort={Latin character},
  description={One of the letters \texttt{a}, \ldots, \texttt{z}, 
  \texttt{A}, \ldots, \texttt{Z}\@.
  See also \gls{exlatinchar}.}
}

\newglossaryentry{exlatinchar}{%
  name={Extended Latin Character\itermdef{extended Latin character}},
  text={extended Latin character\iterm{extended Latin character}},
  sort={extended Latin character},
  description={A character that's created by combining \glspl{latinchar}
  to form ligatures (e.g.\ \ae) or by applying diacritical marks
  to a~\gls*{latinchar} or characters (e.g.\ \'a or \o). 
  See also \gls{nonlatinchar}.}
}

\newglossaryentry{latexexlatinchar}{%
  name={Standard \LaTeX\ Extended Latin Character\itermdef{standard
\LaTeX\ extended Latin character}},
  text={standard \LaTeX\ extended Latin character\iterm{standard
\LaTeX\ extended Latin character}},
  sort={standard LaTeX extended Latin character},
  description={An \gls{exlatinchar} that can be created by a~core
\LaTeX\ command, such as \cs{o} (\o) or \cs{'}\texttt{e} (\'e). 
  That is, the character can be produced without the need to load 
  a~particular package.}
}

\newglossaryentry{nonlatinchar}{%
  name={Non-Latin Character\itermdef{non-Latin character}},
  text={non-Latin character\iterm{non-Latin character}},
  sort={non-Latin character},
  description={An \gls{exlatinchar} or a~character that isn't
  a~\gls{latinchar}.}
}

\newglossaryentry{latinalph}{%
  name={Latin Alphabet\itermdef{Latin alphabet}},
  text={Latin alphabet\iterm{Latin alphabet}},
  sort={Latin alphabet},
  description={The alphabet consisting of \glspl{latinchar}.
  See also \gls{exlatinalph}.}
}

\newglossaryentry{exlatinalph}{%
  name={Extended Latin Alphabet\itermdef{Extended Latin Alphabet}},
  text={extended Latin alphabet},
  sort={extended Latin alphabet},
  description={An alphabet consisting of \glspl{latinchar}
  and \glspl{exlatinchar}.}
}

\newglossaryentry{nonlatinalph}{%
  name={Non-Latin Alphabet\itermdef{Non-Latin Alphabet}},
  text={non-Latin alphabet},
  sort={non-Latin alphabet},
  description={An alphabet consisting of \glspl{nonlatinchar}.}
}

\ifpdf
\externaldocument{glossaries-code}
\fi

\doxitem{Option}{option}{package options}
\doxitem{GlsKey}{key}{glossary keys}
\doxitem{Style}{style}{glossary styles}
\doxitem{Counter}{counter}{glossary counters}

\setcounter{IndexColumns}{2}

\newcommand*{\tick}{\ding{51}}

\newcommand*{\yes}{\ding{52}}
\newcommand*{\no}{\ding{56}}

\makeatletter
\newcommand*{\optionlabel}[1]{%
 \@glstarget{option#1}{}\textbf{Option~#1}}
\makeatother

\newcommand*{\opt}[1]{\hyperlink{option#1}{Option~#1}}
\newcommand*{\optsor}[2]{Options~\hyperlink{option#1}{#1}
or~\hyperlink{option#2}{#2}}
\newcommand*{\optsand}[2]{Options~\hyperlink{option#1}{#1}
and~\hyperlink{option#2}{#2}}


\newcommand*{\ifirstuse}{\iterm{first use}}
\newcommand*{\ifirstuseflag}{\iterm{first use>flag}}
\newcommand*{\ifirstusetext}{\iterm{first use>text}}

\newcommand*{\firstuse}{\gls{firstuse}}
\newcommand*{\firstuseflag}{\gls{firstuseflag}}
\newcommand*{\firstusetext}{\gls{firstusetext}}


\newcommand*{\istkey}[1]{\appfmt{#1}\index{makeindex=\appfmt{makeindex}>#1=\texttt{#1}|hyperpage}}
\newcommand*{\locfmt}[1]{\texttt{#1}\SpecialMainIndex{#1}}
\newcommand*{\mkidxspch}{\index{makeindex=\appfmt{makeindex}>special characters|hyperpage}}

\newcommand*{\igloskey}[2][newglossaryentry]{\icsopt{#1}{#2}}
\newcommand*{\gloskey}[2][newglossaryentry]{\csopt{#1}{#2}}

\newcommand*{\glostyle}[1]{\textsf{#1}\index{glossary styles:>#1={\protect\ttfamily#1}|main}}

\newcommand*{\acrstyle}[1]{\textsf{#1}\index{acronym styles:>#1={\protect\ttfamily#1}|main}}

\newcounter{sample}
\newcommand{\exitem}[2][sample]{%
  \item[\texttt{#1#2.tex}]%
  \refstepcounter{sample}\label{ex:#1#2}}

\newenvironment{samplelist}%
{\begin{description}}%
{\end{description}}

\newcommand*{\samplefile}[2][sample]{%
  \hyperref[ex:#1#2]{\texttt{#1#2.tex}}}

\ifpdf
  \newcommand*{\htmldoc}[2]{\qt{#1}}
\else
  \newcommand*{\htmldoc}[2]{\href{#2.html}{\qt{#1}}}
\fi

\begin{document}
\MakeShortVerb{"}
\DeleteShortVerb{\|}

 \title{User Manual for glossaries.sty v4.08}
 \author{Nicola L.C. Talbot\\%
  \url{http://www.dickimaw-books.com/}}

 \date{2014-07-30}
 \maketitle

\begin{abstract}
The \styfmt{glossaries} package provides a means to define terms or
acronyms or symbols that can be referenced within your document.
Sorted lists with collated \glslink{entrylocation}{locations} can be 
generated either using \TeX\ or using a supplementary \gls{indexingapp}.
\end{abstract}

\begin{important}
Documents have various styles when it comes to presenting glossaries
or lists of terms or notation. People have their own preferences and
to a large extent this is determined by the kind of information that
needs to go in the glossary. They may just have symbols with
terse descriptions or they may have long technical words with
complicated descriptions. The \styfmt{glossaries} package is
flexible enough to accommodate such varied requirements, but this
flexibility comes at a price: a~big manual.

\aargh\ If you're freaking out at the size of this manual, start with
\texttt{glossariesbegin.pdf} (\qt{The glossaries package: a guide
for beginnners}). You should find it in the same directory as this
document or try \texttt{texdoc glossariesbegin.pdf}. Once you've got
to grips with the basics, then come back to this manual to find out
how to adjust the settings.
\end{important}

\noindent
The \styfmt{glossaries} bundle comes with the following documentation:
\begin{description}
\item[\url{glossariesbegin.pdf}] 
If you are a complete beginner, start with 
\htmldoc{The glossaries package: a guide for
beginners}{glossariesbegin}.

\item[\url{glossary2glossaries.pdf}] 
If you are moving over from the obsolete \sty{glossary} package,
read \htmldoc{Upgrading from the glossary package to the
glossaries package}{glossary2glossaries}.

\item[glossaries-user.pdf]
This document is the main user guide for the \styfmt{glossaries}
package.

\item[\url{mfirstuc-manual.pdf}]
The commands provided by the \sty{mfirstuc} package are briefly
described in \htmldoc{mfirstuc.sty: uppercasing first
letter}{mfirstuc-manual}.

\item[\url{glossaries-code.pdf}]
Advanced users wishing to know more about the inner workings of all the
packages provided in the \styfmt{glossaries} bundle should read
\qt{Documented Code for glossaries v4.08}.
This includes the documented code for the \sty{mfirstuc} package.

\item[INSTALL] Installation instructions.

\item[CHANGES] Change log.

\item[README] Package summary.

\end{description}

\begin{important}
If you use \sty{hyperref} and \styfmt{glossaries}, you must load
\sty{hyperref} \emph{first}. Similarly the \sty{doc} package must
also be loaded before \styfmt{glossaries}. (If \sty{doc} is loaded,
the file extensions for the default main glossary are changed to
\filetype{gls2}, \filetype{glo2} and \filetype{.glg2} to avoid
conflict with \sty{doc}'s changes
glossary.)\hypertarget{pdflatexnote}{}%

If you are using \sty{hyperref}, it's best to use \app{pdflatex}
rather than \app{latex} (DVI format) as \appfmt{pdflatex} deals with
hyperlinks much better. If you use the DVI format, you will
encounter problems where you have long hyperlinks or hyperlinks in
subscripts or superscripts. This is an issue with the DVI format not
with \styfmt{glossaries}.
\end{important}

Other documents that describe using the \styfmt{glossaries} package include:
\href{http://www.dickimaw-books.com/latex/thesis/}{Using LaTeX to Write a PhD Thesis}
and
\href{http://www.latex-community.org/know-how/latex/55-latex-general/263-glossaries-nomenclature-lists-of-symbols-and-acronyms}{Glossaries,
Nomenclature, Lists of Symbols and Acronyms}.

\clearpage
\pdfbookmark[0]{Contents}{contents}
\tableofcontents
\clearpage
\pdfbookmark[0]{List of Examples}{examples}
\listofexamples
\clearpage
\pdfbookmark[0]{List of Tables}{tables}
\listoftables

\clearpage
\printglossaries

\glsresetall

 \chapter{Introduction}
\label{sec:intro}

The \styfmt{glossaries} package is provided to assist generating
lists of terms, symbols or abbreviations (glossaries). It has a certain amount of flexibility, allowing the
user to customize the format of the glossary and define multiple
glossaries. It also supports glossary styles that
include symbols (in addition to a name and description) for glossary
entries. There is provision for loading a database of glossary
terms. Only those terms used\footnote{That is, if the term has been
referenced using any of the commands described in
\sectionref{sec:glslink} and \sectionref{sec:glsadd} or via
\ics{glssee} (or the \gloskey{see} key) or commands such as
\ics{acrshort}.}\ in the document will be added to the glossary.

\textbf{This package replaces the \sty{glossary} package which is
now obsolete.} Please see the document \qtdocref{Upgrading from the
glossary package to the glossaries package}{glossary2glossaries} for
assistance in upgrading.

One of the strengths of this package is its flexibility, however
the drawback of this is the necessity of having a large manual
that can cover all the various settings. If you are daunted by the
size of the manual, try starting off with the much shorter
\docref{guide for beginners}{glossariesbegin}.

\begin{important}
There's a~common misconception that you have to have Perl installed
in order to use the \styfmt{glossaries} package. Perl is \emph{not}
a~requirement but it does increase the available options,
particularly if you use an \gls{exlatinalph} or a~\gls{nonlatinalph}.
\end{important}

The basic idea behind the \styfmt{glossaries} package is that you
first define your entries (terms, symbols or abbreviations). Then
you can reference these within your document (like \cs{cite} or
\cs{ref}).  You can also, optionally, display a~list of the entries
you have referenced in your document (the glossary). This last part,
displaying the glossary, is the part that most new users find
difficult. There are three options:

\begin{description}
\item[]\optionlabel1: 

 This is the simplest option but it's slow and if
 you want a sorted list, it doesn't work well for \glspl{exlatinalph} or 
 \glspl{nonlatinalph}. However, if you use the
 \pkgopt[false]{sanitizesort} package option (the default for
 Option~1) then the \glslink{latexexlatinchar}{standard \LaTeX\ accent commands} will be
 ignored, so if an entry's name is set to \verb|{\'e}lite| then the
 sort will default to \texttt{elite} if 
 \pkgopt[false]{sanitizesort} is used
 and will default to \verb|\'elite| if \pkgopt[true]{sanitizesort}
 is used.

  \begin{enumerate}
    \item Add \cs{makenoidxglossaries} to your preamble (before you
    start defining your entries, as described in
    \sectionref{sec:newglosentry}).

    \item Put
\begin{definition}
\cs{printnoidxglossary}
\end{definition}
    where you want your list of entries to appear (described in
    \sectionref{sec:printglossary}).

    \item Run \LaTeX\ twice on your document. (As you would do to
    make a~table of contents appear.) For example, click twice on
    the \qt{typeset} or \qt{build} or \qt{PDF\LaTeX} button in your editor.
  \end{enumerate}

\item\optionlabel2:

   This option uses a~\gls{cli} application called \gls{makeindex} to sort 
   the entries. This application comes with all modern \TeX\ distributions, 
   but it's hard-coded for the non-extended \gls{latinalph}, so 
   it doesn't work well for \glspl{exlatinalph} or
   \glspl{nonlatinalph}. This process involves making \LaTeX\ write the 
   glossary information to a temporary file which \gls{makeindex} reads. 
   Then \gls{makeindex} writes a~new file containing the code to typeset 
   the glossary. \LaTeX\ then reads this file in on the next run.

   \begin{enumerate}
    \item Add \cs{makeglossaries} to your preamble (before you start
    defining your entries, as described in
    \sectionref{sec:newglosentry}).

    \item Put
\begin{definition}
\cs{printglossary}
\end{definition}
    where you want your list of entries to appear (described in
    \sectionref{sec:printglossary}).

    \item Run \LaTeX\ on your document. This creates files with the
    extensions \texttt{.glo} and \texttt{.ist} (for example, if your 
    \LaTeX\ document is called \texttt{myDoc.tex}, then you'll have 
    two extra files called \texttt{myDoc.glo} and \texttt{myDoc.ist}).
    If you look at your document at this point, you won't see the 
    glossary as it hasn't been created yet.

    \item Run \gls{makeindex} with the \texttt{.glo} file as the
    input file and the \texttt{.ist} file as the style so that
    it creates an output file with the extension \texttt{.gls}. If
    you have access to a terminal or a command prompt (for example, the
    MSDOS command prompt for Windows users or the bash console for
    Unix-like users) then you need to run the command:
\begin{verbatim}
makeindex -s myDoc.ist -o myDoc.gls myDoc.glo
\end{verbatim}
   (Replace \texttt{myDoc} with the base name of your \LaTeX\
    document file. Avoid spaces in the file name.) If you don't know
    how to use the command prompt, then you can probably access
    \gls{makeindex} via your text editor, but each editor has a
    different method of doing this, so I~can't give a~general
    description. You will have to check your editor's manual.

    The default sort is word order (\qt{sea lion} comes before
\qt{seal}). 
    If you want letter ordering you need to add the \texttt{-l}
    switch:
\begin{verbatim}
makeindex -l -s myDoc.ist -o myDoc.gls myDoc.glo
\end{verbatim}
    (See \sectionref{sec:makeindexapp} for further details on using 
    \gls*{makeindex} explicitly.)

    \item Once you have successfully completed the previous step,
    you can now run \LaTeX\ on your document again.
   \end{enumerate}

   This is the default option (although you still need to use
   \cs{makeglossaries} to ensure the glossary files are created).

\item\optionlabel3:

   This option uses a~\gls{cli} application called
   \gls{xindy} to sort the entries. This application is more flexible than 
   \texttt{makeindex} and is able to sort \glspl{exlatinalph} or 
   \glspl{nonlatinalph}. The \gls{xindy} application comes with \TeX~Live 
   but not with MiK\TeX.  Since \gls{xindy} is a Perl script, if you are
   using MiK\TeX\ you will not only need to install \gls{xindy}, you
   will also need to install Perl. In a~similar way to \opt2, this 
   option involves making \LaTeX\ write the glossary information to 
   a~temporary file which \gls{xindy} reads. Then \gls{xindy} 
   writes a~new file containing the code to typeset the glossary. 
   \LaTeX\ then reads this file in on the next run.

   \begin{enumerate}
     \item Add the \pkgopt{xindy} option to the \styfmt{glossaries}
package option list:
\begin{verbatim}
\usepackage[xindy]{glossaries}
\end{verbatim}

     \item Add \cs{makeglossaries} to your preamble (before you start
     defining your entries, as described in \sectionref{sec:newglosentry}).

    \item Run \LaTeX\ on your document. This creates files with the
    extensions \texttt{.glo} and \texttt{.xdy} (for example, if your 
    \LaTeX\ document is called \texttt{myDoc.tex}, then you'll have 
    two extra files called \texttt{myDoc.glo} and \texttt{myDoc.xdy}).
    If you look at your document at this point, you won't see the 
    glossary as it hasn't been created yet.

    \item Run \gls{xindy} with the \texttt{.glo} file as the
    input file and the \texttt{.xdy} file as a~module so that
    it creates an output file with the extension \texttt{.gls}. You 
    also need to set the language name and input encoding. If
    you have access to a terminal or a command prompt (for example, the
    MSDOS command prompt for Windows users or the bash console for
    Unix-like users) then you need to run the command (all on one
    line):
\begin{verbatim}
xindy  -L english -C utf8 -I xindy -M myDoc 
-t myDoc.glg -o myDoc.gls myDoc.glo
\end{verbatim}
    (Replace \texttt{myDoc} with the base name of your \LaTeX\
    document file. Avoid spaces in the file name. If necessary, also replace
    \texttt{english} with the name of your language and \texttt{utf8}
    with your input encoding.) If you don't know
    how to use the command prompt, then you can probably access
    \gls{xindy} via your text editor, but each editor has a
    different method of doing this, so I~can't give a~general
    description. You will have to check your editor's manual.

    The default sort is word order (\qt{sea lion} comes before
\qt{seal}). 
    If you want letter ordering you need to add the
    \pkgopt[letter]{order} package option:
\begin{verbatim}
\usepackage[xindy,order=letter]{glossaries}
\end{verbatim}
    (See \sectionref{sec:xindyapp} for further details on using 
    \gls*{xindy} explicitly.)

    \item Once you have successfully completed the previous step,
    you can now run \LaTeX\ on your document again.

   \end{enumerate}

\end{description}

For \optsand23, it can be difficult to remember all the
parameters required for \gls{makeindex} or \gls{xindy}, so the
\styfmt{glossaries} package provides a~script called
\gls{makeglossaries} that reads the \texttt{.aux} file to
determine what settings you have used and will then run
\gls{makeindex} or \gls{xindy}. Again, this is a~command line
application and can be run in a~terminal or command prompt. For
example, if your \LaTeX\ document is in the file \texttt{myDoc.tex},
then run:
\begin{verbatim}
makeglossaries myDoc
\end{verbatim}
(Replace \texttt{myDoc} with the base name of your \LaTeX\ document
file. Avoid spaces in the file name.) This is described in more
detail in \sectionref{sec:makeglossaries}.

\begin{important}
The \texttt{.gls} and \texttt{.glo} are temporary files
created to help build your document. You should not edit or explicitly input
them. However, you may need to delete them if something goes wrong
and you need to do a fresh build.
\end{important}

An overview of these three options is given in
\tableref{tab:options}.

\begin{table}[htbp]
 \caption{Glossary Options: Pros and Cons}
 \label{tab:options}
 {%
 \centering
 \begin{tabular}{>{\raggedright}p{0.35\textwidth}ccc}
   & \bfseries \opt1 & \bfseries \opt2 & \bfseries \opt3\\
   Requires an external application? &
   \no & \yes & \yes\\
   Requires Perl? &
   \no & \no & \yes\\
   Can sort \glspl{exlatinalph}
   or \glspl{nonlatinalph}? &
   \no\textsuperscript{\textdagger} & \no & \yes\\
   Efficient sort algorithm? &
   \no & \yes & \yes\\
   Can use a different sort method for each glossary? &
   \yes & \no & \no\\
   Can form ranges in the location lists? &
   \no & \yes & \yes\\
   Can have non-standard locations in the location lists? &
   \yes & \no & \yes\\
   Maximum hierarchical depth &
   Unlimited & 3 & Unlimited\\
   \ics{glsdisplaynumberlist} reliable? &
   \yes & \no & \no\\
   \ics{newglossaryentry} restricted to preamble? &
   \yes & \no & \no\\
   Requires additional write registers? &
   \no & \yes & \yes\\
   Default value of \pkgopt{sanitizesort} package option &
   \texttt{false} & \texttt{true} & \texttt{true}
 \end{tabular}
 \par
 }\textsuperscript{\textdagger} Strips standard \LaTeX\ accents
(that is, accents generated by core \LaTeX\ commands) so,
for example, \ics{AA} is treated the same as A.
\end{table}

This document uses the \styfmt{glossaries} package. For example,
when viewing the PDF version of this document in a
hyperlinked-enabled PDF viewer (such as Adobe Reader or Okular) if
you click on the word \qt{\gls{xindy}} you'll be taken to the entry
in the glossary where there's a brief description of
the term \qt{\gls*{xindy}}.

The remainder of this introductory section covers the following:
\begin{itemize}
\item \sectionref{sec:samples} lists the sample documents provided 
with this package.

\item \sectionref{sec:languages} provides information for users who
wish to write in a language other than English.

\item \sectionref{sec:makeglossaries} describes how to use an
\gls{indexingapp} to create the sorted glossaries for your document
(\optsor23).

\end{itemize}

\section{Sample Documents}
\label{sec:samples}

The \styfmt{glossaries} package is provided with some sample
documents that illustrate the various functions. These should
be located in the \texttt{samples} subdirectory (folder) of the
\styfmt{glossaries} documentation directory. This location varies
according to your operating system and \TeX\ distribution. You
can use \texttt{texdoc} to locate the main glossaries documentation.
For example, in a
\href{http://www.dickimaw-books.com/latex/novices/html/terminal.html}{terminal or command prompt}, type:
\begin{prompt}
texdoc -l glossaries
\end{prompt}
This should display a list of all the files in the glossaries
documentation directory with their full pathnames.

If you can't find the sample files on your computer, they are also available
from your nearest CTAN mirror at
\url{http://mirror.ctan.org/macros/latex/contrib/glossaries/samples/}.

The sample documents are as follows\footnote{Note that although I've written
\texttt{latex} in this section, it's better to use \texttt{pdflatex}, where
possible, for the reasons given \hyperlink{pdflatexnote}{earlier}.}:
\begin{samplelist}
\exitem[minimal]{gls} This document is a
minimal working example. You can test your installation using this
file. To create the complete document you will need to do the
following steps:
  \begin{enumerate}
  \item Run \texttt{minimalgls.tex} through \LaTeX\ either by 
  typing
\begin{prompt}
latex minimalgls
\end{prompt}
  in a terminal or by using the relevant button or menu item in
  your text editor or front-end. This will create the required 
  associated files but you will not see the glossary. If you use 
  PDF\LaTeX\ you will also get warnings about non-existent 
  references that look something like:
\begin{verbatim}
pdfTeX warning (dest): name{glo:aca} has been 
referenced but does not exist, 
replaced by a fixed one
\end{verbatim}
  These warnings may be ignored on the first run.

  If you get a \verb"Missing \begin{document}" error, then 
  it's most likely that your version of \sty{xkeyval} is 
  out of date. Check the log file for a warning of that nature. 
  If this is the case, you will need to update the \styfmt{xkeyval}
  package.

  \item Run \gls{makeglossaries} on the document (\sectionref{sec:makeglossaries}). This can
  be done on a terminal either by typing
\begin{prompt}
makeglossaries minimalgls
\end{prompt}
  or by typing
\begin{prompt}
perl makeglossaries minimalgls
\end{prompt}
  If your system doesn't recognise the command \texttt{perl} then
  it's likely you don't have Perl installed. In which case you
  will need to use \gls{makeindex} directly. You can do this
  in a terminal by typing (all on one line):
\begin{prompt}
makeindex -s minimalgls.ist -t minimalgls.glg -o minimalgls.gls minimalgls.glo
\end{prompt}
  (See \sectionref{sec:makeindexapp} for further details on using 
   \gls*{makeindex} explicitly.)

  Note that if you need to specify the full path and the path
  contains spaces, you will need to delimit the file names with
  the double-quote character.

  \item Run \texttt{minimalgls.tex} through \LaTeX\ again (as step~1)
  \end{enumerate}
You should now have a complete document. The number following
each entry in the glossary is the location number. By default, this 
is the page number where the entry was referenced.

\exitem{-noidx} This document illustrates how to use the
\styfmt{glossaries} package without an external \gls{indexingapp} (\opt1).
To create the complete document, you need to do:
\begin{prompt}
latex sample-noidx
latex sample-noidx
\end{prompt}

\exitem{-noidx-utf8} As the previous example, except that it uses
the \sty{inputenc} package.
To create the complete document, you need to do:
\begin{prompt}
latex sample-noidx-utf8
latex sample-noidx-utf8
\end{prompt}

\exitem{4col} This document illustrates
a four column glossary where the entries have a symbol in addition
to the name and description. To create the complete document, you
need to do:
\begin{prompt}
latex sample4col
makeglossaries sample4col
latex sample4col
\end{prompt}
As before, if you don't have Perl installed, you will need to use
\gls{makeindex} directly instead of using
\gls{makeglossaries}. The vertical gap between entries is the
gap created at the start of each group. This can be suppressed
using the \pkgopt{nogroupskip} package option.

\exitem{Acr} This document has some
sample acronyms. It also adds the glossary to the table of contents,
so an extra run through \LaTeX\ is required to ensure the document
is up to date:
\begin{prompt}
latex sampleAcr
makeglossaries sampleAcr
latex sampleAcr
latex sampleAcr
\end{prompt}

\exitem{AcrDesc} This is similar to
the previous example, except that the acronyms have an associated
description. As with the previous example, the glossary is added to
the table of contents, so an extra run through \LaTeX\ is required:
\begin{prompt}
latex sampleAcrDesc
makeglossaries sampleAcrDesc
latex sampleAcrDesc
latex sampleAcrDesc
\end{prompt}

\exitem{Desc} This is similar to the
previous example, except that it defines the acronyms using
\ics{newglossaryentry} instead of \ics{newacronym}. As with the
previous example, the glossary is added to the table of contents, so
an extra run through \LaTeX\ is required:
\begin{prompt}
latex sampleDesc
makeglossaries sampleDesc
latex sampleDesc
latex sampleDesc
\end{prompt}

\exitem{CustomAcr} This document has some sample acronyms with a
custom acronym style. It also adds the glossary to the table of
contents, so an extra run through \LaTeX\ is required:
\begin{prompt}
latex sampleCustomAcr
makeglossaries sampleCustomAcr
latex sampleCustomAcr
latex sampleCustomAcr
\end{prompt}

\exitem{FnAcrDesc} This is similar to \samplefile{AcrDesc}, except that it 
uses the \acrstyle{footnote-sc-desc} style. As with the previous example, 
the glossary is added to the table of contents, so an extra run through 
\LaTeX\ is required:
\begin{prompt}
latex sampleFnAcrDesc
makeglossaries sampleFnAcrDesc
latex sampleFnAcrDesc
latex sampleFnAcrDesc
\end{prompt}

\exitem{-FnDesc} This example
defines a custom display format that puts the description in a
footnote on first use.
\begin{prompt}
latex sample-FnDesc
makeglossaries sample-FnDesc
latex sample-FnDesc
\end{prompt}

\exitem{-custom-acronym}
This document illustrates how to define your own acronym style if
the predefined styles don't suit your requirements.
\begin{prompt}
latex sample-custom-acronym
makeglossaries sample-custom-acronym
latex sample-custom-acronym
\end{prompt}

\exitem{-crossref}
This document illustrates how to cross-reference entries in the
glossary.
\begin{prompt}
latex sample-crossref
makeglossaries sample-crossref
latex sample-crossref
\end{prompt}

\exitem{DB} This document illustrates how
to load external files containing the glossary definitions. It also
illustrates how to define a new glossary type. This document has the
\gls{numberlist} suppressed and uses \ics{glsaddall} to add all
the entries to the glossaries without referencing each one
explicitly. To create the document do:
\begin{prompt}
latex sampleDB
makeglossaries sampleDB
latex sampleDB
\end{prompt}
The glossary definitions are stored in the accompanying files
\texttt{database1.tex} and \texttt{database2.tex}. Note that if you
don't have Perl installed, you will need to use \gls{makeindex}
twice instead of a single call to \gls{makeglossaries}:
\begin{enumerate}
\item Create the main glossary (all on one line):
\begin{prompt}
makeindex -s sampleDB.ist -t sampleDB.glg -o sampleDB.gls sampleDB.glo
\end{prompt}
\item Create the secondary glossary (all on one line):
\begin{prompt}
makeindex -s sampleDB.ist -t sampleDB.nlg -o sampleDB.not sampleDB.ntn
\end{prompt}
\end{enumerate}

\exitem{Eq} This document illustrates how
to change the location to something other than the page number. In
this case, the \texttt{equation} counter is used since all glossary
entries appear inside an \env{equation} environment. To create
the document do:
\begin{prompt}
latex sampleEq
makeglossaries sampleEq
latex sampleEq
\end{prompt}

\exitem{EqPg} This is similar to the
previous example, but the \glspl{numberlist} are a
mixture of page numbers and equation numbers. This example adds the
glossary to the table of contents, so an extra \LaTeX\ run is
required:
\begin{prompt}
latex sampleEqPg
makeglossaries sampleEqPg
latex sampleEqPg
latex sampleEqPg
\end{prompt}

\exitem{Sec} This document also
illustrates how to change the location to something other than the
page number. In this case, the \texttt{section} counter is used.
This example adds the glossary to the table of contents, so an extra
\LaTeX\ run is required:
\begin{prompt}
latex sampleSec
makeglossaries sampleSec
latex sampleSec
latex sampleSec
\end{prompt}

\exitem{Ntn} This document illustrates
how to create an additional glossary type. This example adds the
glossary to the table of contents, so an extra \LaTeX\ run is
required:
\begin{prompt}
latex sampleNtn
makeglossaries sampleNtn
latex sampleNtn
latex sampleNtn
\end{prompt}
Note that if you don't have Perl installed, you will need to use
\gls{makeindex} twice instead of a single call to
\gls{makeglossaries}:
\begin{enumerate}
\item Create the main glossary (all on one line):
\begin{prompt}
makeindex -s sampleNtn.ist -t sampleNtn.glg -o sampleNtn.gls sampleNtn.glo
\end{prompt}
\item Create the secondary glossary (all on one line):
\begin{prompt}
makeindex -s sampleNtn.ist -t sampleNtn.nlg -o sampleNtn.not sampleNtn.ntn
\end{prompt}
\end{enumerate}

\exitem{} This document illustrates some of
the basics, including how to create child entries that use the same
name as the parent entry. This example adds the glossary to the
table of contents and it also uses \ics{glsrefentry}, so an extra \LaTeX\ 
run is required:
\begin{prompt}
latex sample
makeglossaries sample
latex sample
latex sample
\end{prompt}
You can see the difference between word and letter ordering if you
substitute \pkgopt[word]{order} with \pkgopt[letter]{order}. (Note
that this will only have an effect if you use
\gls{makeglossaries}. If you use \gls{makeindex} explicitly,
you will need to use the \texttt{-l} switch to indicate letter
ordering.)

\exitem{-inline} This document is
like \texttt{sample.tex}, above, but uses the \glostyle{inline}
glossary style to put the glossary in a footnote.

\exitem{tree} This document illustrates
a hierarchical glossary structure where child entries have different
names to their corresponding parent entry. To create the document
do:
\begin{prompt}
latex sampletree
makeglossaries sampletree
latex sampletree
\end{prompt}

\exitem{-dual} This document
illustrates how to define an entry that both appears in the list of
acronyms and in the main glossary. To create the document do:
\begin{prompt}
latex sample-dual
makeglossaries sample-dual
latex sample-dual
\end{prompt}

\exitem{-langdict} This document
illustrates how to use the glossaries package to create English
to French and French to English dictionaries. To create the document
do:
\begin{prompt}
latex sample-langdict
makeglossaries sample-langdict
latex sample-langdict
\end{prompt}

\exitem{xdy} This document illustrates
how to use the \styfmt{glossaries} package with \gls{xindy} instead
of \gls{makeindex}. The document uses UTF8 encoding (with the
\sty{inputenc} package). The encoding is picked up by
\gls{makeglossaries}. By default, this document will create a
\gls{xindy} style file called \texttt{samplexdy.xdy}, but if you
uncomment the lines
\begin{verbatim}
\setStyleFile{samplexdy-mc}
\noist
\GlsSetXdyLanguage{}
\end{verbatim}
it will set the style file to \texttt{samplexdy-mc.xdy} instead.
This provides an additional letter group for entries starting with
\qt{Mc} or \qt{Mac}. If you use \gls{makeglossaries}, you don't
need to supply any additional information. If you don't use
\gls*{makeglossaries}, you will need to specify the required
information. Note that if you set the style file to
\texttt{samplexdy-mc.xdy} you must also specify \ics{noist},
otherwise the \styfmt{glossaries} package will overwrite
\texttt{samplexdy-mc.xdy} and you will lose the \qt{Mc} letter
group.

To create the document do:
\begin{prompt}
latex samplexdy
makeglossaries samplexdy
latex samplexdy
\end{prompt}
If you don't have Perl installed, you will have to call 
\gls{xindy} explicitly instead of using \gls{makeglossaries}.
If you are using the default style file \texttt{samplexdy.xdy}, then
do (no line breaks):
\begin{prompt}
xindy -L english -C utf8 -I xindy -M samplexdy -t samplexdy.glg -o samplexdy.gls samplexdy.glo
\end{prompt}
otherwise, if you are using \texttt{samplexdy-mc.xdy}, then do
(no line breaks):
\begin{prompt}
xindy -I xindy -M samplexdy-mc -t samplexdy.glg -o samplexdy.gls samplexdy.glo
\end{prompt}

\exitem{xdy2} This document illustrates
how to use the \styfmt{glossaries} package where the location
numbers don't follow a standard format. This example will only work
with \gls{xindy}. To create the document do:
\begin{prompt}
pdflatex samplexdy2
makeglossaries samplexdy2
pdflatex samplexdy2
\end{prompt}
If you can't use \gls{makeglossaries} then you need to do (all on
one line):
\begin{prompt}
xindy -L english -C utf8 -I xindy -M samplexdy2 -t samplexdy2.glg -o samplexdy2.gls samplexdy2.glo
\end{prompt}
See \sectionref{sec:xindyloc} for further details.

\exitem{utf8} This is another example
that uses \gls{xindy}. Unlike \gls{makeindex},
\gls{xindy} can cope with \glspl{nonlatinchar}. This
document uses UTF8 encoding. To create the document do:
\begin{prompt}
latex sampleutf8
makeglossaries sampleutf8
latex sampleutf8
\end{prompt}
If you don't have Perl installed, you will have to call
\gls{xindy} explicitly instead of using \gls{makeglossaries}
(no line breaks):
\begin{prompt}
xindy -L english -C utf8 -I xindy -M sampleutf8 -t sampleutf8.glg -o sampleutf8.gls sampleutf8.glo
\end{prompt}

If you remove the \pkgopt{xindy} option from \texttt{sampleutf8.tex} 
and do:
\begin{prompt}
latex sampleutf8
makeglossaries sampleutf8
latex sampleutf8
\end{prompt}
you will see that the entries that start with an \gls{exlatinchar}
now appear in the symbols group, and the word \qt{man\oe uvre} is now 
after \qt{manor} instead of before it. If you are unable to use
\gls{makeglossaries}, the call to \gls{makeindex} is as
follows (no line breaks):
\begin{prompt}
makeindex -s sampleutf8.ist -t sampleutf8.glg -o sampleutf8.gls sampleutf8.glo
\end{prompt}

\exitem{-index} This document uses
the \styfmt{glossaries} package to create both a glossary and an
index. This requires two \gls{makeglossaries} calls to ensure the
document is up to date:
\begin{prompt}
latex sample-index
makeglossaries sample-index
latex sample-index
makeglossaries sample-index
latex sample-index
\end{prompt}

\exitem{-newkeys} This document illustrates how add custom keys.

\exitem{-numberlist} This document illustrates how to reference the
\gls{numberlist} in the document text. This requires an additional
\LaTeX\ run:
\begin{prompt}
latex sample-numberlist
makeglossaries sample-numberlist
latex sample-numberlist
latex sample-numberlist
\end{prompt}

\exitem{People} This document illustrates how you can hook into the
standard sort mechanism to adjust the way the sort key is set. This
requires an additional run to ensure the table of contents is
up-to-date:
\begin{prompt}
latex samplePeople
makeglossaries samplePeople
latex samplePeople
latex samplePeople
\end{prompt}

\exitem{Sort} This is another document that illustrates how to hook
into the standard sort mechanism. An additional run is required to
ensure the table of contents is up-to-date:
\begin{prompt}
latex sampleSort
makeglossaries sampleSort
latex sampleSort
latex sampleSort
\end{prompt}

\exitem{-nomathhyper} This document illustrates how to selectively
enable and disable entry hyperlinks in \ics{glsentryfmt}.

\exitem{-entryfmt} This document illustrates how to change the
way an entry is displayed in the text.

\exitem{-prefix} This document illustrates the use of the
\sty{glossaries-prefix} package. An additional run is required to
ensure the table of contents is up-to-date:
\begin{prompt}
latex sample-prefix
makeglossaries sample-prefix
latex sample-prefix
latex sample-prefix
\end{prompt}

\exitem{accsupp} This document
uses the experimental \sty{glossaries-accsupp} package. The
symbol is set to the replacement text. Note that some PDF
viewers don't use the accessibility support. Information 
about the \styfmt{glossaries-accsupp} package can be found in
\sectionref{sec:accsupp}.

\exitem{-ignored} This document defines an ignored glossary for
common terms that don't need a definition.

\end{samplelist}

\section{Dummy Entries for Testing}
\label{sec:lipsum}

In addition to the sample files described above, \sty{glossaries} also provides
some files containing lorum ipsum dummy entries. These are provided
for testing purposes and are on \TeX's path (in
\texttt{tex\slash latex\slash glossaries\slash test-entries}) so they can be included
via \ics{input} or \ics{loadglsentries}. The files are as follows:
\begin{description}
\item[example-glossaries-brief.tex] These entries all have brief
descriptions.

\item[example-glossaries-long.tex] These entries all have long
descriptions.

\item[example-glossaries-multipar.tex] These entries all have
multi-paragraph descriptions.

\item[example-glossaries-symbols.tex] These entries all use the
\gloskey{symbol} key.

\item[example-glossaries-images.tex] These entries use the
\gloskey{user1} key to store the name of an image file. (The images
are provided by the \sty{mwe} package and should be on \TeX's path.) 
One entry doesn't have an associated image to help test for a~missing key.

\item[example-glossaries-acronym.tex] These entries are all acronyms.

\item[example-glossaries-acronym-desc.tex] These entries are all
acronyms that use the \gloskey{description} key.

\item[example-glossaries-acronyms-lang.tex] These entries are all
acronyms, where some of them have a~translation supplied in the
\gloskey{user1} key.

\item[example-glossaries-parent.tex] These are hierarchical entries
where the child entries use the \gloskey{name} key.

\item[example-glossaries-childnoname.tex] These are hierarchical entries
where the child entries don't use the \gloskey{name} key.

\item[example-glossaries-cite.tex] These entries use the
\gloskey{user1} key to store a citation key (or comma-separated list
of citation keys). The citations are defined in \texttt{xampl.bib},
which should be available on all modern \TeX\ distributions.
One entry doesn't have an associated citation to help test for a~missing
key.
\end{description}

The sample file \texttt{glossary-lipsum-examples.tex} in the 
\texttt{doc\slash latex\slash glossaries\slash samples} directory
uses all these files.

\section{Multi-Lingual Support}
\label{sec:languages}

As from version 1.17, the \styfmt{glossaries} package can now be
used with \gls{xindy} as well as \gls{makeindex}. If you
are writing in a language that uses an \gls{exlatinalph} or
\gls{nonlatinalph} it is recommended that you use \gls*{xindy}
as \gls*{makeindex} is hard-coded for the non-extended
\gls{latinalph}. This
means that you are not restricted to the A, \ldots, Z letter groups.
If you want to use \gls*{xindy}, remember to use the
\pkgopt{xindy} package option. For example:
\begin{verbatim}
\documentclass[frenchb]{article}
\usepackage[utf8]{inputenc}
\usepackage[T1]{fontenc}
\usepackage{babel}
\usepackage[xindy]{glossaries}
\end{verbatim}

\begin{important}
Note that although a~\gls{nonlatinchar}, such as \'e, looks like a plain
character in your tex file, it's actually a~macro and can therefore
cause expansion problems. You may need to switch off the field
expansions with \cs{glsnoexpandfields}.

If you use a~\gls{nonlatinchar} (or other expandable) character at the start of
an entry name, you must place it in a group, or it will cause
a problem for commands that convert the first letter to upper case
(e.g.\ \ics{Gls}) due to expansion issues. For example:
\begin{alltt}
\verb|\newglossaryentry}{elite}{name={{|\'e\verb|}lite},|
description=\verb|{select group or class}}|
\end{alltt}
\end{important}

If you use the \sty{inputenc} package, \gls{makeglossaries}
will pick up the encoding from the auxiliary file. If you use
\gls{xindy} explicitly instead of via \gls*{makeglossaries},
you may need to specify the encoding using the \texttt{-C} 
option. Read the \gls*{xindy} manual for further details.

\subsection{Changing the Fixed Names}
\label{sec:fixednames}

As from version 1.08, the \styfmt{glossaries} package now has
limited multi-lingual support, thanks to all the people who have sent
me the relevant translations either via email or via 
\texttt{comp.text.tex}.
However you must load \sty{babel} or \sty{polyglossia} \emph{before} 
\styfmt{glossaries} to enable this. Note that if \sty{babel} is loaded 
and the \sty{translator} package is detected on \TeX's path, then the
\sty{translator} package will be loaded automatically, unless you
use the \pkgopt[false]{translate} or \pkgopt[babel]{translate}
package options.  However,
it may not pick up on the required languages so, if the predefined
text is not translated, you may need to explicitly load the
\sty{translator} package with the required languages. For example:
\begin{verbatim}
\usepackage[spanish]{babel}
\usepackage[spanish]{translator}
\usepackage{glossaries}
\end{verbatim}
Alternatively, specify the language as a class option rather
than a package option. For example:
\begin{verbatim}
\documentclass[spanish]{report}

\usepackage{babel}
\usepackage{glossaries}
\end{verbatim}

If you want to use \sty{ngerman} or \sty{german} instead
of \sty{babel}, you will need to include the \sty{translator} package 
to provide the translations. For example:
\begin{verbatim}
\documentclass[ngerman]{article}
\usepackage{ngerman}
\usepackage{translator}
\usepackage{glossaries}
\end{verbatim}

The languages are currently supported by the
\styfmt{glossaries} package are listed in
\tableref{tab:supportedlanguages}. Please note that (apart from
spelling mistakes) I don't intend to change the default translations
as it will cause compatibility problems.

If you want to add a~language not currently supported, you can post 
the contents of your \texttt{.dict} file on my feature request form
at \url{http://www.dickimaw-books.com/feature-request.html}. Please
use \LaTeX\ commands for \glspl{nonlatinchar} as the file must be
independent of the input encoding otherwise it won't be of any use
to people who use a~different encoding to yourself.

\begin{table}[htbp]
\caption{Supported Languages}
\label{tab:supportedlanguages}
\centering
\begin{tabular}{lc}
\bfseries Language & \bfseries As from version\\
Brazilian Portuguese & 1.17\\
Danish & 1.08\\
Dutch & 1.08\\
English & 1.08\\
French & 1.08\\
German & 1.08\\
Irish & 1.08\\
Italian & 1.08\\
Hungarian & 1.08\\
Polish & 1.13\\
Serbian & 2.06\\
Spanish & 1.08
\end{tabular}
\end{table}

The language dependent commands and \sty{translator} keys used by the 
glossaries package are listed in \tableref{tab:predefinednames}.

\begin{table}[htbp]
\caption{Customised Text}
\label{tab:predefinednames}
\centering
\setlength{\tabcolsep}{3pt}
\begin{tabular}{@{}l>{\raggedright}p{0.3\linewidth}>{\raggedright}p{0.4\linewidth}@{}}
\bfseries Command Name & \bfseries Translator Key Word &
\bfseries Purpose\cr
\ics{glossaryname} & \texttt{Glossary} & Title of the main glossary.\cr
\ics{acronymname} & \texttt{Acronyms} & Title of the list of acronyms
(when used with package option \pkgopt{acronym}).\cr
\ics{entryname} & \texttt{Notation (glossaries)} & 
Header for first column in the glossary (for 2, 3 or 4 column glossaries 
that support headers).\cr
\ics{descriptionname} & \texttt{Description (glossaries)} &
Header for second column in the glossary (for 2, 3 or 4 column glossaries
that support headers).\cr
\ics{symbolname} & \texttt{Symbol (glossaries)} & Header for symbol
column in the glossary for glossary styles that support this option.\cr
\ics{pagelistname} & \texttt{Page List (glossaries)} & 
Header for page list column in the glossary for glossaries that support 
this option.\cr
\ics{glssymbolsgroupname} & \texttt{Symbols (glossaries)} & 
Header for symbols section of the glossary for glossary styles that 
support this option.\cr
\ics{glsnumbersgroupname} & \texttt{Numbers (glossaries)} & Header for
numbers section of the glossary for glossary styles that support
this option.
\end{tabular}
\end{table}

Due to the varied nature of glossaries, it's likely that the
predefined translations may not be appropriate. If you are using the
\sty{babel} package and the \styfmt{glossaries} package option \pkgopt[babel]{translate}, you need to be familiar with the advice given in
\urlref{http://www.tex.ac.uk/cgi-bin/texfaq2html?label=latexwords}{changing
the words babel uses}. If you are using the \sty{translator}
package, then you can provide your own dictionary with the necessary
modifications (using \cs{deftranslation}) and load it using
\cs{usedictionary}.

\begin{important}
Note that the dictionaries are loaded at the beginning of the
document, so it won't have any effect if you put \cs{deftranslation}
in the preamble. It should be put in your personal dictionary
instead (as in the example below). See the \sty{translator}
documentation for further details. (Now with \sty{beamer}
documentation.)
\end{important}

Your custom dictionary doesn't have to be just a translation from
English to another language. You may prefer to have a dictionary for
a particular type of document. For example, suppose your
institution's in-house reports have to have the glossary labelled as
\qt{Nomenclature} and the page list should be labelled
\qt{Location}, then you can create a file called, say,
\begin{verbatim}
myinstitute-glossaries-dictionary-English.dict
\end{verbatim}
that contains the following:
\begin{verbatim}
\ProvidesDictionary{myinstitute-glossaries-dictionary}{English}
\deftranslation{Glossary}{Nomenclature}
\deftranslation{Page List (glossaries)}{Location}
\end{verbatim}
You can now load it using:
\begin{verbatim}
\usedictionary{myinstitute-glossaries-dictionary}
\end{verbatim}
(Make sure that \texttt{myinstitute-glossaries-dictionary-English.dict}
can be found by \TeX.) If you want to share your custom dictionary,
you can upload it to \href{http://www.ctan.org/}{CTAN}.

If you are using \sty{babel} and don't want to use the
\sty{translator} interface, you can use the package
option \pkgopt[babel]{translate}. For example:
\begin{verbatim}
\documentclass[british]{article}

\usepackage{babel}
\usepackage[translate=babel]{glossaries}

\addto\captionsbritish{%
    \renewcommand*{\glossaryname}{List of Terms}%
    \renewcommand*{\acronymname}{List of Acronyms}%
}
\end{verbatim}

If you are using \sty{polyglossia} instead of \sty{babel}, 
\sty{glossaries-polyglossia} will automatically be loaded unless
you specify the package option \pkgopt[false]{translate}.

Note that \gls{xindy} provides much better multi-lingual support
than \gls{makeindex}, so it's recommended that you use \gls*{xindy}
if you have glossary entries that contain 
\glspl{nonlatinchar}. See \sectionref{sec:xindy} for further
details.

\section{Generating the Associated Glossary Files}
\label{sec:makeglossaries}

\begin{important}
This section is only applicable if you have chosen \optsor23. You can
ignore this section if you have chosen \opt1.
\end{important}

If this section seriously confuses you, and you can't work out how
to run \gls{makeglossaries} or \gls{makeindex}, you can try using
the \pkgopt{automake} package option, described in 
\sectionref{sec:pkgopts-sort}.

In order to generate a sorted glossary with compact
\glspl{numberlist}, it is necessary to use an external
\gls{indexingapp} as an intermediate step (unless you have chosen
\opt1). It is this application that creates the file containing the
code that typesets the glossary. If this step is omitted, the
glossaries will not appear in your document. The two
\glspl*{indexingapp} that are most commonly used with \LaTeX\ are
\gls{makeindex} and \gls{xindy}. As from version 1.17, the
\styfmt{glossaries} package can be used with either of these
applications. Previous versions were designed to be used with
\gls*{makeindex} only. Note that \gls*{xindy} has much better
multi-lingual support than \gls*{makeindex}, so \gls*{xindy} is
recommended if you're not writing in English. Commands that only
have an effect when \gls*{xindy} is used are described in
\sectionref{sec:xindy}.

\begin{important}
This is a multi-stage process, but there are methods of automating
document compilation using applications such as \app{latexmk} and
\app{arara}. See
\url{http://www.dickimaw-books.com/latex/thesis/html/build.html} for
more information.
\end{important}

The \styfmt{glossaries} package comes with the Perl script
\gls{makeglossaries} which will run \gls{makeindex} or \gls{xindy}
on all the glossary files using a customized style file (which is
created by \ics{makeglossaries}). See
\sectionref{sec:makeglossariesapp} for further
details. Perl is stable, cross-platform, open source software that
is used by a number of \TeX-related applications. Most Unix-like
operating systems come with a~Perl interpreter. \TeX~Live also comes
with a~Perl interpreter. MiK\TeX\ doesn't come with a~Perl
interpreter so if you are a~Windows MiK\TeX\ user you will need to
install Perl if you want to use \gls{makeglossaries}.
Further information is available at \url{http://www.perl.org/about.html}
and
\href{http://tex.stackexchange.com/questions/158796/miktex-and-perl-scripts-and-one-python-script}{MiKTeX and Perl scripts (and one Python script)}.

The advantages of using \gls*{makeglossaries}:
\begin{itemize}
\item It automatically detects whether to use \gls*{makeindex} or
\gls*{xindy} and sets the relevant application switches.

\item One call of \gls*{makeglossaries} will run 
\gls*{makeindex}\slash\gls*{xindy} for each glossary type.

\item If things go wrong, \gls{makeglossaries} will scan the
messages from \gls{makeindex} or \gls{xindy} and attempt to diagnose
the problem in relation to the \styfmt{glossaries} package. This
will hopefully provide more helpful messages in some cases. If it
can't diagnose the problem, you will have to read the relevant transcript
file and see if you can work it out from the \gls*{makeindex} or
\gls*{xindy} messages.

\end{itemize}

There is also a Java GUI alternative called \gls{makeglossariesgui},
distributed separately, that has diagnostic tools.

Whilst it is strongly recommended that you use the
\gls{makeglossaries} script or \gls{makeglossariesgui}, it is
possible to use the \styfmt{glossaries} package without using either
application. However, note that some commands and package options
have no effect if you don't use \gls*{makeglossaries} or
\gls*{makeglossariesgui}. These are listed in
\tableref{tab:makeglossariesCmds}.

\begin{important}
If you are choosing not to use \gls*{makeglossaries} because you
don't want to install Perl, you will only be able to use
\gls*{makeindex} as \gls*{xindy} also requires Perl.
\end{important}

Note that if any of your entries use an entry
that is not referenced outside the glossary, you will need to
do an additional \gls{makeglossaries}, \gls{makeindex}
or \gls{xindy} run, as appropriate. For example,
suppose you have defined the following entries:\footnote{As from
v3.01 \ics{gls} is no longer fragile and doesn't need protecting.}
\begin{verbatim}
\newglossaryentry{citrusfruit}{name={citrus fruit},
description={fruit of any citrus tree. (See also 
\gls{orange})}}

\newglossaryentry{orange}{name={orange},
description={an orange coloured fruit.}}
\end{verbatim}
and suppose you have \verb|\gls{citrusfruit}| in your document
but don't reference the \texttt{orange} entry, then the
\texttt{orange} entry won't appear in your glossary until
you first create the glossary and then do another run
of \gls{makeglossaries}, \gls{makeindex} or \gls{xindy}.
For example, if the document is called \texttt{myDoc.tex}, then
you must do:
\begin{prompt}
latex myDoc
makeglossaries myDoc
latex myDoc
makeglossaries myDoc
latex myDoc
\end{prompt}

Likewise, an additional \gls{makeglossaries} and \LaTeX\ run
may be required if the document pages shift with re-runs. For
example, if the page numbering is not reset after the table of
contents, the insertion of the table of contents on the second
\LaTeX\ run may push glossary entries across page boundaries, which
means that the \glspl{numberlist} in the glossary may 
need updating.

The examples in this document assume that you are accessing
\gls{makeglossaries}, \gls{xindy} or \gls{makeindex} via a terminal.
Windows users can use the MSDOS Prompt which is usually accessed via
the \menu{Start}\submenu{All Programs} menu or
\menu{Start}\submenu{All Programs}\submenu{Accessories} menu.

Alternatively, your text editor may have the facility to create a
function that will call the required application. The article
\href{http://www.latex-community.org/index.php?option=com_content&view=article\&id=263:glossaries-nomenclature-lists-of-symbols-and-acronyms&catid=55:latex-general&Itemid=114}{\qt{Glossaries, Nomenclature, List of Symbols and
Acronyms}}
in the \urlfootref{http://www.latex-community.org/}{\LaTeX\
Community's} Know How section
describes how to do this for TeXnicCenter, and the thread
\href{http://groups.google.com/group/comp.text.tex/browse_thread/thread/edd83831b81b0759?hl=en}{\qt{Executing Glossaries' makeindex from a WinEdt
macro}} on the \texttt{comp.text.tex} newsgroup
describes how to do it for WinEdt. \href{http://www.dickimaw-books.com/latex/thesis/html/build.html}{Section
1.1 (Building Your Document)}
of \urlfootref{http://www.dickimaw-books.com/latex/thesis/}{\qt{Using \LaTeX\ to Write a PhD Thesis}} describes how to do it
for TeXWorks.
For other editors see the editor's
user manual for further details.

If any problems occur, remember to check the transcript files 
(e.g.\ \filetype{.glg} or \filetype{.alg}) for messages.

\begin{table}[htbp]
\caption[Commands and package options that have no effect when
using xindy or makeindex explicitly]{Commands and package options that have no effect when
using \gls{xindy} or \gls{makeindex} explicitly}
\label{tab:makeglossariesCmds}
\vskip\baselineskip
\begin{tabular}{@{}lll@{}}
\bfseries Command or Package Option & \gls{makeindex} & 
\gls{xindy}\\
\pkgopt[letter]{order} & use \texttt{-l} &
 use \texttt{-M ord/letorder}\\
\pkgopt[word]{order} & default & default\\
\pkgopt{xindy}=\{\pkgoptfmt{language=}\meta{lang}\pkgoptfmt{,codename=}\meta{code}\} &
N/A & use \texttt{-L} \meta{lang} \texttt{-C} \meta{code}\\
\ics{GlsSetXdyLanguage}\marg{lang} & N/A &
use \texttt{-L} \meta{lang}\\
\ics{GlsSetXdyCodePage}\marg{code} & N/A &
use \texttt{-C} \meta{code}
\end{tabular}
\par
\end{table}


\subsection{Using the makeglossaries Perl Script}
\label{sec:makeglossariesapp}

The \gls{makeglossaries} script picks up the relevant information
from the auxiliary (\filetype{.aux}) file and will either call
\gls{xindy} or \gls{makeindex}, depending on the supplied
information. Therefore, you only need to pass the document's name
without the extension to \gls*{makeglossaries}. For example, if your
document is called \texttt{myDoc.tex}, type the following in your
terminal:
\begin{prompt}
latex myDoc
makeglossaries myDoc
latex myDoc
\end{prompt}
You may need to explicitly load \gls{makeglossaries} into Perl:
\begin{prompt}
perl makeglossaries myDoc
\end{prompt}

Windows users: TeX~Live on Windows has its own internal Perl
interpreter and provides \texttt{makeglossaries.exe} as a~convenient
wrapper for the \gls{makeglossaries} Perl script. MiKTeX also
provides a~wrapper \texttt{makeglossaries.exe} but doesn't provide
a~Perl interpreter, which is still required even if you run MiKTeX's
\texttt{makeglossaries.exe}, so with MiKTeX you'll need to install
Perl. There's more information about this at
\url{http://tex.stackexchange.com/q/158796/19862} on the TeX.SX
site. Alternatively, there is a batch file called
\texttt{makeglossaries.bat} that should be located in the same
folder as the \gls{makeglossaries} Perl script. This just explicitly
loads the script into Perl. If you've installed Perl but for some
reason your operating system can't find \texttt{perl.exe}, you can
edit the \texttt{makeglossaries.bat} file to include the full path
to \texttt{perl.exe} (but take care as this file will be overwritten
next time you update the \styfmt{glossaries} package). If you move
the \texttt{.bat} file to a new location, you will also need to
supply the full path to the \gls{makeglossaries} Perl script. (Don't
also move the Perl script as well or you may miss out on updates to
\gls{makeglossaries}.)

The \gls{makeglossaries} script attempts to fork the
\gls{makeindex}\slash\gls{xindy} process using \texttt{open()} on the
piped redirection \verb"2>&1 |" and parses the processor output to
help diagnose problems.  If this method fails \gls{makeglossaries}
will print an \qt{Unable to fork} warning and will retry without
redirection. If you run \gls{makeglossaries} on an operating system
that doesn't support this form of redirection, then you can use the
\texttt{-Q} switch to suppress this warning or you can use the
\texttt{-k} switch to make \gls{makeglossaries} automatically use
the fallback method without attempting the redirection. Without this
redirection, the \texttt{-q} (quiet) switch doesn't work as well.

You can specify in which directory the \filetype{.aux}, 
\filetype{.glo} etc files are located using the \texttt{-d} switch.
For example:
\begin{prompt}
pdflatex -output-directory myTmpDir myDoc
makeglossaries -d myTmpDir myDoc
\end{prompt}
Note that \gls*{makeglossaries} assumes by default that
\gls*{makeindex}\slash\gls*{xindy} is on your operating system's
path. If this isn't the case, you can specify the full pathname
using \texttt{-m} \meta{path/to/makeindex} for \gls*{makeindex}
or \texttt{-x} \meta{path/to/xindy} for \gls*{xindy}.

The \gls{makeglossaries} script contains POD (Plain Old
Documentation). If you want, you can create a man page for
\gls*{makeglossaries} using \app{pod2man} and move the 
resulting file onto the man path. Alternatively do
\texttt{makeglossaries -{}-help} for a list of all options or
\texttt{makeglossaries -{}-version} for the version number.

\begin{important}
When upgrading the \styfmt{glossaries} package, make sure you also
upgrade your version of \gls{makeglossaries}. The current version is
2.15.
\end{important}

\subsection[Using xindy explicitly (Option~3)]{Using
\gls{xindy} explicitly (\ifpdf \opt3\else Option 3\fi)}
\label{sec:xindyapp}

\Gls{xindy} comes with \TeX~Live, but not with MiK\TeX. However
Mik\TeX\ users can install it. See
\href{http://tex.stackexchange.com/questions/71167/how-to-use-xindy-with-miktex}{How
to use Xindy with MikTeX} on
\urlfootref{http://www.stackexchange.com/}{\TeX\ on StackExchange}.

If you want to use \gls{xindy} to process the glossary
files, you must make sure you have used the 
\pkgopt{xindy} package option:
\begin{verbatim}
\usepackage[xindy]{glossaries}
\end{verbatim}
This is required regardless of whether you use \gls{xindy}
explicitly or whether it's called implicitly via applications such
as \gls{makeglossaries} or \gls{makeglossariesgui}. This causes the glossary 
entries to be written in raw \gls*{xindy} format, so you need to
use \texttt{-I xindy} \emph{not} \texttt{-I tex}.

To run \gls{xindy} type the following in your terminal 
(all on one line):
\begin{prompt}
xindy -L \meta{language} -C \meta{encoding} -I xindy -M \meta{style} -t \meta{base}.glg -o \meta{base}.gls \meta{base}.glo
\end{prompt}
where \meta{language} is the required language name, \meta{encoding}
is the encoding, \meta{base} is the name of the document without the
\filetype{.tex} extension and \meta{style} is the name of the
\gls{xindy} style file without the \filetype{.xdy} extension.
The default name for this style file is \meta{base}\filetype{.xdy}
but can be changed via \ics{setStyleFile}\marg{style}. You may need
to specify the full path name depending on the current working
directory. If any of the file names contain spaces, you must delimit
them using double-quotes.

For example, if your document is called \texttt{myDoc.tex} and
you are using UTF8 encoding in English, then type the
following in your terminal:
\begin{prompt}
xindy -L english -C utf8 -I xindy -M myDoc -t myDoc.glg -o myDoc.gls myDoc.glo
\end{prompt}

Note that this just creates the main glossary. You need to do
the same for each of the other glossaries (including the
list of acronyms if you have used the \pkgopt{acronym}
package option), substituting \filetype{.glg}, \filetype{.gls}
and \filetype{.glo} with the relevant extensions. For example,
if you have used the \pkgopt{acronym} package option, then 
you would need to do:
\begin{prompt}
xindy -L english -C utf8 -I xindy -M myDoc -t myDoc.alg -o myDoc.acr myDoc.acn
\end{prompt}
For additional glossaries, the extensions are those supplied
when you created the glossary with \ics{newglossary}.

Note that if you use \gls{makeglossaries} instead, you can
replace all those calls to \gls{xindy} with just one call
to \gls*{makeglossaries}:
\begin{prompt}
makeglossaries myDoc
\end{prompt}
Note also that some commands and package options have no effect if 
you use \gls{xindy} explicitly instead of using 
\gls*{makeglossaries}. These are listed in 
\tableref{tab:makeglossariesCmds}.


\subsection[Using makeindex explicitly (Option~2)]{Using
\gls{makeindex} explicitly (\ifpdf \opt2\else Option 2\fi)}
\label{sec:makeindexapp}

If you want to use \gls{makeindex} explicitly, you must
make sure that you haven't used the \pkgopt{xindy} package
option or the glossary entries will be written in the wrong
format. To run \gls*{makeindex}, type the following in
your terminal:
\begin{prompt}
makeindex -s \meta{style}.ist -t \meta{base}.glg -o \meta{base}.gls \meta{base}.glo
\end{prompt}
where \meta{base} is the name of your document without the
\filetype{.tex} extension and \meta{style}\filetype{.ist} is the 
name of the \gls{makeindex} style file. By default, this is
\meta{base}\filetype{.ist}, but may be changed via
\ics{setStyleFile}\marg{style}. Note that there are other options, 
such as \texttt{-l} (letter ordering). See the \gls*{makeindex}
manual for further details.

For example, if your document is called \texttt{myDoc.tex}, then
type the following at the terminal:
\begin{prompt}
makeindex -s myDoc.ist -t myDoc.glg -o myDoc.gls myDoc.glo
\end{prompt}
Note that this only creates the main glossary. If you have
additional glossaries (for example, if you have used the
\pkgopt{acronym} package option) then you must call 
\gls{makeindex} for each glossary, substituting 
\filetype{.glg}, \filetype{.gls} and \filetype{.glo} with the
relevant extensions. For example, if you have used the
\pkgopt{acronym} package option, then you need to type the
following in your terminal:
\begin{prompt}
makeindex -s myDoc.ist -t myDoc.alg -o myDoc.acr myDoc.acn
\end{prompt}
For additional glossaries, the extensions are those supplied
when you created the glossary with \ics{newglossary}.

Note that if you use \gls{makeglossaries} instead, you can
replace all those calls to \gls{makeindex} with just one call
to \gls*{makeglossaries}:
\begin{prompt}
makeglossaries myDoc
\end{prompt}
Note also that some commands and package options have no effect if 
you use \gls*{makeindex} explicitly instead of using 
\gls{makeglossaries}. These are listed in 
\tableref{tab:makeglossariesCmds}.


\subsection{Note to Front-End and Script Developers}
\label{sec:notedev}

The information needed to determine whether to use \gls{xindy}
or \gls{makeindex} and the information needed to call those
applications is stored in the auxiliary file. This information can
be gathered by a front-end, editor or script to make the glossaries
where appropriate. This section describes how the information is
stored in the auxiliary file.

The file extensions used by each defined glossary are given by
\begin{definition}[\DescribeMacro{\@newglossary}]
\cs{@newglossary}\marg{label}\marg{log}\marg{out-ext}\marg{in-ext}
\end{definition}
where \meta{in-ext} is the extension of the
\emph{\gls{indexingapp}['s]} input file (the output file from the
\styfmt{glossaries} package's point of view), \meta{out-ext} is the
extension of the \emph{\gls*{indexingapp}['s]} output file (the
input file from the \styfmt{glossaries} package's point of view) and
\meta{log} is the extension of the \gls*{indexingapp}['s] transcript
file. The label for the glossary is also given for information
purposes only, but is not required by the \glspl*{indexingapp}. For
example, the information for the default main glossary is written
as:
\begin{verbatim}
\@newglossary{main}{glg}{gls}{glo}
\end{verbatim}

The \gls{indexingapp}['s] style file is specified by
\begin{definition}[\DescribeMacro{\@istfilename}]
\cs{@istfilename}\marg{filename}
\end{definition}
The file extension indicates whether to use \gls{makeindex}
(\filetype{.ist}) or \gls{xindy} (\filetype{.xdy}). Note that
the glossary information is formatted differently depending on
which \gls*{indexingapp} is supposed to be used, so it's 
important to call the correct one.

Word or letter ordering is specified by:
\begin{definition}[\DescribeMacro{\@glsorder}]
\cs{@glsorder}\marg{order}
\end{definition}
where \meta{order} can be either \texttt{word} or \texttt{letter}.

If \gls{xindy} should be used, the language and code page
for each glossary is specified by
\begin{definition}[\DescribeMacro{\@xdylanguage}\DescribeMacro{\@gls@codepage}]
\cs{@xdylanguage}\marg{label}\marg{language}\\
\cs{@gls@codepage}\marg{label}\marg{code}
\end{definition}
where \meta{label} identifies the glossary, \meta{language} is
the root language (e.g.\ \texttt{english}) and \meta{code}
is the encoding (e.g.\ \texttt{utf8}). These commands are omitted
if \gls{makeindex} should be used.

If \opt1 has been used, the \texttt{.aux} file will contain
\begin{definition}
\cs{@gls@reference}\marg{type}\marg{label}\marg{location}
\end{definition}
for every time an entry has been referenced.

\chapter{Package Options}
\label{sec:pkgopts}

This section describes the available \styfmt{glossaries} package
options. You may omit the \texttt{=true} for boolean options. (For
example, \pkgopt{acronym} is equivalent to \pkgopt[true]{acronym}).

\begin{important}
Note that \meta{key}=\meta{value} package options can't be passed via the
document class options. (This includes options where the
\meta{value} part may be omitted, such as \pkgopt{acronym}.) This is
a general limitation not restricted to the \styfmt{glossaries}
package. Options that aren't \meta{key}=\meta{value} (such as
\pkgopt{makeindex}) may be passed via the document class options.
\end{important}

\section{General Options}
\label{sec:pkgopts-general}

\begin{description}
\item[\pkgopt{nowarn}] This suppresses all warnings generated by
the \styfmt{glossaries} package. Don't use this option if you're new to using
\styfmt{glossaries} as the warnings are designed to help detect
common mistakes (such as forgetting to use \ics{makeglossaries}).

\item[\pkgopt{noredefwarn}] If you load \styfmt{glossaries} with
a~class or another package that already defines glossary related
commands, by default \styfmt{glossaries} will warn you that it's
redefining those commands. If you are aware of the consequences of
using \styfmt{glossaries} with that class or package and you don't
want to be warned about it, use this option to suppress those
warnings. Other warnings will still be issued unless you use the
\pkgopt{nowarn} option described above.

\item[\pkgopt{nomain}] This suppresses the creation of the main
glossary and associated \texttt{.glo} file, if unrequired. Note that
if you use this option, you must create another glossary in which to
put all your entries (either via the \pkgopt{acronym} (or
\pkgopt{acronyms}) package option described in
\sectionref{sec:pkgopts-acronym} or via the \pkgopt{symbols},
\pkgopt{numbers} or \pkgopt{index} options described in
\sectionref{sec:pkgopts-other} or via \ics{newglossary} described in
\sectionref{sec:newglossary}).

\begin{important}
If you don't use the main glossary and you don't use this option,
\gls{makeglossaries} will produce the following warning:
\begin{alltt}
Warning: File '\emph{filename}.glo' is empty.
Have you used any entries defined in glossary 
'main'?
Remember to use package option 'nomain' if 
you don't want to use the main glossary.
\end{alltt}
If you did actually want to use the main glossary and you see this
warning, check that you have referenced the entries in that glossary
via commands such as \ics{gls}.
\end{important}

\item[\pkgopt{sanitizesort}] This is a boolean option that
determines whether or not to \gls{sanitize} the sort value when
writing to the external glossary file. For example, suppose you
define an entry as follows:
\begin{verbatim}
\newglossaryentry{hash}{name={\#},sort={#},
 description={hash symbol}}
\end{verbatim}
The sort value (\verb|#|) must be sanitized before writing it to the
glossary file, otherwise \LaTeX\ will try to interpret it as a
parameter reference. If, on the other hand, you want the sort value
expanded, you need to switch off the sanitization. For example,
suppose you do:
\begin{verbatim}
\newcommand{\mysortvalue}{AAA}
\newglossaryentry{sample}{%
  name={sample},
  sort={\mysortvalue},
  description={an example}}
\end{verbatim}
and you actually want \cs{mysortvalue} expanded, so that the entry
is sorted according to \texttt{AAA}, then use the package option
\pkgopt[false]{sanitizesort}.

The default for \optsand23 is \pkgopt[true]{sanitizesort}, and the
default for \opt1 is \pkgopt[false]{sanitizesort}.

\item[\pkgopt{savewrites}] This is a boolean option to minimise the
number of write registers used by the \styfmt{glossaries} package.
(Default is \pkgopt[false]{savewrites}.) There are only a limited
number of write registers, and if you have a large number of
glossaries or if you are using a class or other packages that
create a lot of external files, you may exceed the maximum number
of available registers. If \pkgopt{savewrites} is set, the glossary
information will be stored in token registers until the end of the
document when they will be written to the external files. If you run
out of token registers, you can use \sty{etex}.

\begin{important}
This option can significantly slow document compilation. As an
alternative, you can use the \sty{scrwfile} package (part of the
KOMA-Script bundle) and not use this option.
\end{important}

You can also reduce the number of write registers by using
\opt1 or by ensuring you define all your glossary entries in the
preamble.

\begin{important}
If you want to use \TeX's \ics{write18} mechanism to call
\gls{makeindex} or \gls{xindy} from your document and use
\pkgopt{savewrites}, you must create the external files with 
\cs{glswritefiles} before you call \gls*{makeindex}/\gls*{xindy}. Also set
\cs{glswritefiles} to nothing or \cs{relax} before the end of the
document to avoid rewriting the files. For example:
\begin{verbatim}
\glswritefiles
\write18{makeindex -s \istfilename\space 
-t \jobname.glg -o \jobname.gls \jobname}
\let\glswritefiles\relax
\end{verbatim}
\end{important}

\item[\pkgopt{translate}] This can take the following values:

  \begin{description}
  \item[{\pkgopt[true]{translate}}] If \sty{babel} has been loaded
  and the \sty{translator} package is installed, \sty{translator}
  will be loaded and the translations will be provided by the
  \sty{translator} package interface. You can modify the 
  translations by providing your own dictionary. If the 
  \sty{translator} package isn't installed and \sty{babel} is
  loaded, the \sty{glossaries-babel} package will 
  be loaded and the translations will be provided using \styfmt{babel}'s
  \cs{addto}\cs{caption}\meta{language} mechanism. If 
  \sty{polyglossia} has been loaded, \sty{glossaries-polyglossia}
  will be loaded.

  \item[{\pkgopt[false]{translate}}] Don't provide translations, even
  if \sty{babel} or \sty{polyglossia} has been loaded.
  (Note that \sty{babel} provides the command \ics{glossaryname}
  so that will still be translated if you have loaded \sty{babel}.)

  \item[{\pkgopt[babel]{translate}}] Don't load the \sty{translator}
package. Instead load \sty{glossaries-babel}.

\begin{important}
I recommend you use \pkgopt[babel]{translate} if you have any
problems with the translations or with PDF bookmarks, but to maintain backward
compatibility, if \sty{babel} has been loaded the default is
\pkgopt[true]{translate}.
\end{important}

  \end{description}

  If \pkgopt{translate} is specified without a value,
  \pkgopt[true]{translate} is assumed. If \pkgopt{translate} isn't
specified, \pkgopt[true]{translate} is assumed if \sty{babel}, 
\sty{polyglossia} or \sty{translator} have been loaded. Otherwise
\pkgopt[false]{translate} is assumed.

See \sectionref{sec:fixednames} for further details.

\item[\pkgopt{notranslate}] This is equivalent to
\pkgopt[false]{translate} and may be passed via the document class
options.

\item[\pkgopt{nohypertypes}] Use this option if you have multiple
glossaries and you want to suppress the entry hyperlinks for a
particular glossary or glossaries. The value of this option should
be a comma-separated list of glossary types where \ics{gls} etc
shouldn't have hyperlinks by default. Make sure you enclose the
value in braces if it contains any commas. Example:
\begin{verbatim}
\usepackage[acronym,nohypertypes={acronym,notation}]
  {glossaries}
\newglossary[nlg]{notation}{not}{ntn}{Notation}
\end{verbatim}
The values must be fully expanded, so \textbf{don't} try
\texttt{nohypertypes\discretionary{}{}{}=\ics{acronymtype}}. You may also use
\begin{definition}
\ics{GlsDeclareNoHyperList}\marg{list}
\end{definition}
instead or additionally.
See \sectionref{sec:glslink} for further details.

\item[\pkgopt{hyperfirst}] This is a boolean option that specifies
whether each term has a hyperlink on \firstuse. The default is 
\pkgopt[true]{hyperfirst} (terms on \gls{firstuse} have a hyperlink, 
unless explicitly suppressed using starred versions of commands
such as \ics{gls*} or by identifying the glossary with 
\pkgopt{nohypertypes}, described above). Note that
\pkgopt{nohypertypes} overrides \pkgopt[true]{hyperfirst}.
This option only affects commands that check the \firstuseflag, such
as the \glslike\ commands (for example, \ics{gls} or
\ics{glsdisp}), but not the \glstextlike\ commands
(for example, \ics{glslink} or \ics{glstext}). 

The \pkgopt{hyperfirst} setting applies to
all glossary types (unless identified by \pkgopt{nohypertypes} or
defined with \ics{newignoredglossary}). It can be overridden on an
individual basis by explicitly setting the \gloskey[glslink]{hyper} key
when referencing an entry (or by using the plus or starred
version of the referencing command).

It may be that you only want to apply this to just the acronyms
(where the first use explains the meaning of the acronym) but not
for ordinary glossary entries (where the first use is identical to
subsequent uses). In this case, you can use \pkgopt[false]{hyperfirst} and
apply \cs{glsunsetall} to all the regular (non-acronym) glossaries.
For example:
\begin{verbatim}
 \usepackage[acronym,hyperfirst=false]{glossaries}
 % acronym and glossary entry definitions

 % at the end of the preamble
 \glsunsetall[main]
\end{verbatim}

Alternatively you can redefine the hook
\begin{definition}[\DescribeMacro\glslinkcheckfirsthyperhook]
\cs{glslinkcheckfirsthyperhook}
\end{definition}
which is used by the commands that check the \firstuseflag, such
as \ics{gls}. Within the definition of this command, you can use
\ics{glslabel} to reference the entry label and \ics{glstype} to
reference the glossary type. You can also use \ics{ifglsused}
to determine if the entry has been used. You can test if an entry is
an acronym by checking if it has the \gloskey{long} key set using
\ics{ifglshaslong}. For example, to switch off the hyperlink on
first use just for acronyms:
\begin{verbatim}
\renewcommand*{\glslinkcheckfirsthyperhook}{%
 \ifglsused{\glslabel}{}%
 {%
   \ifglshaslong{\glslabel}{\setkeys{glslink}{hyper=false}}{}%
 }%
}
\end{verbatim}

Note that this hook isn't used by the commands that don't check the
\firstuseflag, such as \ics{glstext}.

\item[\pkgopt{savenumberlist}] This is a boolean option that
specifies whether or not to gather and store the \gls{numberlist}
for each entry. The default is \pkgopt[false]{savenumberlist}. (See
\ics{glsentrynumberlist} and \ics{glsdisplaynumberlist} in
\sectionref{sec:glsnolink}.) This is always true if you
use \opt1.

\end{description}

\section{Sectioning, Headings and TOC Options}
\label{sec:pkgopts-sec}

\begin{description}
\item[\pkgopt{toc}] Add the glossaries to the table of contents.
Note that an extra \LaTeX\ run is required with this option.
Alternatively, you can switch this function on and off using
\begin{definition}[\DescribeMacro{\glstoctrue}]
\cs{glstoctrue}
\end{definition}
and
\begin{definition}[\DescribeMacro{\glstocfalse}]
\cs{glstocfalse}
\end{definition}

\item[\pkgopt{numberline}] When used with \pkgopt{toc}, this will
add \ics{numberline}\verb|{}| in the final argument of 
\ics{addcontentsline}. This will align the table of contents entry 
with the numbered section titles. Note that this option has no
effect if the \pkgopt{toc} option is omitted. If \pkgopt{toc} is
used without \pkgopt{numberline}, the title will be aligned with
the section numbers rather than the section titles.

\item[\pkgopt{section}] This is a \meta{key}=\meta{value} option.  Its
value should be the name of a sectional unit (e.g.\ chapter).
This will make the glossaries appear in the named sectional unit,
otherwise each glossary will appear in a chapter, if chapters
exist, otherwise in a section. Unnumbered sectional units will be used
by default. Example:
\begin{verbatim}
\usepackage[section=subsection]{glossaries}
\end{verbatim}
You can omit the value if you want to use sections, i.e.\
\begin{verbatim}
\usepackage[section]{glossaries}
\end{verbatim}
is equivalent to
\begin{verbatim}
\usepackage[section=section]{glossaries}
\end{verbatim}
You can change this value later in the document using
\begin{definition}[\DescribeMacro{\setglossarysection}]
\cs{setglossarysection}\marg{name}
\end{definition}
where \meta{name} is the sectional unit.

The start of each glossary adds information to the page header via
\begin{definition}[\DescribeMacro{\glsglossarymark}]
\cs{glsglossarymark}\marg{glossary title}
\end{definition}
By default this uses \cs{@mkboth}\footnote{unless \cls{memoir} is
loaded, which case it uses \ics{markboth}} but you may 
need to redefine it.
For example, to only change the right header:
\begin{verbatim}
\renewcommand{\glsglossarymark}[1]{\markright{#1}}
\end{verbatim}
or to prevent it from changing the headers:
\begin{verbatim}
\renewcommand{\glsglossarymark}[1]{}
\end{verbatim}
If you want \cs{glsglossarymark} to use \cs{MakeUppercase} in the header, use the
\pkgopt{ucmark} option described below.

Occasionally you may find that another package defines 
\linebreak\cs{cleardoublepage} when it is not required. This may cause an 
unwanted blank page to appear before each glossary. This can
be fixed by redefining \DescribeMacro\glsclearpage\cs{glsclearpage}:
\begin{verbatim}
\renewcommand*{\glsclearpage}{\clearpage}
\end{verbatim}

\item[\pkgopt{ucmark}] This is a boolean option (default:
\pkgopt[false]{ucmark}, unless \cls{memoir} has been loaded, in
which case it defaults to \pkgopt[true]{ucmark}). If
set, \ics{glsglossarymark} uses
\ics{MakeTextUppercase}\footnote{Actually it uses
\ics{mfirstucMakeUppercase} which is set to \sty{textcase}'s \cs{MakeTextUppercase}
by the
\styfmt{glossaries} package. This makes it consistent with
\ics{makefirstuc}. (The \sty{textcase} package is automatically
loaded by \styfmt{glossaries}.)}.
You can test whether this option has been set or not using
\begin{definition}[\DescribeMacro\ifglsucmark]
\cs{ifglsucmark} \meta{true part}\cs{else} \meta{false part}\cs{fi}
\end{definition}
For example:
\begin{verbatim}
\renewcommand{\glsglossarymark}[1]{%
  \ifglsucmark
    \markright{\MakeTextUppercase{#1}}%
  \else
    \markright{#1}%
  \fi}
\end{verbatim}
If \cls{memoir} has been loaded and \pkgopt{ucfirst} is set, then
\cls{memoir}'s \ics{memUChead} is used.

\item[\pkgopt{numberedsection}]
The glossaries are placed in unnumbered sectional units by default,
but this can be changed using \pkgopt{numberedsection}. This option can take
one of the following values:
\begin{itemize}
\item \pkgoptval{false}{numberedsection}: no number, i.e.\ use starred form
of sectioning command (e.g.\ \cs{chapter*} or \cs{section*});

\item \pkgoptval{nolabel}{numberedsection}:
use a numbered section, i.e.\ the unstarred form of sectioning
command (e.g.\ \cs{chapter} or \cs{section}), but 
the section not labelled;

\item \pkgoptval{autolabel}{numberedsection}: numbered with automatic
labelling. Each glossary uses the unstarred form of a sectioning
command (e.g.\ \cs{chapter} or \cs{section}) and is assigned a label
(via \ics{label}). The label is formed from 
\begin{definition}[\DescribeMacro{\glsautoprefix}]
\cs{glsautoprefix} \meta{type}
\end{definition}
where 
\meta{type} is the label identifying that glossary. The default
value of \cs{glsautoprefix} is empty.  For example, if you load \styfmt{glossaries}
using:
\begin{verbatim}
\usepackage[section,numberedsection=autolabel]
  {glossaries}
\end{verbatim}
then each glossary will appear in a numbered section, and can
be referenced using something like:
\begin{verbatim}
The main glossary is in section~\ref{main} and 
the list of acronyms is in section~\ref{acronym}.
\end{verbatim}
If you can't decide whether to have the acronyms in the main
glossary or a separate list of acronyms, you can use
\ics{acronymtype} which is set to \texttt{main} if the
\pkgopt{acronym} option is not used and is set to \texttt{acronym}
if the \pkgopt{acronym} option is used. For example:
\begin{verbatim}
The list of acronyms is in section~\ref{\acronymtype}.
\end{verbatim}
You can redefine the prefix if the default label clashes with
another label in your document.
For example:
\begin{verbatim}
\renewcommand*{\glsautoprefix}{glo:}
\end{verbatim}
will add \texttt{glo:} to the automatically generated label, so
you can then, for example, refer to the list of acronyms as follows:
\begin{verbatim}
The list of acronyms is in 
section~\ref{glo:\acronymtype}.
\end{verbatim}
Or, if you are undecided on a prefix:
\begin{verbatim}
The list of acronyms is in 
section~\ref{\glsautoprefix\acronymtype}.
\end{verbatim}

\item \pkgoptval{nameref}{numberedsection}: this is like
\pkgoptval{autolabel}{numberedsection} but uses an unnumbered
sectioning command (e.g.\ \ics{chapter*} or \ics{section*}). It's
designed for use with the \sty{nameref} package. For example:
\begin{verbatim}
\usepackage{nameref}
\usepackage[numberedsection=nameref]{glossaries}
\end{verbatim}
Now \verb|\nameref{main}| will display the (TOC) section title
associated with the \texttt{main} glossary. As above, you can
redefine \cs{glsautoprefix} to provide a prefix for the label.
\end{itemize}

\end{description}

\section{Glossary Appearance Options}
\label{sec:pkgopts-printglos}

\begin{description}
\item[\pkgopt{entrycounter}] This is a boolean option. (Default
is \pkgopt[false]{entrycounter}.) If set, each main (level~0)
glossary entry will be numbered when using the standard glossary
styles. This option creates the counter
\DescribeCounter{glossaryentry}\ctrfmt{glossaryentry}.

If you use this option, you can reference the entry number
within the document using
\begin{definition}[\DescribeMacro{\glsrefentry}]
\cs{glsrefentry}\marg{label}
\end{definition}
where \meta{label} is the label associated with that glossary entry.

\begin{important}
If you use \cs{glsrefentry}, you must run \LaTeX\ twice after
creating the glossary files using \gls{makeglossaries},
\gls{makeindex} or \gls{xindy} to ensure the cross-references are
up-to-date.
\end{important}

\item[\pkgopt{counterwithin}] This is a \meta{key}=\meta{value}
option where \meta{value} is the name of a counter. If used, this
option will automatically set \pkgopt[true]{entrycounter} and the
\ctr{glossaryentry} counter will be reset every time \meta{value} is
incremented.


\begin{important}
The \ctr{glossaryentry} counter isn't automatically reset at the
start of each glossary, except when glossary section numbering is on
and the counter used by \pkgopt{counterwithin} is the same as the
counter used in the glossary's sectioning command.
\end{important}

If you want the counter reset at the start of each glossary, you can
redefine \ics{glossarypreamble} to use
\begin{definition}[\DescribeMacro{\glsresetentrycounter}]
\cs{glsresetentrycounter}
\end{definition}
which sets \ctr{glossaryentry} to zero:
\begin{verbatim}
\renewcommand{\glossarypreamble}{%
  \glsresetentrycounter
}
\end{verbatim}
or if you are using \ics{setglossarypreamble}, add it to each
glossary preamble, as required. For example:
\begin{verbatim}
\setglossarypreamble[acronym]{%
  \glsresetentrycounter
  The preamble text here for the list of acronyms.
}
\setglossarypreamble{%
  \glsresetentrycounter
  The preamble text here for the main glossary.
}
\end{verbatim}

\item[\pkgopt{subentrycounter}] This is a boolean option. (Default
is \pkgopt[false]{subentrycounter}.) If set, each level~1
glossary entry will be numbered when using the standard glossary
styles. This option creates the counter
\DescribeCounter{glossarysubentry}\ctrfmt{glossarysubentry}.
The counter is reset with each main (level~0) entry. Note that this
package option is independent of \pkgopt{entrycounter}. You can
reference the number within the document using
\ics{glsrefentry}\marg{label} where \meta{label} is the label
associated with the sub-entry.

\item[\pkgopt{style}] This is a \meta{key}=\meta{value} option.
(Default is \pkgopt[list]{style}.) Its value should be the name of
the glossary style to use. This key may only be used for styles
defined in \sty{glossary-list}, \sty{glossary-long},
\sty{glossary-super} or \sty{glossary-tree}. Alternatively, you can
set the style using
\begin{definition}
\cs{setglossarystyle}\marg{style name}
\end{definition}
(See \sectionref{sec:styles} for further details.)

\item[\pkgopt{nolong}] This prevents the \styfmt{glossaries} package
from automatically loading \sty{glossary-long} (which means that
the \sty{longtable} package also won't be loaded). This reduces
overhead by not defining unwanted styles and commands. Note that if
you use this option, you won't be able to use any of the
glossary styles defined in the \styfmt{glossary-long} package (unless
you explicitly load \sty{glossary-long}).

\item[\pkgopt{nosuper}] This prevents the \styfmt{glossaries} package
from automatically loading \sty{glossary-super} (which means that
the \sty{supertabular} package also won't be loaded). This reduces
overhead by not defining unwanted styles and commands. Note that if
you use this option, you won't be able to use any of the
glossary styles defined in the \styfmt{glossary-super} package
(unless you explicitly load \sty{glossary-super}).

\item[\pkgopt{nolist}] This prevents the \styfmt{glossaries} package
from automatically loading \sty{glossary-list}. This reduces
overhead by not defining unwanted styles. Note that if
you use this option, you won't be able to use any of the
glossary styles defined in the \styfmt{glossary-list} package
(unless you explicitly load \sty{glossary-list}).
Note that since the default style is \glostyle{list}, you will 
also need to use the \pkgopt{style} option to set the style to
something else.

\item[\pkgopt{notree}] This prevents the \styfmt{glossaries} package
from automatically loading \sty{glossary-tree}. This reduces
overhead by not defining unwanted styles. Note that if
you use this option, you won't be able to use any of the
glossary styles defined in the \styfmt{glossary-tree} package
(unless you explicitly load \sty{glossary-tree}).

\item[\pkgopt{nostyles}] This prevents all the predefined styles
from being loaded. If you use this option, you need to load a
glossary style package (such as \sty{glossary-mcols}). Also if you
use this option, you can't use the \pkgopt{style} package option.
Instead you must either use \ics{setglossarystyle}\marg{style} or the
\gloskey[printglossary]{style} key in the optional argument to
\ics{printglossary}. Example:
\begin{verbatim}
\usepackage[nostyles]{glossaries}
\usepackage{glossary-mcols}
\setglossarystyle{mcoltree}
\end{verbatim}

\item[\pkgopt{nonumberlist}] This option will suppress the 
associated \glspl{numberlist} in the glossaries (see also 
\sectionref{sec:numberlists}).

\item[\pkgopt{seeautonumberlist}] If you suppress the
\glspl{numberlist} with \pkgopt{nonumberlist}, described above, this
will also suppress any cross-referencing information supplied by the
\gloskey{see} key in \ics{newglossaryentry} or \ics{glssee}. If you
use \pkgopt{seeautonumberlist}, the \gloskey{see} key will
automatically implement \gloskey{nonumberlist=false} for that entry.
(Note this doesn't affect \cs{glssee}.) For further details see
\sectionref{sec:crossref}.

\item[\pkgopt{counter}] This is a \meta{key}=\meta{value} option.
(Default is \pkgopt[page]{counter}.) The value should be the name of
the default counter to use in the \glspl{numberlist}
(see \sectionref{sec:numberlists}).

\item[\pkgopt{nopostdot}] This is a boolean option. If no value is
specified, \texttt{true} is assumed. When set to \texttt{true}, this 
option suppresses the default post description dot used by some of 
the predefined styles. The default setting is
\pkgopt[false]{nopostdot}.

\item[\pkgopt{nogroupskip}] This is a boolean option. If no value is
specified, \texttt{true} is assumed. When set to \texttt{true}, this
option suppresses the default vertical gap between groups used by
some of the predefined styles. The default setting is
\pkgopt[false]{nogroupskip}.

\end{description}

\section{Sorting Options}
\label{sec:pkgopts-sort}

\begin{description}
\item[\pkgopt{sort}] If you use \optsor23, this package option is
the only way of specifying how to sort the glossaries. Only \opt1 
allows you to specify sort methods for individual glossaries
via the \gloskey[printnoidxglossary]{sort} key in the optional
argument of \ics{printnoidxglossary}. If you have multiple
glossaries in your document and you are using \opt1, only use
the package options
\pkgopt[def]{sort} or \pkgopt[use]{sort} if you want to set this
sort method for \emph{all} your glossaries.

This is a \meta{key}=\meta{value} option where
\meta{value} may be one of the following:
\begin{itemize}
\item \pkgoptval{standard}{sort} : entries are sorted according to
the value of the \gloskey{sort} key used in \ics{newglossaryentry}
(if present) or the \gloskey{name} key (if \gloskey{sort} key is
missing);

\item \pkgoptval{def}{sort} : entries are sorted in the order in
which they were defined (the \gloskey{sort} key in
\cs{newglossaryentry} is ignored);

\item \pkgoptval{use}{sort} : entries are sorted according to the
order in which they are used in the document (the \gloskey{sort} key
in \cs{newglossaryentry} is ignored).

Both \pkgopt[def]{sort} and \pkgopt[use]{sort} set the sort key to a
six digit number via
\begin{definition}[\DescribeMacro\glssortnumberfmt]
\cs{glssortnumberfmt}\marg{number}
\end{definition}
(padded with leading zeros, where necessary). This can be 
redefined, if required, before the entries are defined (in the
case of \pkgopt[def]{sort}) or before the entries are used
(in the case of \pkgopt[use]{sort}).
\end{itemize}

The default is \pkgopt[standard]{sort}. When the standard sort
option is in use, you can hook into the sort mechanism by
redefining:
\begin{definition}[\DescribeMacro{\glsprestandardsort}]
\cs{glsprestandardsort}\marg{sort cs}\marg{type}\marg{label}
\end{definition}
where \meta{sort cs} is a temporary control sequence that stores the
sort value (which was either explicitly set via the \gloskey{sort}
key or implicitly set via the \gloskey{name} key) before any escaping of the
\gls{makeindex}\slash\gls{xindy} special characters is performed.
By default \cs{glsprestandardsort} just does:
\begin{definition}[\DescribeMacro{\glsdosanitizesort}]
\cs{glsdosanitizesort}
\end{definition}
which \gls{sanitize}[s] \meta{sort cs} if the \pkgopt{sanitizesort} package 
option is set (or does nothing if the package option 
\pkgopt[false]{sanitizesort} is used).

The other arguments, \meta{type} and \meta{label}, are the 
glossary type and the entry label for the current entry. Note that
\meta{type} will always be a control sequence, but \meta{label} will
be in the form used in the first argument of \ics{newglossaryentry}.

\begin{important}
Redefining \cs{glsprestandardsort} won't affect any entries that 
have already been defined and will have no effect at all if you 
are using \pkgopt[def]{sort} or \pkgopt[use]{sort}.
\end{important}

\begin{example}{Mixing Alphabetical and Order of Definition Sorting}{ex:diffsorts}
Suppose I have three glossaries: \texttt{main},
\texttt{acronym} and \texttt{notation}, and let's suppose
I want the \texttt{main} and \texttt{acronym} glossaries to be
sorted alphabetically, but the \texttt{notation} type should be
sorted in order of definition. 

For \opt1, I~just need to set the \gloskey[printnoidxglossary]{sort} key in 
the optional argument of \ics{printnoidxglossary}:
\begin{verbatim}
\printnoidxglossary[sort=word]
\printnoidxglossary[type=acronym,sort=word]
\printnoidxglossary[type=notation,sort=def]
\end{verbatim}

For \optsor23, I can set the sort to \texttt{standard}
(which is the default, but can be explicitly set via the
package option \pkgopt[standard]{sort}), and I can either define
all my \texttt{main} and \texttt{acronym} entries, then
redefine \cs{glsprestandardsort} to set \meta{sort cs} to
an incremented integer, and then define all my
\texttt{notation} entries. Alternatively, I can redefine
\cs{glsprestandardsort} to check for the glossary type and only
modify \meta{sort cs} if \meta{type} is \texttt{notation}.

The first option can be achieved as follows:
\begin{verbatim}
\newcounter{sortcount}

\renewcommand{\glsprestandardsort}[3]{%
  \stepcounter{sortcount}%
  \edef#1{\glssortnumberfmt{\arabic{sortcount}}}%
}
\end{verbatim}
The second option can be achieved as follows:
\begin{verbatim}
\newcounter{sortcount}

\renewcommand{\glsprestandardsort}[3]{%
  \ifdefstring{#2}{notation}%
  {%
     \stepcounter{sortcount}%
     \edef#1{\glssortnumberfmt{\arabic{sortcount}}}%
  }%
  {%
     \glsdosanitizesort
  }%
}
\end{verbatim}
(\cs{ifdefstring} is defined by the \sty{etoolbox} package.)
For a complete document, see the sample file \samplefile{Sort}.
\end{example}

\begin{example}{Customizing Standard Sort (Options 2 or 3)}{ex:customsort}
Suppose you want a glossary of people and you want the names listed
as \meta{first-name} \meta{surname} in the glossary, but you want the names
sorted by \meta{surname}, \meta{first-name}. You can do this by
defining a command called, say,
\cs{name}\marg{first-name}\marg{surname} that you can use in the
\gloskey{name} key when you define the entry, but hook into the
standard sort mechanism to temporarily redefine \cs{name} while the
sort value is being set.

First, define two commands to set the person's name:
\begin{verbatim}
\newcommand{\sortname}[2]{#2, #1}
\newcommand{\textname}[2]{#1 #2}
\end{verbatim}
and \cs{name} needs to be initialised to \cs{textname}:
\begin{verbatim}
\let\name\textname
\end{verbatim}
Now redefine \cs{glsprestandardsort} so that it temporarily sets
\cs{name} to \cs{sortname} and expands the sort value, then sets
\cs{name} to \cs{textname} so that the person's name appears as
\meta{first-name} \meta{surname} in the text:
\begin{verbatim}
\renewcommand{\glsprestandardsort}[3]{%
 \let\name\sortname
 \edef#1{\expandafter\expandonce\expandafter{#1}}%
 \let\name\textname
 \glsdosanitizesort
}
\end{verbatim}
(The somewhat complicate use of \cs{expandafter} etc helps to
protect fragile commands, but care is still needed.)

Now the entries can be defined:
\begin{verbatim}
\newglossaryentry{joebloggs}{name={\name{Joe}{Bloggs}},
  description={some information about Joe Bloggs}}

\newglossaryentry{johnsmith}{name={\name{John}{Smith}},
  description={some information about John Smith}}
\end{verbatim}
For a complete document, see the sample file \samplefile{People}.
\end{example}

\item[\pkgopt{order}] This may take two values: 
\pkgoptval{word}{order} or \pkgoptval{letter}{order}. The default
is word ordering.

\begin{important}
Note that the \pkgopt{order} option has no effect if you don't use 
\gls{makeglossaries}.
\end{important}

If you use \opt1, this setting will be used if you use 
\gloskey[printnoidxglossary]{sort}\texttt{=standard} in 
the optional argument of \ics{printnoidxglossary}:
\begin{verbatim}
\printnoidxglossary[sort=standard]
\end{verbatim}
Alternatively, you can specify the order for individual glossaries:
\begin{verbatim}
\printnoidxglossary[sort=word]
\printnoidxglossary[type=acronym,sort=letter]
\end{verbatim}

\item[\pkgopt{makeindex}] (\opt2) The glossary information and
indexing style file will be written in \gls{makeindex} format. If
you use \gls{makeglossaries}, it will automatically detect that it
needs to call \gls*{makeindex}. If you don't use
\gls*{makeglossaries}, you need to remember to use \gls*{makeindex}
not \gls{xindy}. The indexing style file will been given 
a~\filetype{.ist} extension.

You may omit this package option if you are using \opt2 as this is the
default. It's available in case you need to override the effect of an earlier
occurrence of \pkgopt{xindy} in the package option list.

\item[\pkgopt{xindy}] (\opt3) The glossary information and indexing style
file will be written in \gls{xindy} format. If you use
\gls{makeglossaries}, it will automatically detect that it needs to
call \gls*{xindy}. If you don't use \gls*{makeglossaries}, you need to
remember to use \gls*{xindy} not \gls{makeindex}. The indexing style
file will been given a \filetype{.xdy} extension.

This package option may additionally have a value that
is a \meta{key}=\meta{value} comma-separated list to override the
language and codepage. For example:
\begin{verbatim}
\usepackage[xindy={language=english,codepage=utf8}]
  {glossaries}
\end{verbatim}
You can also specify whether you want a number group in the
glossary. This defaults to true, but can be suppressed. For example:
\begin{verbatim}
\usepackage[xindy={glsnumbers=false}]{glossaries}
\end{verbatim}
If no value is supplied to this package option (either simply
writing \texttt{xindy} or writing \verb|xindy={}|) then the
language, codepage and number group settings are unchanged.  See
\sectionref{sec:xindy} for further details on using \gls{xindy} with
the \styfmt{glossaries} package.

\item[\pkgopt{xindygloss}]  (\opt3) This is equivalent to
\verb|xindy={}| (that is, the \pkgopt{xindy} option without any value
supplied) and may be used as a document class option. The language
and code page can be set via \ics{GlsSetXdyLanguage} and
\ics{GlsSetXdyCodePage} (see \sectionref{sec:langenc}.)

\item[\pkgopt{xindynoglsnumbers}] (\opt3) This is equivalent to
\verb|xindy={glsnumbers=false}| and may be used as a document class
option.

\item[\pkgopt{automake}] This is a boolean option (new to version
4.08) that will attempt to run \gls{makeindex} or \gls{xindy} using
\TeX's \ics{write18} mechanism at the end of the document.  Since
this mechanism can be a security risk, some \TeX\ distributions
disable it completely, in which case this option won't have an
effect. (If this option doesn't appear to work, search the log file 
for \qt{runsystem} and see if it is followed by
\qt{enabled} or \qt{disabled}.)

Some distributions allow \cs{write18} in a restricted mode.
This mode has a~limited number of trusted applications, which
usually includes \gls{makeindex} but may not include \gls{xindy}. So
if you have the restricted mode on, \pkgopt{automake} should work
with \gls*{makeindex} but may not work with \gls{xindy}.

However even in unrestricted mode this option may not work with
\gls*{xindy} as \gls*{xindy} uses language names that don't always
correspond with \ics{babel}'s language names. (The
\gls{makeglossaries} script applies mappings to assist you.)
Note that you still need at least two \LaTeX\ runs to ensure the document
is up-to-date with this setting.

Since this package option attempts to run the \gls{indexingapp} on
every \LaTeX\ run, its use should be considered a last resort for
those who can't work out how to incorporate the \gls*{indexingapp}
into their document build. The default value for this option is
\pkgopt[false]{automake}.

\end{description}

\section{Acronym Options}
\label{sec:pkgopts-acronym}

\begin{description}
\item[\pkgopt{acronym}] This creates a new glossary with the
label \texttt{acronym}. This is equivalent to:
\begin{verbatim}
\newglossary[alg]{acronym}{acr}{acn}{\acronymname}
\end{verbatim}
It will also define
\begin{definition}[\DescribeMacro\printacronyms]
\cs{printacronyms}\oarg{options}
\end{definition}
that's equivalent to
\begin{alltt}
\ics{printglossary}[type=acronym,\meta{options}]
\end{alltt}
(unless that command is already defined before the beginning of the
document or the package option \pkgopt{compatible-3.07} is used).

If you are using \opt1, you need to use
\begin{alltt}
\ics{printnoidxglossary}[type=acronym,\meta{options}]
\end{alltt}
to display the list of acronyms.

If the \pkgopt{acronym} package option is used, \ics{acronymtype}
is set to \texttt{acronym} otherwise it is set to 
\texttt{main}.\footnote{Actually it sets \ics{acronymtype} to
\ics{glsdefaulttype} if the \pkgopt{acronym} package option is
not used, but \ics{glsdefaulttype} usually has the value
\texttt{main} unless the \pkgopt{nomain} option has been used.} 
Entries that are defined using \ics{newacronym} are placed in
the glossary whose label is given by \ics{acronymtype}, unless
another glossary is explicitly specified.

\begin{important}
Remember to use the \pkgopt{nomain} package option if you're only
interested in using this \texttt{acronym} glossary.
\end{important}

\item[\pkgopt{acronyms}] This is equivalent to
\pkgopt[true]{acronym} and may be used in the document class option
list.

\item[\pkgopt{acronymlists}] By default, only the \cs{acronymtype}
glossary is considered to be a list of acronyms. If you have other
lists of acronyms, you can specify them as a comma-separated list
in the value of \pkgopt{acronymlists}. For example, if you use the
\pkgopt{acronym} package option but you also want the \texttt{main} 
glossary to also contain a list of acronyms, you can do:
\begin{verbatim}
\usepackage[acronym,acronymlists={main}]{glossaries}
\end{verbatim}
No check is performed to determine if the listed glossaries exist,
so you can add glossaries you haven't defined yet. For example:
\begin{verbatim}
\usepackage[acronym,acronymlists={main,acronym2}]
  {glossaries}
\newglossary[alg2]{acronym2}{acr2}{acn2}%
  {Statistical Acronyms}
\end{verbatim}
You can use
\begin{definition}[\DescribeMacro\DeclareAcronymList]
\cs{DeclareAcronymList}\marg{list}
\end{definition}
instead of or in addition to the \pkgopt{acronymlists} option.
This will add the glossaries given in \meta{list} to the list of glossaries
that are identified as lists of acronyms. To replace the list of acronym lists
with a new list use:
\begin{definition}[\DescribeMacro\SetAcronymLists]
\cs{SetAcronymLists}\marg{list}
\end{definition}

You can determine if a glossary has been identified as being a list
of acronyms using:
\begin{definition}[\DescribeMacro\glsIfListOfAcronyms]
\cs{glsIfListOfAcronyms}\marg{label}\marg{true part}\marg{false part}
\end{definition}

\item[\pkgopt{shortcuts}] This option provides shortcut commands
for acronyms.  See \sectionref{sec:acronyms} for further details.
Alternatively you can use:
\begin{definition}[\DescribeMacro\DefineAcronymShortcuts]
\cs{DefineAcronymShortcuts}
\end{definition}
\end{description}

\subsection{Deprecated Acronym Style Options}
\label{sec:pkgopts-old-acronym}

The package options listed in this section are now deprecated but
are kept for backward-compatibility. Use \ics{setacronymstyle}
instead. See \sectionref{sec:acronyms} for further details.

\begin{description}
\item[\pkgopt{description}] This option changes the definition of
\ics{newacronym} to allow a description.
This option may be replaced by
\begin{verbatim}
\setacronymstyle{long-short-desc}
\end{verbatim}
or (with \pkgopt{smallcaps})
\begin{verbatim}
\setacronymstyle{long-sc-short-desc}
\end{verbatim}
or (with \pkgopt{smaller})
\begin{verbatim}
\setacronymstyle{long-sm-short-desc}
\end{verbatim}
or (with \pkgopt{footnote})
\begin{verbatim}
\setacronymstyle{footnote-desc}
\end{verbatim}
or (with \pkgopt{footnote} and \pkgopt{smallcaps})
\begin{verbatim}
\setacronymstyle{footnote-sc-desc}
\end{verbatim}
or (with \pkgopt{footnote} and \pkgopt{smaller})
\begin{verbatim}
\setacronymstyle{footnote-sm-desc}
\end{verbatim}
or (with \pkgopt{dua})
\begin{verbatim}
\setacronymstyle{dua-desc}
\end{verbatim}

\item[\pkgopt{smallcaps}] This option changes the definition of
\ics{newacronym} and the way that acronyms are displayed. 
This option may be replaced by:
\begin{verbatim}
\setacronymstyle{long-sc-short}
\end{verbatim}
or (with \pkgopt{description})
\begin{verbatim}
\setacronymstyle{long-sc-short-desc}
\end{verbatim}
or (with \pkgopt{description} and \pkgopt{footnote})
\begin{verbatim}
\setacronymstyle{footnote-sc-desc}
\end{verbatim}

\item[\pkgopt{smaller}] This option changes the definition of
\ics{newacronym} and the way that acronyms are displayed.
\begin{important}
If you use this option, you will need to include the \sty{relsize}
package or otherwise define \ics{textsmaller} or redefine
\ics{acronymfont}.
\end{important}
This option may be replaced by:
\begin{verbatim}
\setacronymstyle{long-sm-short}
\end{verbatim}
or (with \pkgopt{description})
\begin{verbatim}
\setacronymstyle{long-sm-short-desc}
\end{verbatim}
or (with \pkgopt{description} and \pkgopt{footnote})
\begin{verbatim}
\setacronymstyle{footnote-sm-desc}
\end{verbatim}

\item[\pkgopt{footnote}] This option changes the definition of
\ics{newacronym} and the way that acronyms are displayed.
This option may be replaced by:
\begin{verbatim}
\setacronymstyle{footnote}
\end{verbatim}
or (with \pkgopt{smallcaps})
\begin{verbatim}
\setacronymstyle{footnote-sc}
\end{verbatim}
or (with \pkgopt{smaller})
\begin{verbatim}
\setacronymstyle{footnote-sm}
\end{verbatim}
or (with \pkgopt{description})
\begin{verbatim}
\setacronymstyle{footnote-desc}
\end{verbatim}
or (with \pkgopt{smallcaps} and \pkgopt{description})
\begin{verbatim}
\setacronymstyle{footnote-sc-desc}
\end{verbatim}
or (with \pkgopt{smaller} and \pkgopt{description})
\begin{verbatim}
\setacronymstyle{footnote-sm-desc}
\end{verbatim}


\item[\pkgopt{dua}] This option changes the definition of
\ics{newacronym} so that acronyms are always expanded. 
This option may be replaced by:
\begin{verbatim}
\setacronymstyle{dua}
\end{verbatim}
or (with \pkgopt{description})
\begin{verbatim}
\setacronymstyle{dua-desc}
\end{verbatim}

\end{description}

\section{Other Options}
\label{sec:pkgopts-other}

Other available options that don't fit any of the above categories
are:
\begin{description}
\item[\pkgopt{symbols}] This option defines a new glossary type with
the label \texttt{symbols} via
\begin{verbatim}
\newglossary[slg]{symbols}{sls}{slo}{\glssymbolsgroupname}
\end{verbatim}
It also defines
\begin{definition}[\DescribeMacro\printsymbols]
\cs{printsymbols}\oarg{options}
\end{definition}
which is a synonym for
\begin{alltt}
\ics{printglossary}[type=symbols,\meta{options}]
\end{alltt}

If you use \opt1, you need to use:
\begin{alltt}
\ics{printnoidxglossary}[type=symbols,\meta{options}]
\end{alltt}
to display the list of symbols.

\begin{important}
Remember to use the \pkgopt{nomain} package option if you're only
interested in using this \texttt{symbols} glossary.
\end{important}

\item[\pkgopt{numbers}] This option defines a new glossary type with
the label \texttt{numbers} via
\begin{verbatim}
\newglossary[nlg]{numbers}{nls}{nlo}{\glsnumbersgroupname}
\end{verbatim}
It also defines
\begin{definition}[\DescribeMacro\printnumbers]
\cs{printnumbers}\oarg{options}
\end{definition}
which is a synonym for
\begin{alltt}
\ics{printglossary}[type=numbers,\meta{options}]
\end{alltt}

If you use \opt1, you need to use:
\begin{alltt}
\ics{printnoidxglossary}[type=numbers,\meta{options}]
\end{alltt}
to display the list of numbers.

\begin{important}
Remember to use the \pkgopt{nomain} package option if you're only
interested in using this \texttt{numbers} glossary.
\end{important}

\item[\pkgopt{index}] This option defines a new glossary type with
the label \texttt{index} via
\begin{verbatim}
\newglossary[ilg]{index}{ind}{idx}{\indexname}%
\end{verbatim}
It also defines
\begin{definition}[\DescribeMacro\newterm]
\cs{newterm}\oarg{options}\marg{term}
\end{definition}
which is a synonym for
\begin{alltt}
\ics{newglossaryentry}\marg{term}[type=index,name=\marg{term},%
description=\ics{nopostdesc},\meta{options}]
\end{alltt}
and
\begin{definition}[\DescribeMacro\printindex]
\cs{printindex}\oarg{options}
\end{definition}
which is a synonym for
\begin{alltt}
\ics{printglossary}[type=index,\meta{options}]
\end{alltt}

If you use \opt1, you need to use:
\begin{alltt}
\ics{printnoidxglossary}[type=index,\meta{options}]
\end{alltt}
to display this glossary.

\begin{important}
Remember to use the \pkgopt{nomain} package option if you're only
interested in using this \texttt{index} glossary. Note that you
can't mix this option with \ics{index}. Either use
\styfmt{glossaries} for the indexing or use a~custom indexing
package, such as \sty{makeidx}, \sty{index} or \sty{imakeidx}.
(You can, of course, load one of those packages and 
load \styfmt{glossaries} without the \pkgopt{index} package option.)
\end{important}

Since the index isn't designed for terms with descriptions, you
might also want to disable the hyperlinks for this glossary using
the package option \pkgopt[index]{nohypertypes} or the command
\begin{display}
\ics{GlsDeclareNoHyperList}\verb|{index}|
\end{display}

The example file \samplefile{-index} illustrates the use of the
\pkgopt{index} package option.

\item[\pkgopt{compatible-2.07}] Compatibility mode for old documents
created using version 2.07 or below.

\item[\pkgopt{compatible-3.07}] Compatibility mode for old documents
created using version 3.07 or below.

\end{description}

\section{Setting Options After the Package is Loaded}
\label{sec:setupglossaries}

Some of the options described above may also be set after the
\styfmt{glossaries} package has been loaded using
\begin{definition}[\DescribeMacro{\setupglossaries}]
\cs{setupglossaries}\marg{key-val list}
\end{definition}
The following package options \textbf{can't} be used in
\cs{setupglossaries}: \pkgopt{xindy}, \pkgopt{xindygloss},
\pkgopt{xindynoglsnumbers}, \pkgopt{makeindex},
\pkgopt{nolong}, \pkgopt{nosuper}, \pkgopt{nolist},
\pkgopt{notree}, \pkgopt{nostyles}, \pkgopt{nomain},
\pkgopt{compatible-2.07}, \pkgopt{translate}, \pkgopt{notranslate}, 
\pkgopt{acronym}. These options have to be set while the package is
loading, except for the \pkgopt{xindy} sub-options which can be set
using commands like \ics{GlsSetXdyLanguage} (see
\sectionref{sec:xindy} for further details).

\begin{important}
If you need to use this command, use it as soon as
possible after loading \styfmt{glossaries} otherwise you might 
end up using it too late for the change to take effect. For example, 
if you try changing the acronym styles (such as \pkgopt{smallcaps}) 
after you have started defining your acronyms, you are likely to get 
unexpected results. If you try changing the sort option after you have 
started to define entries, you may get unexpected results.
\end{important}

\chapter{Setting Up}
\label{sec:setup}

In the preamble you need to indicate whether you want to use \opt1,
\opt2 or \opt3. It's not possible to mix these options within
a~document.

\section[Option 1]{\ifpdf\opt1\else Option 1\fi}
\label{sec:setupopt1}

The command
\begin{definition}[\DescribeMacro{\makenoidxglossaries}]
\cs{makenoidxglossaries}
\end{definition}
must be placed in the preamble. This sets up the internal commands
required to make \opt1 work.
\textbf{If you omit \cs{makenoidxglossaries} none of
the glossaries will be displayed.}

\section[Options 2 and 3]{\ifpdf\optsand23\else Options 2 and 3\fi}
\label{sec:setupopt23}

The command
\begin{definition}[\DescribeMacro{\makeglossaries}]
\cs{makeglossaries}
\end{definition}
must be placed in the preamble in order to create the customised
\gls{makeindex} (\filetype{.ist}) or \gls{xindy} (\filetype{.xdy})
style file (for \optsor23, respectively) and to ensure that glossary 
entries are written to the appropriate output files. 
\textbf{If you omit \cs{makeglossaries} none of
the glossary files will be created.}

\begin{important}
Note that some of the commands provided by the \styfmt{glossaries}
package must not be used after \cs{makeglossaries} as they are
required when creating the customised style file. If you attempt
to use those commands after \cs{makeglossaries} you will generate
an error.

Similarly, there are some commands that must not be used before
\cs{makeglossaries}.
\end{important}

You can suppress the creation of the customised \gls{xindy}
or \gls{makeindex} style file using
\begin{definition}[\DescribeMacro{\noist}]
\cs{noist}
\end{definition}
That this command must not be used after \cs{makeglossaries}.
\begin{important}
Note that if you have a custom \filetype{.xdy} file created when using 
\styfmt{glossaries} version 2.07 or below, you will need to use the
\pkgopt{compatible-2.07} package option with it.
\end{important}

The default name for the customised style file is given by
\ics{jobname}\filetype{.ist} (\opt2) or 
\ics{jobname}\filetype{.xdy} (\opt3). This name may be
changed using:
\begin{definition}[\DescribeMacro{\setStyleFile}]
\cs{setStyleFile}\marg{name}
\end{definition}
where \meta{name} is the name of the style file without the 
extension. Note that this command must not be used after
\cs{makeglossaries}.

Each glossary entry is assigned a \gls{numberlist} that lists all 
the locations in the document where that entry was used. By default,
the location refers to the page number but this may be overridden
using the \pkgopt{counter} package option. The default form of
the location number assumes a full stop compositor (e.g.\ 1.2),
but if your location numbers use a different compositor (e.g. 1-2)
you need to set this using
\begin{definition}[\DescribeMacro{\glsSetCompositor}]
\cs{glsSetCompositor}\marg{symbol}
\end{definition}
For example:
\begin{verbatim}
\glsSetCompositor{-}
\end{verbatim}
This command must not be used after \cs{makeglossaries}.

If you use \opt3, you can have a different compositor for page
numbers starting with an upper case alphabetical character using:
\begin{definition}[\DescribeMacro{\glsSetAlphaCompositor}]
\cs{glsSetAlphaCompositor}\marg{symbol}
\end{definition}
This command has no effect if you use \opt2. For example, if you want 
\glspl{numberlist} containing a mixture of A-1 and 2.3 style
formats, then do:
\begin{verbatim}
\glsSetCompositor{.}\glsSetAlphaCompositor{-}
\end{verbatim}
See \sectionref{sec:numberlists} for further information about
\glspl{numberlist}.

\chapter{Defining Glossary Entries}
\label{sec:newglosentry}

All glossary entries must be defined before they are used, so it is
better to define them in the preamble to ensure this. In fact, some
commands such as \cs{longnewglossaryentry}
may only be used in the preamble. See \sectionref{sec:docdefs} for
a discussion of the problems with defining entries within the
document instead of in the preamble.
\begin{important}
\opt1 enforces the preamble-only restriction on
\ics{newglossaryentry}.
\end{important}

Only those entries that are referenced in the document 
(using any of the commands described in
\sectionref{sec:glslink}, \sectionref{sec:glsadd} or
\sectionref{sec:crossref}) will appear in the glossary. See
\sectionref{sec:printglossary} to find out how to display the
glossary.


New glossary entries are defined using the command:
\begin{definition}[\DescribeMacro{\newglossaryentry}]
\cs{newglossaryentry}\marg{label}\marg{key=value list}
\end{definition}
This is a short command, so values in \meta{key-val list} can't
contain any paragraph breaks. If you have a long description that
needs to span multiple paragraphs, use
\begin{definition}[\DescribeMacro{\longnewglossaryentry}]
\cs{longnewglossaryentry}\marg{label}\marg{key=value
list}\marg{long description}
\end{definition}
instead. Note that this command may only be used in the preamble.
Be careful of unwanted spaces. \cs{longnewglossaryentry} 
will remove trailing spaces in the description (via \cs{unskip})
but won't remove leading spaces (otherwise it would interfere with
commands like \cs{Glsentrydesc}).

There are also commands that will only define the entry if it
hasn't already been defined:
\begin{definition}[\DescribeMacro\provideglossaryentry]
\cs{provideglossaryentry}\marg{label}\marg{key=value list}
\end{definition}
and\par
\DescribeMacro\longprovideglossaryentry
\begin{definition}
\cs{longprovideglossaryentry}\marg{label}\marg{key=value
list}\marg{long description}
\end{definition}
(These are both preamble-only commands.)

For all the above commands, the first argument, \meta{label}, must be 
a~unique label with which to identify this entry. This can't contain
any non-expandable commands or active characters.

\begin{important}
Note that although an \gls{exlatinchar} or other \gls{nonlatinchar}, 
such as \'e or \ss, looks like a plain character in your
\texttt{.tex} file, it's actually a~macro (an active character) and
therefore can't be used in the label. Also be careful of
\sty{babel}'s options that change certain punctuation characters
(such as \texttt{:} or \texttt{-}) to active characters.
\end{important}

The second argument, \meta{key=value list}, is
a \meta{key}=\meta{value} list that supplies the relevant
information about this entry. There are two required fields:
\gloskey{description} and either \gloskey{name} or \gloskey{parent}.
The description is set in the third argument of
\cs{longnewglossaryentry} and \cs{longprovideglossaryentry}. With
the other commands it's set via the \gloskey{description} key.
Available fields are listed below:

\begin{description}
\item[{\gloskey{name}}] The name of the entry (as it will appear in 
the glossary). If this key is omitted and the \gloskey{parent}
key is supplied, this value will be the same as the parent's name.

\item[{\gloskey{description}}] A brief description of this term (to
appear in the glossary). Within this value, you can use 
\begin{definition}[\DescribeMacro{\nopostdesc}]
\cs{nopostdesc}
\end{definition}
to suppress the
description terminator for this entry. For example, if this
entry is a parent entry that doesn't require a description, you
can do \verb|description={\nopostdesc}|. If you want a paragraph
break in the description use
\begin{definition}[\DescribeMacro{\glspar}]
\cs{glspar}
\end{definition}
or, better, use \cs{longnewglossaryentry}.
However, note that not all glossary styles support multi-line
descriptions. If you are using one of the tabular-like glossary
styles that permit multi-line descriptions, use \ics{newline}
not \verb"\\" if you want to force a line break.

\item[{\gloskey{parent}}] The label of the parent entry. Note that
the parent entry must be defined before its sub-entries.
See \sectionref{sec:subentries} for further details.

\item[{\gloskey{descriptionplural}}] The plural form of the
description, if required. If omitted, the value
is set to the same as the \gloskey{description} key.

\item[{\gloskey{text}}] How this entry will appear in the document text
when using \ics{gls} (or one of its upper case variants). If this
field is omitted, the value of the \gloskey{name} key is used.

\item[{\gloskey{first}}] How the entry will appear in the document text
on \firstuse\ with \ics{gls} (or one of its upper case
variants). If this field is omitted, the value of the \gloskey{text}
key is used. Note that if you use \ics{glspl}, \ics{Glspl},
\ics{GLSpl}, \ics{glsdisp} before using \ics{gls}, the
\gloskey{firstplural} value won't be used with \ics{gls}.

\item[{\gloskey{plural}}] How the entry will appear in the document text
when using \ics{glspl} (or one of its upper case variants).  If this
field is omitted, the value is obtained by appending
\ics{glspluralsuffix} to the value of the \gloskey{text} field.  The
default value of \ics{glspluralsuffix} is the letter \qt{s}.

\item[{\gloskey{firstplural}}] How the entry will appear in the
document text on \firstuse\ with \ics{glspl} (or one of its
upper case variants). If this field is omitted, the value is obtained
from the \gloskey{plural} key, if the \gloskey{first} key is omitted,
or by appending \ics{glspluralsuffix} to the value of the
\gloskey{first} field, if the \gloskey{first} field is present. Note
that if you use \ics{gls}, \ics{Gls}, \ics{GLS}, \cs{glsdisp} before
using \ics{glspl}, the \gloskey{firstplural} value won't be used
with \ics{glspl}.

\textbf{Note:} prior to version 1.13, the default value of 
\gloskey{firstplural} was always taken by appending \qt{s} to the
\gloskey{first} key, which meant that you had to specify both
\gloskey{plural} and \gloskey{firstplural}, even if you hadn't
used the \gloskey{first} key.

\item[{\gloskey{symbol}}] This field is provided to allow the user
to specify an associated symbol. If omitted, the value is set to
\cs{relax}. Note that not all glossary styles display the symbol.

\item[{\gloskey{symbolplural}}] This is the plural form of the
symbol (as passed to \ics{glsdisplay} and \ics{glsdisplayfirst}
by \ics{glspl}, \ics{Glspl} and \ics{GLSpl}). If omitted, the value
is set to the same as the \gloskey{symbol} key.

\item[{\gloskey{sort}}] This value indicates how this entry should
be sorted. If omitted, the value is given
by the \gloskey{name} field unless one of the package options
\pkgopt[def]{sort} and \pkgopt[use]{sort} have been used. In
general, it's best to use the \gloskey{sort} key if the
\gloskey{name} contains commands (e.g.\ \verb|\ensuremath{\alpha}|).
You can also override the \gloskey{sort} key by redefining
\ics{glsprestandardsort} (see \sectionref{sec:pkgopts-sort}).

\opt1 by default strips the \glslink{latexexlatinchar}{standard
\LaTeX\ accents} (that is, accents generated by core \LaTeX\ commands) from the
\gloskey{name} key when it sets the \gloskey{sort} key. So with
\opt1:
\begin{verbatim}
\newglossaryentry{elite}{%
  name={{\'e}lite},
  description={select group of people}
}
\end{verbatim}
This is equivalent to:
\begin{verbatim}
\newglossaryentry{elite}{%
  name={{\'e}lite},
  description={select group of people},
  sort={elite}
}
\end{verbatim}
Unless you use the package option \pkgopt[true]{sanitizesort}, in
which case it's equivalent to:
\begin{verbatim}
\newglossaryentry{elite}{%
  name={{\'e}lite},
  description={select group of people},
  sort={\'elite}
}
\end{verbatim}
This will place the entry before the \qt{A} letter group since the
sort value starts with a symbol.

Similarly if you use the \sty{inputenc} package:
\begin{alltt}
\verb|\newglossaryentry{elite}{%|
  name=\verb|{{|\'e\verb|}|lite\verb|}|,
  description=\verb|{select group of people}|
\verb|}|
\end{alltt}
This is equivalent to
\begin{alltt}
\verb|\newglossaryentry{elite}{%|
  name=\verb|{{|\'e\verb|}|lite\verb|}|,
  description=\verb|{select group of people}|,
  sort=elite
\verb|}|
\end{alltt}
Unless you use the package option \pkgopt[true]{sanitizesort}, in
which case it's equivalent to:
\begin{alltt}
\verb|\newglossaryentry{elite}{%|
  name=\verb|{{|\'e\verb|}|lite\verb|}|,
  description=\verb|{select group of people}|,
  sort=\'elite
\verb|}|
\end{alltt}
Again, this will place the entry before the \qt{A} group.

With \optsand23, the default value of \gloskey{sort} will either be
set to the \gloskey{name} key (if \pkgopt[true]{sanitizesort}) or it
will set it to the expansion of the \gloskey{name} key (if
\pkgopt[false]{sanitizesort}).

\begin{important}
Take care with \gls{xindy} (\opt3): if you have entries with the same
\gloskey{sort} value they will be treated as the
same entry.

Take care if you use \opt1 and the \gloskey{name} contains fragile
commands. You will either need to explicitly set the \gloskey{sort}
key or use the \pkgopt[true]{sanitizesort} package option (unless
you use the \texttt{def} or \texttt{use} sort methods).
\end{important}

\item[{\gloskey{type}}] This specifies the label of the glossary in 
which this entry belongs. If omitted, the default glossary is 
assumed unless \ics{newacronym} is used (see
\sectionref{sec:acronyms}).

\item[{\gloskey{user1}, \ldots, \gloskey{user6}%
\igloskey{user2}\igloskey{user3}\igloskey{user4}\igloskey{user5}}]
Six keys provided for any additional information the user may want
to specify.  (For example, an associated dimension or an alternative
plural or some other grammatical construct.) Alternatively, you can
add new keys using \ics{glsaddkey} (see \sectionref{sec:addkey}).
Other keys are also provided by the \sty{glossaries-prefix}
(\sectionref{sec:prefix}) and \sty{glossaries-accsupp}
(\sectionref{sec:accsupp}) packages.

\item[{\gloskey{nonumberlist}}] A boolean key. If the value is
missing or is \texttt{true}, this will suppress the \gls{numberlist} just for
this entry. Conversely, if you have used the package option
\pkgopt{nonumberlist}, you can activate the \gls*{numberlist} just
for this entry with \gloskey{nonumberlist=false}.
(See \sectionref{sec:numberlists}.)

\item[{\gloskey{see}}] Cross-reference another entry. Using the
\gloskey{see} key will automatically add this entry to the
glossary, but will not automatically add the cross-referenced entry.
The referenced entry should be supplied as the value to this key.
If you want to override the \qt{see} tag, you can supply the new
tag in square brackets before the label. For example
\verb|see=[see also]{anotherlabel}|. \textbf{Note that if you have
suppressed the \gls{numberlist}, the cross-referencing information
won't appear in the glossary, as it forms part of the
\gls*{numberlist}.} You can override this for individual
glossary entries using \gloskey{nonumberlist=false} (see above).
Alternatively, you can use the \pkgopt{seeautonumberlist} package
option.  For further details, see \sectionref{sec:crossref}.

\begin{important}
For \optsand23, \ics{makeglossaries} must be used before any occurrence of
\ics{newglossaryentry} that contains the \gloskey{see} key.
\end{important}

\end{description}

The following keys are reserved for \ics{newacronym} (see
\sectionref{sec:acronyms}): \gloskey{long}, \gloskey{longplural},
\gloskey{short} and \gloskey{shortplural}. Additional keys are
provided by the \sty{glossaries-prefix} (\sectionref{sec:prefix}) 
and the \sty{glossaries-accsupp} (\sectionref{sec:accsupp}) packages.
You can also define your own custom keys (see
\sectionref{sec:addkey}).

Note that if the name starts with 
\gls{nonlatinchar}, you must group the character, otherwise it will 
cause a problem for commands like \ics{Gls} and \ics{Glspl}. 
For example:
\begin{verbatim}
\newglossaryentry{elite}{name={{\'e}lite},
description={select group or class}}
\end{verbatim}
Note that the same applies if you are using the \sty{inputenc}
package:
\begin{alltt}
\verb|\newglossaryentry{elite}{name={{|\'e\verb|}lite},|
description=\verb|{select group or class}}|
\end{alltt}
Note that in both of the above examples, you will also need to
supply the \gloskey{sort} key if you are using \opt2
whereas \gls{xindy} (\opt3) is usually able to sort
\glspl{nonlatinchar}
correctly. \opt1 discards accents from 
\glspl{latexexlatinchar} unless you use the \pkgopt[true]{sanitizesort}.

\section{Plurals}
\label{sec:plurals}

You may have noticed from above that you can specify the plural form
when you define a term. If you omit this, the plural will be 
obtained by appending 
\begin{definition}[\DescribeMacro{\glspluralsuffix}]
\cs{glspluralsuffix}
\end{definition}
to the singular form. This command defaults to the letter \qt{s}. 
For example:
\begin{verbatim}
\newglossaryentry{cow}{name=cow,description={a fully grown
female of any bovine animal}}
\end{verbatim}
defines a new entry whose singular form is \qt{cow} and plural form
is \qt{cows}. However, if you are writing in archaic English, you
may want to use \qt{kine} as the plural form, in which case you
would have to do:
\begin{verbatim}
\newglossaryentry{cow}{name=cow,plural=kine,
description={a fully grown female of any bovine animal}}
\end{verbatim}

If you are writing in a language that supports
multiple plurals (for a given term) then use the \gloskey{plural} key
for one of them and one of the user keys to specify the
other plural form. For example:
\begin{verbatim}
\newglossaryentry{cow}{%
  name=cow,%
  description={a fully grown female of any bovine animal 
               (plural cows, archaic plural kine)},%
  user1={kine}}
\end{verbatim}
You can then use \verb|\glspl{cow}| to produce \qt{cows} and 
\verb|\glsuseri{cow}| to produce \qt{kine}. You can, of course,
define an easy to remember synonym. For example:
\begin{verbatim}
\let\glsaltpl\glsuseri
\end{verbatim}
Then you don't have to remember which key you used to store the
second plural. Alternatively, you can define your own keys using
\ics{glsaddkey}, described in \sectionref{sec:addkey}.

If you are using a language that usually forms plurals
by appending a different letter, or sequence of letters, you can
redefine \cs{glspluralsuffix} as required. However, this must be
done \emph{before} the entries are defined. For languages that don't
form plurals by simply appending a suffix, all the plural forms must be 
specified using the \gloskey{plural} key (and the \gloskey{firstplural}
key where necessary). 

\section{Other Grammatical Constructs}
\label{sec:grammar}

You can use the six user keys to provide alternatives, such as
participles. For example:
\begin{verbatim}
\let\glsing\glsuseri
\let\glsd\glsuserii

\newcommand*{\ingkey}{user1}
\newcommand*{\edkey}{user2}

\newcommand*{\newword}[3][]{%
  \newglossaryentry{#2}{%
   name={#2},%
   description={#3},%
   \edkey={#2ed},%
   \ingkey={#2ing},#1%
  }%
}
\end{verbatim}
With the above definitions, I can now define terms like this:
\begin{verbatim}
\newword{play}{to take part in activities for enjoyment}
\newword[\edkey={ran},\ingkey={running}]{run}{to move fast using 
the legs}
\end{verbatim}
and use them in the text:
\begin{verbatim}
Peter is \glsing{play} in the park today.

Jane \glsd{play} in the park yesterday.

Peter and Jane \glsd{run} in the park last week.
\end{verbatim}

Alternatively, you can define your own keys using
\ics{glsaddkey}, described below in \sectionref{sec:addkey}.

\section{Additional Keys}
\label{sec:addkey}

You can now also define your own custom keys using:
\begin{definition}[\DescribeMacro\glsaddkey]
\cs{glsaddkey}\marg{key}%
\marg{default value}%
\marg{no link cs}%
\marg{no link ucfirst cs}%
\marg{link cs}%
\marg{link ucfirst cs}%
\marg{link allcaps cs}
\end{definition}
where:
\begin{description}
\item[\meta{key}] is the new key to use in \ics{newglossaryentry}
(or similar commands such as \ics{longnewglossaryentry});
\item[\meta{default value}] is the default value to use if this key
isn't used in an entry definition (this may reference the current
entry label via \cs{glslabel}, but you will have to switch on
expansion via the starred version of \cs{glsaddkey} and protect
fragile commands);
\item[\meta{no link cs}] is the control sequence to use analogous
to commands like \ics{glsentrytext};
\item[\meta{no link ucfirst cs}] is the control sequence to use analogous
to commands like \ics{Glsentrytext};
\item[\meta{link cs}] is the control sequence to use analogous
to commands like \ics{glstext};
\item[\meta{link ucfirst cs}] is the control sequence to use analogous
to commands like \ics{Glstext};
\item[\meta{link allcaps cs}] is the control sequence to use analogous
to commands like \ics{GLStext}.
\end{description}
The starred version of \cs{glsaddkey} switches on expansion for this
key. The unstarred version doesn't override the current expansion
setting.

\begin{example}{Defining Custom Keys}{ex:addkey}
Suppose I want to define two new keys, \texttt{ed} and \texttt{ing},
that default to the entry text followed by \qt{ed} and \qt{ing},
respectively. The default value will need expanding in both cases, so 
I need to use the starred form:
\begin{verbatim}
 % Define "ed" key:
 \glsaddkey*
  {ed}% key
  {\glsentrytext{\glslabel}ed}% default value
  {\glsentryed}% command analogous to \glsentrytext
  {\Glsentryed}% command analogous to \Glsentrytext
  {\glsed}% command analogous to \glstext
  {\Glsed}% command analogous to \Glstext
  {\GLSed}% command analogous to \GLStext

 % Define "ing" key:
 \glsaddkey*
  {ing}% key
  {\glsentrytext{\glslabel}ing}% default value
  {\glsentrying}% command analogous to \glsentrytext
  {\Glsentrying}% command analogous to \Glsentrytext
  {\glsing}% command analogous to \glstext
  {\Glsing}% command analogous to \Glstext
  {\GLSing}% command analogous to \GLStext
\end{verbatim}
Now I can define some entries:
\begin{verbatim}
 % No need to override defaults for this entry:

 \newglossaryentry{jump}{name={jump},description={}}

 % Need to override defaults on these entries:

 \newglossaryentry{run}{name={run},%
   ed={ran},%
   ing={running},%
   description={}}

 \newglossaryentry{waddle}{name={waddle},%
   ed={waddled},%
   ing={waddling},%
   description={}}
\end{verbatim}

These entries can later be used in the document:
\begin{verbatim}
The dog \glsed{jump} over the duck.

The duck was \glsing{waddle} round the dog.

The dog \glsed{run} away from the duck.
\end{verbatim}
For a complete document, see the sample file \samplefile{-newkeys}.
\end{example}

\section{Expansion}
\label{sec:expansion}

When you define new glossary entries expansion is performed by
default, except for the \gloskey{name}, \gloskey{description},
\gloskey{descriptionplural}, \gloskey{symbol}, \gloskey{symbolplural}
and \gloskey{sort} keys (these keys all have expansion suppressed via
\cs{glssetnoexpandfield}).

You can switch expansion on or off for individual keys using
\begin{definition}[\DescribeMacro\glssetexpandfield]
\cs{glssetexpandfield}\marg{field}
\end{definition}
or
\begin{definition}[\DescribeMacro\glssetnoexpandfield]
\cs{glssetnoexpandfield}\marg{field}
\end{definition}
respectively, where \meta{field} is the field tag corresponding to
the key. In most cases, this is the same as the name of the key
except for those listed in \tableref{tab:fieldmap}.

\begin{table}[htbp]
\caption{Key to Field Mappings}
\label{tab:fieldmap}
\centering
\begin{tabular}{ll}
\bfseries Key & \bfseries Field\\
\gloskey{sort} & \ttfamily sortvalue\\
\gloskey{firstplural} & \ttfamily firstpl\\
\gloskey{description} & \ttfamily desc\\
\gloskey{descriptionplural} & \ttfamily descplural\\
\gloskey{user1} & \ttfamily useri\\
\gloskey{user2} & \ttfamily userii\\
\gloskey{user3} & \ttfamily useriii\\
\gloskey{user4} & \ttfamily useriv\\
\gloskey{user5} & \ttfamily userv\\
\gloskey{user6} & \ttfamily uservi\\
\gloskey{longplural} & \ttfamily longpl\\
\gloskey{shortplural} & \ttfamily shortpl
\end{tabular}
\end{table}

Any keys that haven't had the expansion explicitly set using
\cs{glssetexpandfield} or \cs{glssetnoexpandfield} are governed by
\begin{definition}[\DescribeMacro\glsexpandfields]
\cs{glsexpandfields}
\end{definition}
and
\begin{definition}[\DescribeMacro\glsnoexpandfields]
\cs{glsnoexpandfields}
\end{definition}

If your entries contain any fragile commands, I recommend you switch
off expansion via \cs{glsnoexpandfields}. (This should be used
before you define the entries.)

\section{Sub-Entries}
\label{sec:subentries}

As from version 1.17, it is possible to specify sub-entries. These
may be used to order the glossary into categories, in which case the
sub-entry will have a different name to its parent entry, or it may
be used to distinguish different definitions for the same word, in
which case the sub-entries will have the same name as the parent
entry. Note that not all glossary styles support hierarchical
entries and may display all the entries in a flat format. Of the
styles that support sub-entries, some display the sub-entry's name
whilst others don't. Therefore you need to ensure that you use a
suitable style. (See \sectionref{sec:styles} for a list of predefined
styles.) As from version 3.0, level~1 sub-entries are automatically
numbered in the predefined styles if you use the
\pkgopt{subentrycounter} package option (see
\sectionref{sec:pkgopts-printglos} for further details).

Note that the parent entry will automatically be added to the
glossary if any of its child entries are used in the document.
If the parent entry is not referenced in the document, it will not
have a \gls{numberlist}. Note also that \gls{makeindex} has a
restriction on the maximum sub-entry depth.

\subsection{Hierarchical Categories}
\label{sec:hierarchical}

To arrange a glossary with hierarchical categories, you need to
first define the category and then define the sub-entries using the
relevant category entry as the value of the \gloskey{parent} key.

\begin{example}{Hierarchical Categories---Greek and Roman 
Mathematical Symbols}{ex:hierarchical}

Suppose I want a glossary of mathematical symbols that
are divided into Greek letters and Roman letters. Then I can define
the categories as follows:
\begin{verbatim}
\newglossaryentry{greekletter}{name={Greek letters},
description={\nopostdesc}}

\newglossaryentry{romanletter}{name={Roman letters},
description={\nopostdesc}}
\end{verbatim}

Note that in this example, the category entries don't need a
description so I have set the descriptions to \ics{nopostdesc}.
This gives a blank description and suppresses the
description terminator.

I can now define my sub-entries as follows:
\begin{verbatim}
\newglossaryentry{pi}{name={\ensuremath{\pi}},sort={pi},
description={ratio of the circumference of a circle to 
the diameter},
parent=greekletter}

\newglossaryentry{C}{name={\ensuremath{C}},sort={C},
description={Euler's constant},
parent=romanletter}
\end{verbatim}
For a complete document, see the sample file \samplefile{tree}.
\end{example}

\subsection{Homographs}
\label{sec:homographs}

Sub-entries that have the same name as the parent entry, don't need
to have the \gloskey{name} key. For example, the word \qt{glossary}
can mean a list of technical words or a collection of glosses. In
both cases the plural is \qt{glossaries}. So first define the parent
entry:
\begin{verbatim}
\newglossaryentry{glossary}{name=glossary,
description={\nopostdesc},
plural={glossaries}}
\end{verbatim}
Again, the parent entry has no description, so the description
terminator needs to be suppressed using \ics{nopostdesc}.

Now define the two different meanings of the word:
\begin{verbatim}
\newglossaryentry{glossarylist}{
description={list of technical words},
sort={1},
parent={glossary}}

\newglossaryentry{glossarycol}{
description={collection of glosses},
sort={2},
parent={glossary}}
\end{verbatim}
Note that if I reference the parent entry, the location will be
added to the parent's \gls{numberlist}, whereas if I reference any
of the child entries, the location will be added to the child
entry's number list. Note also that since the sub-entries have the
same name, the \gloskey{sort} key is required unless you are using
the \pkgopt[use]{sort} or \pkgopt[def]{sort} package options (see
\sectionref{sec:pkgopts-sort}). You
can use the \pkgopt{subentrycounter} package option to automatically
number the first-level child entries. See
\sectionref{sec:pkgopts-printglos} for further details.

In the above example, the plural form for both of the child entries
is the same as the parent entry, so the \gloskey{plural} key was
not required for the child entries. However, if the sub-entries
have different plurals, they will need to be specified. For example:
\begin{verbatim}
\newglossaryentry{bravo}{name={bravo},
description={\nopostdesc}}

\newglossaryentry{bravocry}{description={cry of approval 
(pl.\ bravos)},
sort={1},
plural={bravos},
parent=bravo}

\newglossaryentry{bravoruffian}{description={hired 
ruffian or killer (pl.\ bravoes)},
sort={2},
plural={bravoes},
parent=bravo}
\end{verbatim}


\section{Loading Entries From a File}
\label{sec:loadglsentries}

You can store all your glossary entry definitions in another
file and use:
\begin{definition}[\DescribeMacro{\loadglsentries}]
\cs{loadglsentries}\oarg{type}\marg{filename}
\end{definition}
where \meta{filename} is the name of the file containing all the
\ics{newglossaryentry} or \ics{longnewglossaryentry} commands. 
The optional argument
\meta{type} is the name of the glossary to which those entries
should belong, for those entries where the \gloskey{type} key has 
been omitted (or, more specifically, for those entries whose
type has been specified by \ics{glsdefaulttype}, which is what
\ics{newglossaryentry} uses by default).

This is a~preamble-only command. You may also use \ics{input} to load
the file but don't use \ics{include}.

\begin{important}
If you want to use \ics{AtBeginDocument} to \cs{input} all your entries
automatically at the start of the document, add the
\cs{AtBeginDocument} command \emph{before} you load the
\sty{glossaries} package (and \sty{babel}, if you are also loading
that) to avoid the creation of the
\texttt{.glsdefs} file and any associated problems that are caused
by defining commands in the \env{document} environment.
(See \sectionref{sec:docdefs}.)
\end{important}

\begin{example}{Loading Entries from Another File}{ex:loadgls}
Suppose I have a file called \texttt{myentries.tex} which contains:
\begin{verbatim}
\newglossaryentry{perl}{type=main,
name={Perl},
description={A scripting language}}

\newglossaryentry{tex}{name={\TeX},
description={A typesetting language},sort={TeX}}

\newglossaryentry{html}{type=\glsdefaulttype,
name={html},
description={A mark up language}}
\end{verbatim}
and suppose in my document preamble I use the command:
\begin{verbatim}
\loadglsentries[languages]{myentries}
\end{verbatim}
then this will add the entries \texttt{tex} and \texttt{html}
to the glossary whose type is given by \texttt{languages}, but
the entry \texttt{perl} will be added to the main glossary, since
it explicitly sets the type to \texttt{main}.
\end{example}

\textbf{Note:} if you use \ics{newacronym} (see
\sectionref{sec:acronyms}) the type is set as
\verb|type=\acronymtype| unless you explicitly override it. For
example, if my file \texttt{myacronyms.tex} contains:
\begin{verbatim}
\newacronym{aca}{aca}{a contrived acronym}
\end{verbatim}
then (supposing I have defined a new glossary type called
\texttt{altacronym})
\begin{verbatim}
\loadglsentries[altacronym]{myacronyms}
\end{verbatim}
will add \texttt{aca} to the glossary type \texttt{acronym}, if the
package option \pkgopt{acronym} has been specified, or will add
\texttt{aca} to the glossary type \texttt{altacronym}, if the
package option \pkgopt{acronym} is not specified.\footnote{This is
because \ics{acronymtype} is set to \ics{glsdefaulttype} if the
\pkgopt{acronym} package option is not used.}

If you have used the \pkgopt{acronym} package option,
there are two possible solutions to this problem:
\begin{enumerate}
\item Change \texttt{myacronyms.tex} so that entries are defined in
the form:
\begin{verbatim}
\newacronym[type=\glsdefaulttype]{aca}{aca}{a 
contrived acronym}
\end{verbatim}
and do:
\begin{verbatim}
\loadglsentries[altacronym]{myacronyms}
\end{verbatim}

\item Temporarily change \cs{acronymtype} to the target glossary:
\begin{verbatim}
\let\orgacronymtype\acronymtype
\renewcommand{\acronymtype}{altacronym}
\loadglsentries{myacronyms}
\let\acronymtype\orgacronymtype
\end{verbatim}
\end{enumerate}

Note that only those entries that have been used
in the text will appear in the relevant glossaries.
Note also that \cs{loadglsentries} may only be used in the 
preamble.

Remember that you can use \ics{provideglossaryentry} rather than
\ics{newglossaryentry}. Suppose you want to maintain a large database
of acronyms or terms that you're likely to use in your documents,
but you may want to use a modified version of some of those entries.
(Suppose, for example, one document may require a more detailed
description.) Then if you define the entries using
\cs{provideglossaryentry} in your database file, you can override
the definition by simply using \cs{newglossaryentry} before loading
the file. For example, suppose your file (called, say,
\texttt{terms.tex}) contains:
\begin{verbatim}
\provideglossaryentry{mallard}{name=mallard,
 description={a type of duck}}
\end{verbatim}
but suppose your document requires a more detailed description, you
can do:
\begin{verbatim}
\usepackage{glossaries}
\makeglossaries

\newglossaryentry{mallard}{name=mallard,
 description={a dabbling duck where the male has a green head}}

\loadglsentries{terms}
\end{verbatim}
Now the \texttt{mallard} definition in the \texttt{terms.tex} file
will be ignored.

\section{Moving Entries to Another Glossary}
\label{sec:moveentry}

As from version~3.02, you can move an entry from one glossary to
another using:
\begin{definition}[\DescribeMacro{\glsmoveentry}]
\cs{glsmoveentry}\marg{label}\marg{target glossary label}
\end{definition}
where \meta{label} is the unique label identifying the required
entry and \meta{target glossary label} is the unique label
identifying the glossary in which to put the entry.

Note that no check is performed to determine the existence of
the target glossary. If you want to move an entry to a glossary
that's skipped by \ics{printglossaries}, then define an ignored
glossary with \ics{newignoredglossary}. (See
\sectionref{sec:newglossary}.)

\begin{important}
Unpredictable results may occur if you move an entry to a different
glossary from its parent or children.
\end{important}

\section{Drawbacks With Defining Entries in the Document Environment}
\label{sec:docdefs}

Originally, \ics{newglossaryentry} (and \ics{newacronym}) could only be 
used in the preamble. I reluctantly removed this restriction in version 1.13,
but there are issues with defining commands in the \env{document}
environment instead of the preamble, which is why the restriction is
maintained for newer commands. This restriction is also reimposed
for \cs{newglossaryentry} by the new \opt1.

\subsection{Technical Issues}
\label{sec:techissues}

\begin{enumerate}
 \item If you define an entry mid-way through your document, but
subsequently shuffle sections around, you could end up using an
entry before it has been defined.

 \item Entry information is required when displaying the glossary.
If this occurs at the start of the document, but the entries aren't
defined until later, then the entry details are
being looked up before the entry has been defined.

 \item If you use a package, such as \sty{babel}, that makes
certain characters active at the start of the \env{document}
environment, there will be a~problem if those characters have
a~special significance when defining glossary entries.
These characters include the double-quote \verb|"| character, the
exclamation mark \texttt{!} character, the question mark \verb|?|
character, and the pipe \verb"|" character. They
must not be active when defining a~glossary entry where they occur
in the \gloskey{sort} key (and they should be avoided in the label
if they may be active at any point in the document). Additionally, 
the comma \texttt{,} character and the equals \texttt{=} character
should not be active when using commands that have
\meta{key}=\meta{value} arguments.

\end{enumerate}

To overcome the first two problems, as from version 4.0 the
\styfmt{glossaries} package modifies the definition of
\cs{newglossaryentry} at the beginning of the \env{document}
environment so that the definitions are written to an external file
(\cs{jobname}\filetype{.glsdefs}) which is then read in at the start
of the document on the next run. The entry will then only be defined
in the \env{document} environment if it doesn't already exist. This
means that the entry can now be looked up in the glossary, even if
the glossary occurs at the beginning of the document.

There are drawbacks to this mechanism: if you modify an entry
definition, you need a second run to see the effect of your
modification; this method requires an extra \cs{newwrite}, which may
exceed \TeX's maximum allocation; unexpected expansion issues could
occur; if you have very long entries, you could find unexpected line
breaks have been written to the temporary file causing spurious
spaces (or, even worse, a~command name could get split across a line
causing an error message).

The last reason is why \ics{longnewglossaryentry} has the
preamble-only restriction, which I don't intend to lift.

\subsection{Good Practice Issues}
\label{sec:goodpractice}

The above section covers technical issues that can cause your document to have
compilation errors or produce incorrect output. This section
focuses on good writing practice. The main reason cited by users
wanting to define entries within the \env{document} environment rather
than in the preamble is that they want to write the definition as
they type in their document text. This suggests a \qt{stream of
consciousness} style of writing that may be acceptable in certain
literary genres but is inappropriate for factual documents.

When you write technical documents, regardless of whether it's a PhD
thesis or an article for a~journal or proceedings, you must plan what you write
in advance. If you plan in advance, you should have a fairly good
idea of the type of terminology that your document will contain,
so while you are planning, create a new file with all your entry
definitions. If, while you're writing your document, you remember
another term you need, then you can switch over to your definition file and
add it. Most text editors have the ability to have more than one
file open at a time. The other advantage to this approach is that if
you forget the label, you can look it up in the definition file
rather than searching through your document text to find the
definition.

\chapter{Number lists}
\label{sec:numberlists}

Each entry in the glossary has an associated \gls{numberlist}.
By default, these numbers refer to the pages on which that entry has
been used (using any of the commands described in
\sectionref{sec:glslink} and \sectionref{sec:glsadd}). The number
list can be suppressed using the \pkgopt{nonumberlist} package
option, or an alternative counter can be set as the default using
the \pkgopt{counter} package option. The number list is also
referred to as the location list\index{location list|see{number list}}.

Both \gls{makeindex} and \gls{xindy} (\optsand23) concatenate a
sequence of~3 or more consecutive pages into~a range. With 
\gls*{xindy} (\opt3) you can vary the minimum sequence length using
\linebreak
\ics{GlsSetXdyMinRangeLength}\marg{n} where \meta{n} is either
an integer or the keyword \texttt{none} which indicates that there
should be no range formation.

\begin{important}
Note that \cs{GlsSetXdyMinRangeLength} must be used before
\ics{makeglossaries} and has no effect if \ics{noist} is used.
\end{important}

With both \gls{makeindex} and \gls{xindy} (\optsand23), you can replace
the separator and the closing number in the range using:
\begin{definition}[\DescribeMacro{\glsSetSuffixF}]
\cs{glsSetSuffixF}\marg{suffix}
\end{definition}
\begin{definition}[\DescribeMacro{\glsSetSuffixFF}]
\cs{glsSetSuffixFF}\marg{suffix}
\end{definition}
where the former command specifies the suffix to use for a 2 page
list and the latter specifies the suffix to use for longer lists.
For example:
\begin{verbatim}
\glsSetSuffixF{f.}
\glsSetSuffixFF{ff.}
\end{verbatim}
Note that if you use \gls{xindy} (\opt3), you will also need to
set the minimum range length to 1 if you want to change these
suffixes:
\begin{verbatim}
\GlsSetXdyMinRangeLength{1}
\end{verbatim}
Note that if you use the \sty{hyperref} package, you will need
to use \ics{nohyperpage} in the suffix to ensure that the hyperlinks
work correctly. For example:
\begin{verbatim}
\glsSetSuffixF{\nohyperpage{f.}}
\glsSetSuffixFF{\nohyperpage{ff.}}
\end{verbatim}

\begin{important}
Note that \cs{glsSetSuffixF} and \cs{glsSetSuffixFF} must be used 
before \ics{makeglossaries} and have no effect if \ics{noist} is 
used.
\end{important}

\opt1 doesn't form ranges. However, with this option you
can iterate over an entry's \gls{numberlist} using:
\begin{definition}[\DescribeMacro\glsnumberlistloop]
\cs{glsnumberlistloop}\marg{label}\marg{handler cs}\marg{xr handler
cs}
\end{definition}
where \meta{label} is the entry's label and \meta{handler cs} is a
handler control sequence of the form:
\begin{definition}
\meta{handler cs}\marg{prefix}\marg{counter}\marg{format}\marg{location}
\end{definition}
where \meta{prefix} is the \sty{hyperref} prefix, \meta{counter} is
the name of the counter used for the location, \meta{format} is the
format used to display the location (e.g.\ \texttt{textbf})
and \meta{location} is the location. The third argument is the
control sequence to use for any cross-references in the list. This
handler should have the syntax:
\begin{definition}
\meta{xr handler cs}\oarg{tag}\marg{xr list}
\end{definition}
where \meta{tag} is the cross-referenced text (e.g.\ \qt{see}) and
\meta{xr list} is a~comma-separated list of labels. (This actually
has a third argument but it's always empty when used with \opt1.)

For example, if on page~12
I~have used
\begin{verbatim}
\gls[format=textbf]{apple}
\end{verbatim}
and on page~18 I~have used
\begin{verbatim}
\gls[format=emph]{apple}
\end{verbatim}
then
\begin{verbatim}
\glsnumberlistloop{apple}{\myhandler}
\end{verbatim}
will be equivalent to:
\begin{verbatim}
\myhandler{}{page}{textbf}{12}%
\myhandler{}{page}{emph}{18}%
\end{verbatim}
There is a predefined handler that's used to display the
\gls{numberlist} in the glossary:
\begin{definition}[\DescribeMacro\glsnoidxdisplayloc]
\cs{glsnoidxdisplayloc}\marg{prefix}\marg{counter}\marg{format}\marg{location}
\end{definition}
The predefined handler used for the cross-references in the glossary is:
\begin{definition}
\cs{glsseeformat}\oarg{tag}\marg{xr list}\marg{location}
\end{definition}
which is described in \sectionref{sec:customxr}.

\begin{important}
\cs{glsnumberlistloop} is not available for \optsand23.
\end{important}

\chapter{Links to Glossary Entries}
\label{sec:glslink}

Once you have defined a glossary entry using \ics{newglossaryentry}
or \ics{newacronym} (see \sectionref{sec:acronyms}),
you can refer to that entry in the document using one of the
commands listed in \sectionref{sec:gls-like} or
\sectionref{sec:glstext-like}. The text which appears at that
point in the document when using one of these commands is referred
to as the \gls{linktext} (even if there are no hyperlinks). These
commands also add a line to an external file that is
used to generate the relevant entry in the glossary. This
information includes an associated location that is added to the
\gls{numberlist} for that entry. By default, the location refers to
the page number. For further information on number lists, see
\sectionref{sec:numberlists}.  These external files need to be
post-processed by \gls{makeindex} or \gls{xindy} unless you have
chosen \opt1. If you don't use \ics{makeglossaries} these external
files won't be created.

\begin{important}
I strongly recommend that you don't use the commands
defined in this chapter in the arguments of sectioning or caption
commands or any other command that has a moving argument.
\end{important}

The above warning is particularly important if you are using the
\styfmt{glossaries} package in conjunction with the \sty{hyperref}
package. Instead, use one of the \emph{expandable} commands listed in
\sectionref{sec:glsnolink} (such as \ics{glsentrytext} \emph{but
not} the non-expandable
case changing versions like \ics{Glsentrytext}). Alternatively, provide an
alternative via the optional argument to the sectioning\slash caption
command or use \sty{hyperref}'s \ics{texorpdfstring}. Examples:
\begin{verbatim}
\chapter{An overview of \glsentrytext{perl}}
\chapter[An overview of Perl]{An overview of \gls{perl}}
\chapter{An overview of \texorpdfstring{\gls{perl}}{Perl}}
\end{verbatim}

If you want the \gls{linktext} to produce a hyperlink to the
corresponding entry details in the glossary, you should load the
\sty{hyperref} package \emph{before} the \styfmt{glossaries}
package. That's what I've done in this document, so if you see a
hyperlinked term, such as \gls{linktext}, you can click on the word
or phrase and it will take you to a brief description in this
document's glossary.

\begin{important}
If you use the \sty{hyperref} package, I strongly recommend you use
\app{pdflatex} rather than \app{latex} to compile your document, if
possible. The DVI format of \LaTeX\ has limitations with the
hyperlinks that can cause a problem when used with the
\styfmt{glossaries} package. Firstly, the DVI format can't break a
hyperlink across a line whereas PDF\LaTeX\ can. This means that long
glossary entries (for example, the full form of an acronym) won't be
able to break across a line with the DVI format. Secondly, the DVI
format doesn't correctly size hyperlinks in subscripts or
superscripts. This means that if you define a term that may be used
as a subscript or superscript, if you use the DVI format, it won't
come out the correct size.

These are limitations of the DVI format not of the \sty{glossaries}
package.
\end{important}

It may be that you only want terms in certain glossaries to have
hyperlinks, but not for other glossaries. In this case, you can use the
package option \pkgopt{nohypertypes} to identify the glossary lists
that shouldn't have hyperlinked \gls{linktext}. See 
\sectionref{sec:pkgopts-general} for further details.

The way the \gls{linktext} is displayed depends on 
\begin{definition}[\DescribeMacro{\glstextformat}]
\cs{glstextformat}\marg{text}
\end{definition}
For example, to make all \gls{linktext} appear in a sans-serif
font, do:
\begin{verbatim}
\renewcommand*{\glstextformat}[1]{\textsf{#1}}
\end{verbatim}
Further customisation can be done via \ics{defglsentryfmt} or by
redefining \ics{glsentryfmt}. See \sectionref{sec:glsdisplay} for
further details.

Each entry has an associated conditional referred to as the
\firstuseflag. Some of the commands described in this chapter
automatically unset this flag and can also use it
to determine what text should be displayed. These types of commands
are the \glslike\ commands and are described in
\sectionref{sec:gls-like}. The commands that don't reference or change
the \firstuseflag\ are \glstextlike\ commands and are described
in \sectionref{sec:glstext-like}.  See \sectionref{sec:glsunset} for 
commands that unset or reset the \firstuseflag\ without referencing
the entries.

The \glslike\ and \glstextlike\ commands all take a first
optional argument that is a comma-separated list of
\meta{key}=\meta{value}
options, described below. They also have a star-variant, which
inserts \texttt{hyper=false} at the start of the list of options
and a plus-variant, which inserts \texttt{hyper=true} at the start
of the list of options. For example \verb|\gls*{sample}| is the same
as \verb|\gls[hyper=false]{sample}| and \verb|\gls+{sample}| is the
same as \verb|\gls[hyper=true]{sample}|, whereas just
\verb|\gls{sample}| will use the default hyperlink setting which
depends on a number of factors (such as whether the entry is in a
glossary that has been identified in the \pkgopt{nohypertypes} list).
You can override the \gloskey[glslink]{hyper} key in the variant's
optional argument, for example, \verb|\gls*[hyper=true]{sample}| but
this creates redundancy and is best avoided.

The following keys are available for the optional argument:
\begin{description}
\item[{\gloskey[glslink]{hyper}}] This is a boolean key which can
be used to enable/disable the hyperlink to the relevant entry
in the glossary. If this key is omitted, the value is determined by current
settings, as indicated above. For example, when used with a \glslike\ command, if this is the first
use and the \pkgopt[false]{hyperfirst} package option has been used,
then the default value is \texttt{hyper=false}. The hyperlink can be
forced on using \texttt{hyper=true} unless the hyperlinks have been
suppressed using \ics{glsdisablehyper}. You must load the
\sty{hyperref} package before the \sty{glossaries} package to
ensure the hyperlinks work.

\item[{\gloskey[glslink]{format}}] This specifies how to format the
associated location number for this entry in the glossary. This 
value is equivalent to the \gls{makeindex} encap value, and (as
with \ics{index}) the value needs to be the name of a command
\emph{without} the initial backslash. As with \ics{index}, the
characters \verb"(" and \verb")" can also be used to specify the
beginning and ending of a number range.  Again as with \ics{index},
the command should be the name of a command which takes an argument
(which will be the associated location).  Be careful not to use a
declaration (such as \texttt{bfseries}) instead of a text block command
(such as \texttt{textbf}) as the effect is not guaranteed to be
localised. If you want to apply more than one style to a given entry
(e.g.\ \textbf{bold} and \emph{italic}) you will need to create a
command that applies both formats, e.g.\ 
\begin{verbatim}
\newcommand*{\textbfem}[1]{\textbf{\emph{#1}}}
\end{verbatim}
and use that command.

In this document, the standard formats refer to the standard text 
block commands such as \ics{textbf} or \ics{emph} or any of the 
commands listed in \tableref{tab:hyperxx}.

\begin{important}
If you use \gls{xindy} instead of \gls{makeindex}, you
must specify any non-standard formats that you want to use
with the \gloskey[glslink]{format} key using
\ics{GlsAddXdyAttribute}\marg{name}. So if you use
\gls*{xindy} with the above example, you would need to add:
\begin{verbatim}
\GlsAddXdyAttribute{textbfem}
\end{verbatim}
See \sectionref{sec:xindy} for further details.
\end{important}

Note that unlike \ics{index}, you can't have anything following the
command name, such as an asterisk or arguments. If you want to
cross-reference another entry, either use the \gloskey{see} key when
you define the entry or use \ics{glssee} (described in
\sectionref{sec:crossref}).


If you are using hyperlinks and you want to change the font of the
hyperlinked location, don't use \ics{hyperpage} (provided by the
\sty{hyperref} package) as the locations may not refer to a page
number.  Instead, the \styfmt{glossaries} package provides number
formats listed in \tableref{tab:hyperxx}.

\begin{table}[htbp]
\caption{Predefined Hyperlinked Location Formats}
\label{tab:hyperxx}
\centering
\vskip\baselineskip
\begin{tabular}{ll}
\locfmt{hyperrm} & serif hyperlink\\
\locfmt{hypersf} & sans-serif hyperlink\\
\locfmt{hypertt} & monospaced hyperlink\\
\locfmt{hyperbf} & bold hyperlink\\
\locfmt{hypermd} & medium weight hyperlink\\
\locfmt{hyperit} & italic hyperlink\\
\locfmt{hypersl} & slanted hyperlink\\
\locfmt{hyperup} & upright hyperlink\\
\locfmt{hypersc} & small caps hyperlink\\
\locfmt{hyperemph} & emphasized hyperlink
\end{tabular}
\par
\end{table}

Note that if the \ics{hyperlink} command hasn't been defined, the
\texttt{hyper}\meta{xx} formats are equivalent to the analogous
\texttt{text}\meta{xx} font commands (and \texttt{hyperemph} is
equivalent to \texttt{emph}). If you want to make a new format, you
will need to define a command which takes one argument and use that.
For example, if you want the location number to be in a bold
sans-serif font, you can define a command called, say,
\ics{hyperbsf}:
\begin{verbatim}
\newcommand{\hyperbsf}[1]{\textbf{\hypersf{#1}}}
\end{verbatim}
and then use \texttt{hyperbsf} as the value for the \gloskey{format}
key. (See also \ifpdf section~\ref*{sec:code:printglos} \fi
\qt{Displaying the glossary} in the documented code,
\texttt{glossaries-code.pdf}.) Remember that if you use \gls{xindy}, you
will need to add this to the list of location attributes:
\begin{verbatim}
\GlsAddXdyAttribute{hyperbsf}
\end{verbatim}

\item[{\gloskey[glslink]{counter}}] This specifies which counter
to use for this location. This overrides the default counter
used by this entry. (See also \sectionref{sec:numberlists}.)

\item[{\gloskey[glslink]{local}}] This is a boolean key that only
makes a difference when used with \glslike\ commands that change the
entry's \gls{firstuseflag}. If \texttt{local=true}, the change to
the \gls*{firstuseflag} will be localised to the current scope. The
default is \texttt{local=false}.

\end{description}

\section{The \cs{gls}-Like Commands (First Use Flag Queried)}
\label{sec:gls-like}

This section describes the commands that unset the \firstuseflag\ on
completion, and in most cases they use the current state of the flag
to determine the text to be displayed. As described above, these
commands all have a star-variant (\texttt{hyper=false}) and 
a plus-variant (\texttt{hyper=true}) and have an
optional first argument that is a \meta{key}=\meta{value} list.

These commands use \ics{glsentryfmt} or the equivalent definition provided by
\ics{defglsentryfmt} to determine the automatically generated text
and its format (see \sectionref{sec:glsdisplay}).

Apart from \ics{glsdisp}, the commands described in this section
also have a \emph{final} optional argument \meta{insert} which may
be used to insert material into the automatically generated text.

\begin{important}
Since the commands have a final optional argument, take care if
you actually want to display an open square bracket after the command
when the final optional argument is absent. Insert an empty set of
braces \verb|{}| immediately before the opening square bracket to
prevent it from being interpreted as the final argument. For
example:
\begin{verbatim}
\gls{sample} {}[Editor's comment]
\end{verbatim}

Don't use any of the \glslike\ or \glstextlike\ commands in the 
\meta{insert} argument.
\end{important}

\begin{definition}[\DescribeMacro{\gls}]
\cs{gls}\oarg{options}\marg{label}\oarg{insert}
\end{definition}
This command typically determines the \gls{linktext} from the values
of the \gloskey{text} or \gloskey{first} keys supplied when the
entry was defined using \ics{newglossaryentry}. However, if the
entry was defined using \ics{newacronym} and \ics{setacronymstyle} was 
used, then the \gls*{linktext} will usually be determined from the \gloskey{long} or
\gloskey{short} keys.

There are two upper case variants:
\begin{definition}[\DescribeMacro{\Gls}]
\cs{Gls}\oarg{options}\marg{label}\oarg{insert}
\end{definition}
and
\begin{definition}[\DescribeMacro{\GLS}]
\cs{GLS}\oarg{options}\marg{label}\oarg{insert}
\end{definition}
which make the first letter of the link text or all the link text
upper case, respectively.

There are also analogous plural forms:
\begin{definition}[\DescribeMacro{\glspl}]
\cs{glspl}\oarg{options}\marg{label}\oarg{insert}
\end{definition}
\begin{definition}[\DescribeMacro{\Glspl}]
\cs{Glspl}\oarg{options}\marg{label}\oarg{insert}
\end{definition}
\begin{definition}[\DescribeMacro{\GLSpl}]
\cs{GLSpl}\oarg{options}\marg{label}\oarg{insert}
\end{definition}
These typically determine the link text from the \gloskey{plural} or
\gloskey{firstplural} keys supplied when the entry was 
defined using \ics{newglossaryentry} or, if the entry is an acronym
and \cs{setacronymstyle} was used, from the \gloskey{longplural} or
\gloskey{shortplural} keys.

\begin{important}
Be careful when you use glossary entries in math mode especially if you 
are using \sty{hyperref} as it can affect the spacing of subscripts and
superscripts. For example, suppose you have defined the following
entry:
\begin{verbatim}
\newglossaryentry{Falpha}{name={F_\alpha},
description=sample}
\end{verbatim}
and later you use it in math mode:
\begin{verbatim}
$\gls{Falpha}^2$
\end{verbatim}
This will result in $F_\alpha{}^2$ instead of $F_\alpha^2$. In this
situation it's best to bring the superscript into the hyperlink using
the final \meta{insert} optional argument:
\begin{verbatim}
$\gls{Falpha}[^2]$
\end{verbatim}
\end{important}

\begin{definition}[\DescribeMacro{\glsdisp}]
\cs{glsdisp}\oarg{options}\marg{label}\marg{link text}
\end{definition}
This behaves in the same way as the above commands, except
that the \meta{link text} is explicitly set. There's no final
optional argument as any inserted material can be added to the
\meta{link text} argument.

\begin{important}
Don't use any of the \glslike\ or \glstextlike\ commands in the 
\meta{link text} argument of \cs{glsdisp}.
\end{important}

\section{The \cs{glstext}-Like Commands (First Use Flag Not Queried)}
\label{sec:glstext-like}

This section describes the commands that don't change or reference
the \firstuseflag. As described above, these commands all have a
star-variant (\texttt{hyper=false}) and a plus-variant
(\texttt{hyper=true}) and have an optional first argument
that is a \meta{key}=\meta{value} list. These commands also don't
use \ics{glsentryfmt} or the equivalent definition provided by
\ics{defglsentryfmt} (see \sectionref{sec:glsdisplay}). 

Apart from \ics{glslink}, the commands described in this section
also have a \emph{final} optional argument \meta{insert} which may
be used to insert material into the automatically generated text.
See the caveat above in \sectionref{sec:gls-like}.

\begin{definition}[\DescribeMacro{\glslink}]
\cs{glslink}\oarg{options}\marg{label}\marg{link text}
\end{definition}
This command explicitly sets the \gls{linktext} as given in the
final argument.

\begin{important}
Don't use any of the \glslike\ or \glstextlike\ commands in the 
argument of \cs{glslink}.
\end{important}

\begin{definition}[\DescribeMacro{\glstext}]
\cs{glstext}\oarg{options}\marg{label}\oarg{insert}
\end{definition}
This command always uses the value of the \gloskey{text} key as the
\gls{linktext}.

There are also analogous commands:
\begin{definition}[\DescribeMacro{\Glstext}]
\cs{Glstext}\oarg{options}\marg{text}\oarg{insert}
\end{definition}
\begin{definition}[\DescribeMacro{\GLStext}]
\cs{GLStext}\oarg{options}\marg{text}\oarg{insert}
\end{definition}
These convert the first character or all the characters to
uppercase, respectively.

\begin{definition}[\DescribeMacro{\glsfirst}]
\cs{glsfirst}\oarg{options}\marg{label}\oarg{insert}
\end{definition}
This command always uses the value of the \gloskey{first} key as the
\gls{linktext}.

There are also analogous uppercasing commands:
\begin{definition}[\DescribeMacro{\Glsfirst}]
\cs{Glsfirst}\oarg{options}\marg{text}\oarg{insert}
\end{definition}
\begin{definition}[\DescribeMacro{\GLSfirst}]
\cs{GLSfirst}\oarg{options}\marg{text}\oarg{insert}
\end{definition}

\begin{definition}[\DescribeMacro{\glsplural}]
\cs{glsplural}\oarg{options}\marg{label}\oarg{insert}
\end{definition}
This command always uses the value of the \gloskey{plural} key as the
\gls{linktext}.

There are also analogous uppercasing commands:
\begin{definition}[\DescribeMacro{\Glsplural}]
\cs{Glsplural}\oarg{options}\marg{text}\oarg{insert}
\end{definition}
\begin{definition}[\DescribeMacro{\GLSplural}]
\cs{GLSplural}\oarg{options}\marg{text}\oarg{insert}
\end{definition}

\begin{definition}[\DescribeMacro{\glsfirstplural}]
\cs{glsfirstplural}\oarg{options}\marg{label}\oarg{insert}
\end{definition}
This command always uses the value of the \gloskey{firstplural} key as the
\gls{linktext}.

There are also analogous uppercasing commands:
\begin{definition}[\DescribeMacro{\Glsfirstplural}]
\cs{Glsfirstplural}\oarg{options}\marg{text}\oarg{insert}
\end{definition}
\begin{definition}[\DescribeMacro{\GLSfirstplural}]
\cs{GLSfirstplural}\oarg{options}\marg{text}\oarg{insert}
\end{definition}

\begin{definition}[\DescribeMacro{\glsname}]
\cs{glsname}\oarg{options}\marg{label}\oarg{insert}
\end{definition}
This command always uses the value of the \gloskey{name} key as the
\gls{linktext}.

There are also analogous uppercasing commands:
\begin{definition}[\DescribeMacro{\Glsname}]
\cs{Glsname}\oarg{options}\marg{text}\oarg{insert}
\end{definition}
\begin{definition}[\DescribeMacro{\GLSname}]
\cs{GLSname}\oarg{options}\marg{text}\oarg{insert}
\end{definition}

\begin{definition}[\DescribeMacro{\glssymbol}]
\cs{glssymbol}\oarg{options}\marg{label}\oarg{insert}
\end{definition}
This command always uses the value of the \gloskey{symbol} key as the
\gls{linktext}.

There are also analogous uppercasing commands:
\begin{definition}[\DescribeMacro{\Glssymbol}]
\cs{Glssymbol}\oarg{options}\marg{text}\oarg{insert}
\end{definition}
\begin{definition}[\DescribeMacro{\GLSsymbol}]
\cs{GLSsymbol}\oarg{options}\marg{text}\oarg{insert}
\end{definition}

\begin{definition}[\DescribeMacro{\glsdesc}]
\cs{glsdesc}\oarg{options}\marg{label}\oarg{insert}
\end{definition}
This command always uses the value of the \gloskey{description} key as the
\gls{linktext}.

There are also analogous uppercasing commands:
\begin{definition}[\DescribeMacro{\Glsdesc}]
\cs{Glsdesc}\oarg{options}\marg{text}\oarg{insert}
\end{definition}
\begin{definition}[\DescribeMacro{\GLSdesc}]
\cs{GLSdesc}\oarg{options}\marg{text}\oarg{insert}
\end{definition}

\begin{definition}[\DescribeMacro{\glsuseri}]
\cs{glsuseri}\oarg{options}\marg{label}\oarg{insert}
\end{definition}
This command always uses the value of the 
\gloskey{user1} key as the \gls{linktext}.

There are also analogous uppercasing commands:
\begin{definition}[\DescribeMacro{\Glsuseri}]
\cs{Glsuseri}\oarg{options}\marg{text}\oarg{insert}
\end{definition}
\begin{definition}[\DescribeMacro{\GLSuseri}]
\cs{GLSuseri}\oarg{options}\marg{text}\oarg{insert}
\end{definition}

\begin{definition}[\DescribeMacro{\glsuserii}]
\cs{glsuserii}\oarg{options}\marg{text}\oarg{insert}
\end{definition}
This command always uses the value of the 
\gloskey{user2} key as the \gls{linktext}.

There are also analogous uppercasing commands:
\begin{definition}[\DescribeMacro{\Glsuserii}]
\cs{Glsuserii}\oarg{options}\marg{text}\oarg{insert}
\end{definition}
\begin{definition}[\DescribeMacro{\GLSuserii}]
\cs{GLSuserii}\oarg{options}\marg{text}\oarg{insert}
\end{definition}

\begin{definition}[\DescribeMacro{\glsuseriii}]
\cs{glsuseriii}\oarg{options}\marg{text}\oarg{insert}
\end{definition}
This command always uses the value of the 
\gloskey{user3} key as the \gls{linktext}.

There are also analogous uppercasing commands:
\begin{definition}[\DescribeMacro{\Glsuseriii}]
\cs{Glsuseriii}\oarg{options}\marg{text}\oarg{insert}
\end{definition}
\begin{definition}[\DescribeMacro{\GLSuseriii}]
\cs{GLSuseriii}\oarg{options}\marg{text}\oarg{insert}
\end{definition}

\begin{definition}[\DescribeMacro{\glsuseriv}]
\cs{glsuseriv}\oarg{options}\marg{text}\oarg{insert}
\end{definition}
This command always uses the value of the 
\gloskey{user4} key as the \gls{linktext}.

There are also analogous uppercasing commands:
\begin{definition}[\DescribeMacro{\Glsuseriv}]
\cs{Glsuseriv}\oarg{options}\marg{text}\oarg{insert}
\end{definition}
\begin{definition}[\DescribeMacro{\GLSuseriv}]
\cs{GLSuseriv}\oarg{options}\marg{text}\oarg{insert}
\end{definition}

\begin{definition}[\DescribeMacro{\glsuserv}]
\cs{glsuserv}\oarg{options}\marg{text}\oarg{insert}
\end{definition}
This command always uses the value of the 
\gloskey{user5} key as the \gls{linktext}.

There are also analogous uppercasing commands:
\begin{definition}[\DescribeMacro{\Glsuserv}]
\cs{Glsuserv}\oarg{options}\marg{text}\oarg{insert}
\end{definition}
\begin{definition}[\DescribeMacro{\GLSuserv}]
\cs{GLSuserv}\oarg{options}\marg{text}\oarg{insert}
\end{definition}

\begin{definition}[\DescribeMacro{\glsuservi}]
\cs{glsuservi}\oarg{options}\marg{text}\oarg{insert}
\end{definition}
This command always uses the value of the 
\gloskey{user6} key as the \gls{linktext}.

There are also analogous uppercasing commands:
\begin{definition}[\DescribeMacro{\Glsuservi}]
\cs{Glsuservi}\oarg{options}\marg{text}\oarg{insert}
\end{definition}
\begin{definition}[\DescribeMacro{\GLSuservi}]
\cs{GLSuservi}\oarg{options}\marg{text}\oarg{insert}
\end{definition}

\section{Changing the format of the link text}
\label{sec:glsdisplay}

The default format of the \gls{linktext} for the \glslike\ commands 
is governed by\footnote{\ics{glsdisplayfirst} and \ics{glsdisplay} are now
deprecated. Backwards compatibility should be preserved but
you may need to use the \pkgopt{compatible-3.07} option}:
\begin{definition}[\DescribeMacro{\glsentryfmt}]
\cs{glsentryfmt}
\end{definition}
This may be redefined but if you only want the change the display style 
for a given glossary, then you need to use
\begin{definition}[\DescribeMacro{\defglsentryfmt}]
\cs{defglsentryfmt}\oarg{type}\marg{definition}
\end{definition}
instead of redefining \cs{glsentryfmt}. The optional first argument 
\meta{type} is the glossary type. This defaults to
\ics{glsdefaulttype} if omitted. The second argument is the
entry format definition.

\begin{important}
Note that \cs{glsentryfmt} is the default display format for
entries. Once the display format has been changed for an individual
glossary using \ics{defglsentryfmt}, redefining \cs{glsentryfmt}
won't have an effect on that glossary, you must instead use
\cs{defglsentryfmt} again. Note that glossaries that have
been identified as lists of acronyms (via the package option
\pkgopt{acronymlists} or the command \ics{DeclareAcronymList}, 
see \sectionref{sec:pkgopts-acronym}) use 
\cs{defglsentryfmt} to set their display style.
\end{important}

Within the \meta{definition} argument of \cs{defglsentryfmt}, or if you
want to redefine \cs{glsentryfmt}, you may use the following
commands:

\begin{definition}[\DescribeMacro{\glslabel}]
\cs{glslabel}
\end{definition}
This is the label of the entry being referenced. As from version
4.08, you can also access the glossary entry type using:
\begin{definition}[\DescribeMacro{\glstype}]
\cs{glstype}
\end{definition}
This is defined using \ics{edef} so the replacement text is the
actual glossary type rather than \verb|\glsentrytype{\glslabel}|.

\begin{definition}[\DescribeMacro{\glscustomtext}]
\cs{glscustomtext}
\end{definition}
This is the custom text supplied in \cs{glsdisp}. It's always empty
for \ics{gls}, \ics{glspl} and their upper case variants. (You can 
use \sty{etoolbox}'s \cs{ifdefempty} to determine if
\cs{glscustomtext} is empty.) 

\begin{definition}[\DescribeMacro{\glsinsert}]
\cs{glsinsert}
\end{definition}
The custom text supplied in the final optional argument to \cs{gls},
\cs{glspl} and their upper case variants.

\begin{definition}[\DescribeMacro{\glsifplural}]
\cs{glsifplural}\marg{true text}\marg{false text}
\end{definition}
If \cs{glspl}, \cs{Glspl} or \cs{GLSpl} was used, this command does 
\meta{true text} otherwise it does \meta{false text}.

\begin{definition}[\DescribeMacro{\glscapscase}]
\cs{glscapscase}\marg{no case}\marg{first uc}\marg{all caps}
\end{definition}
If \cs{gls}, \cs{glspl} or \cs{glsdisp} were used, this does \meta{no
case}. If \cs{Gls} or \cs{Glspl} were used, this does \meta{first
uc}. If \cs{GLS} or \cs{GLSpl} were used, this does \meta{all caps}.

\begin{definition}[\DescribeMacro\glsifhyperon]
\cs{glsifhyperon}\marg{hyper true}\marg{hyper false}
\end{definition}
This will do \meta{hyper true} if the hyperlinks are on for the
current reference, otherwise it will do \meta{hyper false}. The
hyperlink may be off even if it wasn't explicitly switched off with
the \gloskey[glslink]{hyper} key or the use of a starred command.
It may be off because the \sty{hyperref} package hasn't been loaded
or because \ics{glsdisablehyper} has been used or because the entry
is in a glossary type that's had the hyperlinks switched off (using
\pkgopt{nohypertypes}) or because it's the \firstuse\ and the
hyperlinks have been suppressed on first use.

Note that \ics{glsifhyper} is now deprecated. 
If you want to know if the command used to reference
this entry was used with the star or plus variant, you can use:
\begin{definition}[\DescribeMacro\glslinkvar]
\cs{glslinkvar}\marg{unmodified}\marg{star}\marg{plus}
\end{definition} 
This will do \meta{unmodified} if the unmodified version was used,
or will do \meta{star} if the starred version was used, or
will do \meta{plus} if the plus version was used.
Note that this doesn't take into account if the 
\gloskey[glslink]{hyper} key was used to override the default
setting, so this command shouldn't be used to guess whether or not
the hyperlink is on for this reference.

Note that you can also use commands such as \ics{ifglsused} within
the definition of \cs{glsentryfmt} (see \sectionref{sec:glsunset}).

If you only want to make minor modifications to \cs{glsentryfmt},
you can use
\begin{definition}[\DescribeMacro{\glsgenentryfmt}]
\cs{glsgenentryfmt}
\end{definition}
This uses the above commands to display just the \gloskey{first}, 
\gloskey{text}, \gloskey{plural} or \gloskey{firstplural} keys 
(or the custom text) with the insert text appended.

Alternatively, if want to change the entry format for acronyms
(defined via \ics{newacronym}) you can use:
\begin{definition}[\DescribeMacro{\glsgenacfmt}]
\cs{glsgenacfmt}
\end{definition}
This uses the values from the \gloskey{long}, \gloskey{short},
\gloskey{longplural} and \gloskey{shortplural} keys, rather than
using the \gloskey{text}, \gloskey{plural}, \gloskey{first} 
and \gloskey{firstplural} keys. The first use singular text is obtained via:
\begin{definition}[\DescribeMacro{\genacrfullformat}]
\cs{genacrfullformat}\marg{label}\marg{insert}
\end{definition}
instead of from the \gloskey{first} key, and the first use plural
text is obtained via:
\begin{definition}[\DescribeMacro{\genplacrfullformat}]
\cs{genplacrfullformat}\marg{label}\marg{insert}
\end{definition}
instead of from the \gloskey{firstplural} key.
In both cases, \meta{label} is the entry's label and \meta{insert}
is the insert text provided in the final optional argument of
commands like \ics{gls}. The default behaviour is to do the long
form (or plural long form) followed by \meta{insert} and a~space and the short form (or plural
short form) in parentheses, where the short form is in the argument
of \ics{firstacronymfont}. There are also first letter upper case
versions:
\begin{definition}[\DescribeMacro{\Genacrfullformat}]
\cs{Genacrfullformat}\marg{label}\marg{insert}
\end{definition}
and
\begin{definition}[\DescribeMacro{\Genplacrfullformat}]
\cs{Genplacrfullformat}\marg{label}\marg{insert}
\end{definition}
By default these perform a protected expansion on their no-case-change
equivalents and then use \ics{makefirstuc} to convert the first
character to upper case. If there are issues caused by this
expansion, you will need to redefine those commands to explicitly
use commands like \ics{Glsentrylong} (which is what the predefined
acronym styles, such as \acrstyle{long-short}, do). Otherwise, you
only need to redefine \ics{genacrfullformat} and
\ics{genplacrfullformat} to change the behaviour of
\ics{glsgenacfmt}. See \sectionref{sec:acronyms} for further details
on changing the style of acronyms.

\begin{important}
Note that \cs{glsentryfmt} is not used by the \glstextlike\ commands.
\end{important}

\begin{example}{Custom Entry Display in Text}{ex:customfmt}
Suppose you want a glossary of measurements and
units, you can use the \gloskey{symbol} key to store the unit:
\begin{verbatim}
\newglossaryentry{distance}{name=distance,
description={The length between two points},
symbol={km}}
\end{verbatim}
and now suppose you want \verb|\gls{distance}| to produce
\qt{distance (km)} on \firstuse, then you can redefine
\ics{glsentryfmt} as follows:
\begin{verbatim}
\renewcommand*{\glsentryfmt}{%
  \glsgenentryfmt
  \ifglsused{\glslabel}{}{\space (\glsentrysymbol{\glslabel})}%
}
\end{verbatim}

(Note that I've used \ics{glsentrysymbol} rather than \ics{glssymbol}
to avoid nested hyperlinks.)

Note also that all of the \gls{linktext} will be formatted according
to \ics{glstextformat} (described earlier). So if you do, say:
\begin{verbatim}
\renewcommand{\glstextformat}[1]{\textbf{#1}}
\renewcommand*{\glsentryfmt}{%
  \glsgenentryfmt
  \ifglsused{\glslabel}{}{\space(\glsentrysymbol{\glslabel})}%
}
\end{verbatim}
then \verb|\gls{distance}| will produce \qt{\textbf{distance (km)}}.

For a complete document, see the sample file \samplefile{-entryfmt}.
\end{example}

\begin{example}{Custom Format for Particular Glossary}{ex:defglsentryfmt}
Suppose you have created a new glossary called
\texttt{notation} and you want to change the way the entry is
displayed on \firstuse\ so that it includes the symbol, you can do:
\begin{verbatim}
\defglsentryfmt[notation]{\glsgenentryfmt
 \ifglsused{\glslabel}{}{\space
   (denoted \glsentrysymbol{\glslabel})}}
\end{verbatim}
Now suppose you have defined an entry as follows:
\begin{verbatim}
\newglossaryentry{set}{type=notation,
  name=set,
  description={A collection of objects},
  symbol={$S$}
}
\end{verbatim}
The \glslink{firstuse}{first time} you reference this entry it will be displayed as:
\qt{set (denoted $S$)} (assuming \ics{gls} was used).

Alternatively, if you expect all the symbols to be set in math mode,
you can do:
\begin{verbatim}
\defglsentryfmt[notation]{\glsgenentryfmt
 \ifglsused{\glslabel}{}{\space
   (denoted $\glsentrysymbol{\glslabel}$)}}
\end{verbatim}
and define entries like this:
\begin{verbatim}
\newglossaryentry{set}{type=notation,
  name=set,
  description={A collection of objects},
  symbol={S}
}
\end{verbatim}
\end{example}

Remember that if you use the \gloskey{symbol} key, you need to use a
glossary style that displays the symbol, as many of the styles
ignore it.

\section{Enabling and disabling hyperlinks to glossary entries}
\label{sec:disablehyperlinks}

If you load the \sty{hyperref} or \sty{html} packages prior to
loading the \styfmt{glossaries} package, the \glslike\ and
\glstextlike\ commands will automatically have hyperlinks
to the relevant glossary entry, unless the \gloskey[glslink]{hyper}
option has been switched off (either explicitly or through 
implicit means, such as via the \pkgopt{nohypertypes} package option).

You can disable or enable links using:
\begin{definition}[\DescribeMacro{\glsdisablehyper}]
\cs{glsdisablehyper}
\end{definition}
and
\begin{definition}[\DescribeMacro{\glsenablehyper}]
\cs{glsenablehyper}
\end{definition}
respectively. The effect can be localised by placing the commands
within a group. Note that you should only use \cs{glsenablehyper}
if the commands \ics{hyperlink} and \ics{hypertarget} have been
defined (for example, by the \sty{hyperref} package).

You can disable just the \gls{firstuse} links using the package
option \pkgopt[false]{hyperfirst}. Note that this option only
affects the \glslike\ commands that recognise the \firstuseflag.

\begin{example}{First Use With Hyperlinked Footnote Description}{ex:hyperdesc}
Suppose I want the \gls*{firstuse} to have a hyperlink to the
description in a footnote instead of hyperlinking to the relevant
place in the glossary. First I need to disable the hyperlinks on
first use via the package option \pkgopt[false]{hyperfirst}:
\begin{verbatim}
\usepackage[hyperfirst=false]{glossaries}
\end{verbatim}
Now I need to redefine \ics{glsentryfmt} (see
\sectionref{sec:glsdisplay}):
\begin{verbatim}
\renewcommand*{\glsentryfmt}{%
  \glsgenentryfmt
  \ifglsused{\glslabel}{}{\footnote{\glsentrydesc{\glslabel}}}%
}
\end{verbatim}

Now the first use won't have hyperlinked text, but will be followed
by a footnote.
See the sample file \samplefile{-FnDesc} for a complete document.
\end{example}

Note that the \pkgopt{hyperfirst} option applies to all defined glossaries.  It
may be that you only want to disable the hyperlinks on
\gls{firstuse} for glossaries that have a different form on
\gls*{firstuse}. This can be achieved by noting that since the
entries that require hyperlinking for all instances have identical
first and subsequent text, they can be unset via \ics{glsunsetall}
(see \sectionref{sec:glsunset}) so that the \pkgopt{hyperfirst}
option doesn't get applied.

\begin{example}{Suppressing Hyperlinks on First Use Just For
Acronyms}{ex:hyperfirst}
Suppose I want to suppress the hyperlink on \gls{firstuse} for
acronyms but not for entries in the main glossary.  I~can load
the \styfmt{glossaries} package using:
\begin{verbatim}
\usepackage[hyperfirst=false,acronym]{glossaries}
\end{verbatim}
Once all glossary entries have been defined I~then do:
\begin{verbatim}
\glsunsetall[main]
\end{verbatim}

\end{example}

For more complex requirements, you might find it easier to switch
off all hyperlinks via \ics{glsdisablehyper} and put the hyperlinks
(where required) within the definition of \ics{glsentryfmt} (see
\sectionref{sec:glsdisplay}) via \ics{glshyperlink} (see
\sectionref{sec:glsnolink}).

\begin{example}{Only Hyperlink in Text Mode Not Math
Mode}{ex:nomathhyper}
This is a bit of a contrived example, but suppose, for some reason,
I only want the \glslike\ commands to have
hyperlinks when used in text mode, but not in math mode. I~can do
this by adding the glossary to the list of nohypertypes and redefining 
\ics{glsentryfmt}:
\begin{verbatim}
\GlsDeclareNoHyperList{main}

\renewcommand*{\glsentryfmt}{%
  \ifmmode
    \glsgenentryfmt
  \else
    \glsifhyperon
    {\glsgenentryfmt}% hyperlink already on
    {\glshyperlink[\glsgenentryfmt]{\glslabel}}%
  \fi
}
\end{verbatim}
Note that this doesn't affect the \glstextlike\ commands, which will
have the hyperlinks off unless they're forced on using the plus
variant.

See the sample file \samplefile{-nomathhyper} for a complete
document.
\end{example}

\chapter{Adding an Entry to the Glossary Without Generating Text}
\label{sec:glsadd}

It is possible to add a line to the glossary file without
generating any text at that point in the document using:
\begin{definition}[\DescribeMacro{\glsadd}]
\cs{glsadd}\oarg{options}\marg{label}
\end{definition}
This is similar to the \glstextlike\ commands, only it doesn't produce
any text (so therefore, there is no \gloskey[glslink]{hyper} key
available in \meta{options} but all the other options that can
be used with \glstextlike\ commands can be passed to \cs{glsadd}).
For example, to add a page range to the glossary number list for the
entry whose label is given by \texttt{set}:
\begin{verbatim}
\glsadd[format=(]{set}
Lots of text about sets spanning many pages.
\glsadd[format=)]{set}
\end{verbatim}

To add all entries that have been defined, use:
\begin{definition}[\DescribeMacro{\glsaddall}]
\cs{glsaddall}\oarg{options}
\end{definition}
The optional argument is the same as for \cs{glsadd}, except
there is also a key \gloskey[glsaddall]{types} which can be
used to specify which glossaries to use. This should be a
comma separated list. For example, if you only want to add
all the entries belonging to the list of acronyms (specified by
the glossary type \ics{acronymtype}) and a list of
notation (specified by the glossary type \texttt{notation}) then you can
do:
\begin{verbatim}
\glsaddall[types={\acronymtype,notation}]
\end{verbatim}

\begin{important}
Note that \cs{glsadd} and \cs{glsaddall} add the current location to
the \gls{numberlist}. In the case of \cs{glsaddall}, all entries in the
glossary will have the same location in the number list. If you want
to use \cs{glsaddall}, it's best to suppress the number list with
the \pkgopt{nonumberlist} package option. (See
sections~\ref{sec:pkgopts-printglos} and~\ref{sec:numberlists}.)
\end{important}

There is now a variation of \cs{glsaddall} that skips any entries
that have already been used:
\begin{definition}[\DescribeMacro\glsaddallunused]
\cs{glsaddallunused}\oarg{list}
\end{definition}
This command uses \cs{glsadd}\texttt{[format=@gobble]} which will
ignore this location in the number list. The optional argument
\meta{list} is a comma-separated list of glossary types. If omitted,
it defaults to the list of all defined glossaries.

If you want to use \cs{glsaddallunused}, it's best to place the
command at the end of the document to ensure that all the commands
you intend to use have already been used. Otherwise you could end up
with a spurious comma or dash in the location list.

\begin{example}{Dual Entries}{ex:dual}
The example file \samplefile{-dual} makes use of \cs{glsadd} to
allow for an entry that should appear both in the main glossary and
in the list of acronyms. This example sets up the list of acronyms
using the \pkgopt{acronym} package option:
\begin{verbatim}
\usepackage[acronym]{glossaries}
\end{verbatim}
A new command is then defined to make it easier to define dual
entries:
\begin{verbatim}
\newcommand*{\newdualentry}[5][]{%
  \newglossaryentry{main-#2}{name={#4},%
  text={#3\glsadd{#2}},%
  description={#5},%
  #1
  }%
  \newacronym{#2}{#3\glsadd{main-#2}}{#4}%
}
\end{verbatim}
This has the following syntax:
\begin{alltt}
\ics{newdualentry}\oarg{options}\marg{label}\marg{abbrv}\marg{long}\marg{description}
\end{alltt}
You can then define a new dual entry:
\begin{verbatim}
\newdualentry{svm}% label
  {SVM}% abbreviation
  {support vector machine}% long form
  {Statistical pattern recognition technique}% description
\end{verbatim}
Now you can reference the acronym with \verb|\gls{svm}| or you can
reference the entry in the main glossary with \verb|\gls{main-svm}|.
\end{example}

\chapter{Cross-Referencing Entries}
\label{sec:crossref}

\begin{important}
You must use \ics{makeglossaries} (\optsor23) or
\ics{makenoidxglossaries} (\opt1) \emph{before} defining any
terms that cross-reference entries. If any of the terms that you have
cross-referenced don't appear in the glossary, check that you have
put \cs{makeglossaries}\slash\cs{makenoidxglossaries} before all 
entry definitions.
\end{important}

There are several ways of cross-referencing entries in the 
glossary: 

\begin{enumerate}
\item You can use commands such as \ics{gls} in the
entries description. For example:
\begin{verbatim}
\newglossaryentry{apple}{name=apple,
description={firm, round fruit. See also \gls{pear}}}
\end{verbatim}
Note that with this method, if you don't use the cross-referenced term 
in the main part of the document, you will need two runs of
\gls{makeglossaries}:
\begin{verbatim}
latex filename
makeglossaries filename
latex filename
makeglossaries filename
latex filename
\end{verbatim}

\item As described in \sectionref{sec:newglosentry}, you can use the
\gloskey{see} key when you define the entry. For example:
\begin{verbatim}
\newglossaryentry{MaclaurinSeries}{name={Maclaurin 
series},
description={Series expansion},
see={TaylorsTheorem}}
\end{verbatim}
Note that in this case, the entry with the \gloskey{see} key will
automatically be added to the glossary, but the cross-referenced
entry won't. You therefore need to ensure that you use the
cross-referenced term with the commands described in
\sectionref{sec:glslink} or \sectionref{sec:glsadd}.

The \qt{see} tag is produce using \ics{seename}, but can be
overridden in specific instances using square brackets
at the start of the \gloskey{see} value. For example:
\begin{verbatim}
\newglossaryentry{MaclaurinSeries}{name={Maclaurin 
series},
description={Series expansion},
see=[see also]{TaylorsTheorem}}
\end{verbatim}
Take care if you want to use the optional argument of commands such
as \ics{newacronym} or \ics{newterm} as the value will need to be
grouped. For example:
\begin{verbatim}
\newterm{seal}
\newterm[see={[see also]seal}]{sea lion}
\end{verbatim}
Similarly if the value contains a list. For example:
\begin{verbatim}
\glossaryentry{lemon}{
  name={lemon},
  description={Yellow citrus fruit}
}
\glossaryentry{lime}
{
  name={lime},
  description={Green citrus fruit}
}
\glossaryentry{citrus}
{
  name={citrus},
  description={Plant in the Rutaceae family},
  see={lemon,lime}
}
\end{verbatim}

\item After you have defined the entry, use
\begin{definition}[\DescribeMacro{\glssee}]
\cs{glssee}\oarg{tag}\marg{label}\marg{xr label list}
\end{definition}
where \meta{xr label list} is a comma-separated list of entry 
labels to be cross-referenced, \meta{label} is the label of the
entry doing the cross-referencing and \meta{tag} is the \qt{see} tag.
(The default value of \meta{tag} is \ics{seename}.)
For example:
\begin{verbatim}
\glssee[see also]{series}{FourierSeries,TaylorsTheorem}
\end{verbatim}
Note that this automatically adds the entry given by \meta{label} to
the glossary but doesn't add the cross-referenced entries (specified
by \meta{xr label list}) to the glossary.
\end{enumerate}

In both cases~2 and 3 above, the cross-referenced information
appears in the \gls{numberlist}, whereas in case~1, the
cross-referenced information appears in the description. (See the
\samplefile{-crossref} example file that comes with this package.)
This means that in cases~2 and~3, the cross-referencing information
won't appear if you have suppressed the \gls*{numberlist}. In this
case, you will need to activate the \gls*{numberlist} for the given
entries using \gloskey{nonumberlist=false}. Alternatively, if you
just use the \gloskey{see} key instead of \ics{glssee}, you can
automatically activate the \gls*{numberlist} using the
\pkgopt{seeautonumberlist} package option.

\section{Customising Cross-reference Text}
\label{sec:customxr}

When you use either the \gloskey{see} key or the command
\cs{glssee}, the cross-referencing information will be typeset in the 
glossary according to:
\begin{definition}[\DescribeMacro{\glsseeformat}]
\cs{glsseeformat}\oarg{tag}\marg{label-list}\marg{location}
\end{definition}
The default definition of \cs{glsseeformat} is:
\begin{display}
\cs{emph}\marg{tag} \cs{glsseelist}\marg{label-list}
\end{display}
Note that the location is always ignored.\footnote{\gls{makeindex}
will always assign a location number, even if it's not needed, so it
needs to be discarded.} For example, if you want the tag to appear
in bold, you can do:\footnote{If you redefine \cs{glsseeformat},
keep the default value of the optional argument as \ics{seename} as
both \gloskey{see} and \cs{glssee} explicitly write
\texttt[\cs{seename}\texttt] in the output file if no optional
argument is given.}
\begin{verbatim}
\renewcommand*{\glsseeformat}[3][\seename]{\textbf{#1} 
 \glsseelist{#2}}
\end{verbatim}


The list of labels is dealt with by \cs{glsseelist}, which iterates
through the list and typesets each entry in the label. The entries
are separated by
\begin{definition}[\DescribeMacro{\glsseesep}]
\cs{glsseesep}
\end{definition}
or (for the last pair)
\begin{definition}[\DescribeMacro{\glsseelastsep}]
\cs{glsseelastsep}
\end{definition}
These default to ``",\space"'' and
\qt{\cs{space}\ics{andname}\cs{space}} respectively. The list entry text
is displayed using:
\begin{definition}[\DescribeMacro{\glsseeitemformat}]
\cs{glsseeitemformat}\marg{label}
\end{definition}
This defaults to \ics{glsentrytext}\marg{label}.\footnote{In
versions before 3.0, \cs{glsentryname} was used, but this could cause
problems when the \gloskey{name} key was \glsd{sanitize}.} For example, to
make the cross-referenced list use small caps:
\begin{verbatim}
\renewcommand{\glsseeitemformat}[1]{%
  \textsc{\glsentrytext{#1}}}
\end{verbatim}

\begin{important}
You can use \ics{glsseeformat} and \ics{glsseelist} in the main body
of the text, but they won't automatically add the cross-referenced
entries to the glossary. If you want them added with that location,
you can do:
\begin{verbatim}
Some information (see also 
\glsseelist{FourierSeries,TaylorsTheorem}%
\glsadd{FourierSeries}\glsadd{TaylorsTheorem}).
\end{verbatim}
\end{important}

\chapter{Using Glossary Terms Without Links}
\label{sec:glsnolink}

The commands described in this section display entry details without
adding any information to the glossary. They don't use
\ics{glstextformat}, they don't have any optional arguments, they
don't affect the \firstuseflag\ and, apart from \ics{glshyperlink},
they don't produce hyperlinks.

\begin{important}
Commands that aren't expandable will be ignored by PDF bookmarks, so
you will need to provide an alternative via \sty{hyperref}'s
\cs{texorpdfstring} if you want to use them in sectioning commands.
(This isn't specific to the \styfmt{glossaries} package.) See the
\sty{hyperref} documentation for further details. All the commands
that convert the first letter to upper case aren't expandable.
The other commands depend on whether their corresponding keys were
assigned non-expandable values.
\end{important}

\begin{definition}[\DescribeMacro{\glsentryname}]
\cs{glsentryname}\marg{label}
\end{definition}
\begin{definition}[\DescribeMacro{\Glsentryname}]
\cs{Glsentryname}\marg{label}
\end{definition}
These commands display the name of the glossary entry given by
\meta{label}, as specified by the \gloskey{name} key.
\cs{Glsentryname} makes the first letter upper case.
Neither of these commands check for the existence of \meta{label}.
The first form \cs{glsentryname} is expandable (unless the name
contains unexpandable commands).

\begin{definition}[\DescribeMacro{\glossentryname}]
\cs{glossentryname}\marg{label}
\end{definition}
This is like \cs{glsnamefont}\{\cs{glsentryname\marg{label}}\}
but also checks for the existence of \meta{label}. This command is
not expandable. It's used in the predefined glossary styles, so if
you want to change the way the name is formatted in the glossary,
you can redefine \cs{glsnamefont} to use the required fonts. For
example:
\begin{verbatim}
\renewcommand*{\glsnamefont}[1]{\textmd{\sffamily #1}}
\end{verbatim}
\begin{definition}[\DescribeMacro{\Glossentryname}]
\cs{Glossentryname}\marg{label}
\end{definition}
This is like \cs{glossentryname} but makes the first letter of the
name upper case.

\begin{definition}[\DescribeMacro{\glsentrytext}]
\cs{glsentrytext}\marg{label}
\end{definition}
\begin{definition}[\DescribeMacro{\Glsentrytext}]
\cs{Glsentrytext}\marg{label}
\end{definition}
These commands display the subsequent use text for the glossary
entry given by \meta{label}, as specified by the \gloskey{text} key.
\cs{Glsentrytext} makes the first letter upper case.
The first form is expandable (unless the text contains unexpandable
commands). The second form is not expandable. Neither checks for the
existence of \meta{label}.

\begin{definition}[\DescribeMacro{\glsentryplural}]
\cs{glsentryplural}\marg{label}
\end{definition}
\begin{definition}[\DescribeMacro{\Glsentryplural}]
\cs{Glsentryplural}\marg{label}
\end{definition}
These commands display the subsequent use plural text for the
glossary entry given by \meta{label}, as specified by the
\gloskey{plural} key. \cs{Glsentryplural} makes the first letter
upper case.
The first form is expandable (unless the value of that key contains unexpandable
commands). The second form is not expandable. Neither checks for the
existence of \meta{label}.

\begin{definition}[\DescribeMacro{\glsentryfirst}]
\cs{glsentryfirst}\marg{label}
\end{definition}
\begin{definition}[\DescribeMacro{\Glsentryfirst}]
\cs{Glsentryfirst}\marg{label}
\end{definition}
These commands display the \firstusetext\ for the glossary entry
given by \meta{label}, as specified by the \gloskey{first} key.
\cs{Glsentryfirst} makes the first letter upper case.
The first form is expandable (unless the value of that key contains unexpandable
commands). The second form is not expandable. Neither checks for the
existence of \meta{label}.

\begin{definition}[\DescribeMacro{\glsentryfirstplural}]
\cs{glsentryfirstplural}\marg{label}
\end{definition}
\begin{definition}[\DescribeMacro{\Glsentryfirstplural}]
\cs{Glsentryfirstplural}\marg{label}
\end{definition}
These commands display the plural form of the \firstusetext\ for the
glossary entry given by \meta{label}, as specified by the
\gloskey{firstplural} key. \cs{Glsentryfirstplural} makes the first
letter upper case.
The first form is expandable (unless the value of that key contains unexpandable
commands). The second form is not expandable. Neither checks for the
existence of \meta{label}.

\begin{definition}[\DescribeMacro{\glsentrydesc}]
\cs{glsentrydesc}\marg{label}
\end{definition}
\begin{definition}[\DescribeMacro{\Glsentrydesc}]
\cs{Glsentrydesc}\marg{label}
\end{definition}
These commands display the description for the glossary entry given
by \meta{label}. \cs{Glsentrydesc} makes the first letter upper case.
The first form is expandable (unless the value of that key contains unexpandable
commands). The second form is not expandable. Neither checks for the
existence of \meta{label}.

\begin{definition}[\DescribeMacro{\glossentrydesc}]
\cs{glossentrydesc}\marg{label}
\end{definition}
This is like \cs{glsentrydesc}\marg{label}
but also checks for the existence of \meta{label}. This command is
not expandable. It's used in the predefined glossary styles to
display the description.
\begin{definition}[\DescribeMacro{\Glossentrydesc}]
\cs{Glossentrydesc}\marg{label}
\end{definition}
This is like \cs{glossentrydesc} but converts the first letter to
upper case. This command is not expandable.

\begin{definition}[\DescribeMacro{\glsentrydescplural}]
\cs{glsentrydescplural}\marg{label}
\end{definition}
\begin{definition}[\DescribeMacro{\Glsentrydescplural}]
\cs{Glsentrydescplural}\marg{label}
\end{definition}
These commands display the plural description for the glossary entry
given by \meta{label}. \cs{Glsentrydescplural} makes the first
letter upper case.
The first form is expandable (unless the value of that key contains unexpandable
commands). The second form is not expandable. Neither checks for the
existence of \meta{label}.


\begin{definition}[\DescribeMacro{\glsentrysymbol}]
\cs{glsentrysymbol}\marg{label}
\end{definition}
\begin{definition}[\DescribeMacro{\Glsentrysymbol}]
\cs{Glsentrysymbol}\marg{label}
\end{definition}
These commands display the symbol for the glossary entry given by
\meta{label}. \cs{Glsentrysymbol} makes the first letter upper case.
The first form is expandable (unless the value of that key contains unexpandable
commands). The second form is not expandable. Neither checks for the
existence of \meta{label}.


\begin{definition}[\DescribeMacro\glsletentryfield]
\cs{glsletentryfield}\marg{cs}\marg{label}\marg{field}
\end{definition}
This command doesn't display anything. It merely fetches the
value associated with the given field (where the available field names 
are listed in \tableref{tab:fieldmap}) and stores the result
in the control sequence \meta{cs}. For example, to store
the description for the entry whose label is \qt{apple} in the
control sequence \verb|\tmp|:
\begin{verbatim}
\glsletentryfield{\tmp}{apple}{desc}
\end{verbatim}

\begin{definition}[\DescribeMacro{\glossentrysymbol}]
\cs{glossentrysymbol}\marg{label}
\end{definition}
This is like \cs{glsentrysymbol}\marg{label}
but also checks for the existence of \meta{label}. This command is
not expandable. It's used in some of the predefined glossary styles to
display the symbol.
\begin{definition}[\DescribeMacro{\Glossentrysymbol}]
\cs{Glossentrysymbol}\marg{label}
\end{definition}
This is like \cs{glossentrysymbol} but converts the first letter to
upper case. This command is not expandable.

\begin{definition}[\DescribeMacro{\glsentrysymbolplural}]
\cs{glsentrysymbolplural}\marg{label}
\end{definition}
\begin{definition}[\DescribeMacro{\Glsentrysymbolplural}]
\cs{Glsentrysymbolplural}\marg{label}
\end{definition}
These commands display the plural symbol for the glossary entry
given by \meta{label}. \cs{Glsentrysymbolplural} makes the first
letter upper case.
The first form is expandable (unless the value of that key contains unexpandable
commands). The second form is not expandable. Neither checks for the
existence of \meta{label}.

\begin{definition}[\DescribeMacro{\glsentryuseri}]
\cs{glsentryuseri}\marg{label}
\end{definition}
\begin{definition}[\DescribeMacro{\Glsentryuseri}]
\cs{Glsentryuseri}\marg{label}
\end{definition}
\begin{definition}[\DescribeMacro{\glsentryuserii}]
\cs{glsentryuserii}\marg{label}
\end{definition}
\begin{definition}[\DescribeMacro{\Glsentryuserii}]
\cs{Glsentryuserii}\marg{label}
\end{definition}
\begin{definition}[\DescribeMacro{\glsentryuseriii}]
\cs{glsentryuseriii}\marg{label}
\end{definition}
\begin{definition}[\DescribeMacro{\Glsentryuseriii}]
\cs{Glsentryuseriii}\marg{label}
\end{definition}
\begin{definition}[\DescribeMacro{\glsentryuseriv}]
\cs{glsentryuseriv}\marg{label}
\end{definition}
\begin{definition}[\DescribeMacro{\Glsentryuseriv}]
\cs{Glsentryuseriv}\marg{label}
\end{definition}
\begin{definition}[\DescribeMacro{\glsentryuserv}]
\cs{glsentryuserv}\marg{label}
\end{definition}
\begin{definition}[\DescribeMacro{\Glsentryuserv}]
\cs{Glsentryuserv}\marg{label}
\end{definition}
\begin{definition}[\DescribeMacro{\glsentryuservi}]
\cs{glsentryuservi}\marg{label}
\end{definition}
\begin{definition}[\DescribeMacro{\Glsentryuservi}]
\cs{Glsentryuservi}\marg{label}
\end{definition}
These commands display the value of the user keys for the glossary
entry given by \meta{label}.
The lower case forms are expandable (unless the value of the key
contains unexpandable commands). The commands beginning with an
upper case letter convert the first letter of the required value to
upper case and are not expandable. None of these commands check for
the existence of \meta{label}.

\begin{definition}[\DescribeMacro{\glshyperlink}]
\cs{glshyperlink}\oarg{link text}\marg{label}
\end{definition}
This command provides a hyperlink to the glossary entry given by
\meta{label} \textbf{but does not add any information to the
glossary file}. The link text is given by
\ics{glsentrytext}\marg{label} by default\footnote{versions before
3.0 used \ics{glsentryname} as the default, but this could cause
problems when \gloskey{name} had been \glsd{sanitize}.}, but can be
overridden using the optional argument. Note that the hyperlink will
be suppressed if you have used \ics{glsdisablehyper} or if you
haven't loaded the \sty{hyperref} package.

\begin{important}
If you use \cs{glshyperlink}, you need to ensure that the relevant
entry has been added to the glossary using any of the commands
described in \sectionref{sec:glslink} or \sectionref{sec:glsadd}
otherwise you will end up with an undefined link.
\end{important}

The next two commands are only available with \opt1 or with the
\pkgopt{savenumberlist} package option:
\begin{definition}[\DescribeMacro{\glsentrynumberlist}]
\cs{glsentrynumberlist}\marg{label}
\end{definition}
\begin{definition}[\DescribeMacro{\glsdisplaynumberlist}]
\cs{glsdisplaynumberlist}\marg{label}
\end{definition}
Both display the \gls{numberlist} for the entry given by
\meta{label}. When used with \opt1 a~rerun is required to ensure
this list is up-to-date, when used with \optsor23
a run of \gls{makeglossaries} (or \gls{makeindex}\slash\gls{xindy})
followed by one or two runs of \LaTeX\ is required.

The first command, \cs{glsentrynumberlist}, simply displays
the number list as is. The second command,
\linebreak\cs{glsdisplaynumberlist}, formats the list using:
\begin{definition}[\DescribeMacro{\glsnumlistsep}]
\cs{glsnumlistsep}
\end{definition}
as the separator between all but the last two elements and
\begin{definition}[\DescribeMacro{\glsnumlistlastsep}]
\cs{glsnumlistlastsep}
\end{definition}
between the final two elements. The defaults are
\verb*|, | and \verb*| \& |, respectively.

\begin{important}
\cs{glsdisplaynumberlist} is fairly experimental. It works with
\opt1, but for \optsor23 it only works when the default counter
format is used (that is, when the \gloskey[glslink]{format} key is
set to \texttt{glsnumberformat}).  This command will only work with
\sty{hyperref} if you choose \opt1.  If you try using this command
with \optsor23 and \sty{hyperref}, \cs{glsentrynumberlist} will be
used instead.
\end{important}

For further information see \ifpdf section~\ref*{sec:code:glsnolink}
\fi \qt{Displaying entry details without adding information to the
glossary} in the documented code (\texttt{glossaries-code.pdf}).

\chapter{Displaying a glossary}
\label{sec:printglossary}

\begin{description}
\itempar{\opt1:}
\begin{definition}[\DescribeMacro{\printnoidxglossaries}]
\cs{printnoidxglossaries}
\end{definition}
(Must be used with \ics{makenoidxglossaries} in the preamble.)

\itempar{\optsand23:}
\begin{definition}[\DescribeMacro{\printglossaries}]
\cs{printglossaries}
\end{definition}
(Must be used with \ics{makeglossaries} in the preamble.)
\end{description}

These commands will display all the glossaries in the order in which they were
defined. Note that, in the case of \optsand23, no glossaries will appear until you have either
used the Perl script \gls{makeglossaries} or have directly used
\gls{makeindex} or \gls{xindy} (as described in
\sectionref{sec:makeglossaries}). If the glossary 
still does not appear after you re-\LaTeX\ your document, check the
\gls*{makeindex}\slash\gls*{xindy} log files to see if there is a problem.
With \opt1, you just need two \LaTeX\ runs to make the glossaries appear, 
but you may need further runs to make the \glspl{numberlist} up-to-date.

An individual glossary can be displayed using:
\begin{description}
\itempar{\opt1:}
\begin{definition}[\DescribeMacro{\printnoidxglossary}]
\cs{printnoidxglossary}\oarg{options}
\end{definition}
(Must be used with \ics{makenoidxglossaries} in the preamble.)

\itempar{\optsand23:}
\begin{definition}[\DescribeMacro{\printglossary}]
\cs{printglossary}\oarg{options}
\end{definition}
(Must be used with \ics{makeglossaries} in the preamble.)
\end{description}
where \meta{options} is a \meta{key}=\meta{value} list of options. 
The following keys are available:
\begin{description}
\item[{\gloskey[printglossary]{type}}] The value of this key
specifies which glossary to print. If omitted, the default
glossary is assumed. For example, to print the list of acronyms:
\begin{verbatim}
\printglossary[type=\acronymtype]
\end{verbatim}
Note that you can't display an ignored glossary, so don't try
setting \gloskey[printglossary]{type} to the name of a glossary that
was defined using \ics{newignoredglossary}. (See
\sectionref{sec:newglossary}.)

\item[{\gloskey[printglossary]{title}}] This is the glossary's
title (overriding the title specified when the glossary was 
defined).

\item[{\gloskey[printglossary]{toctitle}}] This is the title
to use for the table of contents (if the \pkgopt{toc} package
option has been used). It may also be used for the page header,
depending on the page style. If omitted, the value of 
\gloskey[printglossary]{title} is used.

\item[{\gloskey[printglossary]{style}}] This specifies which
glossary style to use for this glossary, overriding the effect
of the \pkgopt{style} package option or \ics{glossarystyle}.

\item[{\gloskey[printglossary]{numberedsection}}] This specifies
whether to use a numbered section for this glossary, overriding
the effect of the \pkgopt{numberedsection} package option. This
key has the same syntax as the \pkgopt{numberedsection} package
option, described in \sectionref{sec:pkgopts-sec}.

\item[{\gloskey[printglossary]{nonumberlist}}] This is a boolean
key. If true (\verb|nonumberlist=true|) the numberlist is suppressed
for this glossary. If false \linebreak(\verb|nonumberlist=false|) the
numberlist is displayed for this glossary.

\item[{\gloskey[printglossary]{nogroupskip}}] This is a boolean
key. If true the vertical gap between groups 
is suppressed for this glossary.

\item[{\gloskey[printglossary]{nopostdot}}] This is a boolean
key. If true the full stop after the description 
is suppressed for this glossary.

\item[{\gloskey[printglossary]{entrycounter}}] This is a boolean
key. Behaves similar to the package option of the same name.
The corresponding package option must be used to make \ics{glsrefentry}
work correctly.

\item[{\gloskey[printglossary]{subentrycounter}}] This is a boolean
key. Behaves similar to the package option of the same name.  If you
want to set both \gloskey{entrycounter} and
\gloskey{subentrycounter}, make sure you specify
\gloskey{entrycounter} first.  The corresponding package option must
be used to make \ics{glsrefentry} work correctly.

\item[{\gloskey[printnoidxglossary]{sort}}] This key is only
available for \opt1. Possible values are: \texttt{word} (word
order), \texttt{letter} (letter order), \texttt{standard} (word or
letter ordering taken from the \pkgopt{order} package option),
\texttt{use} (order of use), \texttt{def} (order of definition)
\texttt{nocase} (case-insensitive) or \texttt{case} (case-sensitive).

{\raggedright
The word and letter order sort methods use
\sty{datatool}'s
\ics{dtlwordindexcompare} and \ics{dtlletterindexcompare} handlers.
The case-insensitive sort method uses \sty{datatool}'s
\ics{dtlicompare} handler.  The case-sensitive sort method uses 
\sty{datatool}'s \ics{dtlcompare} handler. See the \sty{datatool}
documentation for further details.
\par}

If you don't get an error with \texttt{sort=use} and
\texttt{sort=def} but you do get an error with one of the other sort
options, then you probably need to use the
\pkgopt[true]{sanitizesort} package option or make sure none of the
entries have fragile commands in their \gloskey{sort} field.
\end{description}

By default, the glossary is started either by \ics{chapter*} or by
\ics{section*}, depending on whether or not \ics{chapter} is defined.
This can be overridden by the \pkgopt{section} package option or the
\linebreak\ics{setglossarysection} command. Numbered sectional units can be
obtained using the \pkgopt{numberedsection} package option. Each
glossary sets the page header via the command
\begin{definition}[\DescribeMacro\glsglossarymark]
\ics{glsglossarymark}\marg{title}
\end{definition}
If this mechanism is unsuitable for your chosen class file or page
style package, you will need to redefine \ics{glsglossarymark}. Further
information about these options and commands is given in
\sectionref{sec:pkgopts-sec}.

Information can be added to the start of the glossary (after the
title and before the main body of the glossary) by redefining
\begin{definition}[\DescribeMacro{\glossarypreamble}]
\cs{glossarypreamble}
\end{definition}
For example:
\begin{verbatim}
\renewcommand{\glossarypreamble}{Numbers in italic 
indicate primary definitions.}
\end{verbatim}
This needs to be done before the glossary is displayed.

If you want a different preamble per glossary you can use
\begin{definition}[\DescribeMacro{\setglossarypreamble}]
\cs{setglossarypreamble}\oarg{type}\marg{preamble text}
\end{definition}
If \meta{type} is omitted, \cs{glsdefaulttype} is used.

For example:
\begin{verbatim}
\setglossarypreamble{Numbers in italic 
indicate primary definitions.}
\end{verbatim}
This will print the given preamble text for the main glossary,
but not have any preamble text for any other glossaries.

There is an analogous command to \cs{glossarypreamble} called
\begin{definition}[\DescribeMacro{\glossarypostamble}]
\cs{glossarypostamble}
\end{definition}
which is placed at the end of each glossary.

\begin{example}{Switch to Two Column Mode for Glossary}{ex:twocolumn}
Suppose you are using the \glostyle{superheaderborder}
style\footnote{you can't use the \glostyle{longheaderborder} style
for this example as you can't use the \env{longtable} environment in
two column mode.}, and you want the glossary to be in two columns,
but after the glossary you want to switch back to one column mode,
you could do:
\begin{verbatim}
\renewcommand*{\glossarysection}[2][]{%
  \twocolumn[{\chapter*{#2}}]%
  \setlength\glsdescwidth{0.6\linewidth}%
  \glsglossarymark{\glossarytoctitle}%
}

\renewcommand*{\glossarypostamble}{\onecolumn}
\end{verbatim}

\end{example}

Within each glossary, each entry name is formatted according to
\begin{definition}[\DescribeMacro{\glsnamefont}]
\cs{glsnamefont}\marg{name}
\end{definition}
which takes one argument: the entry name. This command is always
used regardless of the glossary style. By default, \cs{glsnamefont}
simply displays its argument in whatever the surrounding font
happens to be. This means that in the list-like glossary styles
(defined in the \sty{glossary-list} style file) the name will appear
in bold, since the name is placed in the optional argument of
\cs{item}, whereas in the tabular styles (defined in the
\sty{glossary-long} and \sty{glossary-super} style files) the name
will appear in the normal font. The hierarchical glossary styles
(defined in the \sty{glossary-tree} style file) also set the name in
bold.

\begin{example}{Changing the Font Used to Display Entry Names in the
Glossary}{ex:glsnamefont}
Suppose you want all the entry names to appear in 
medium weight small caps in your glossaries, then you can do:
\begin{verbatim}
\renewcommand{\glsnamefont}[1]{\textsc{\mdseries #1}}
\end{verbatim}
\end{example}

\chapter[Xindy (Option 3)]{Xindy (\ifpdf\opt3\else Option 3\fi)}
\label{sec:xindy}

If you want to use \gls{xindy} to sort the glossary, you
must use the package option \pkgopt{xindy}:
\begin{verbatim}
\usepackage[xindy]{glossaries}
\end{verbatim}
This ensures that the glossary information is written in 
\gls*{xindy} syntax.

\sectionref{sec:makeglossaries} covers how to use the external
\gls{indexingapp}. This section covers the commands provided
by the \styfmt{glossaries} package that allow you to adjust the
\gls{xindy} style file (\filetype{.xdy}) and parameters.

To assist writing information to the \gls{xindy} style
file, the \styfmt{glossaries} package provides the following
commands:
\begin{definition}[\DescribeMacro{\glsopenbrace}]
\cs{glsopenbrace}
\end{definition}
\begin{definition}[\DescribeMacro{\glsclosebrace}]
\cs{glsclosebrace}
\end{definition}
which produce an open and closing brace. (This is needed because
\verb|\{| and \verb|\}| don't expand to a simple brace character
when written to a file.)

In addition, if you are using a package that makes the double
quote character active (e.g. \sty{ngerman}) you can use:
\begin{definition}[\DescribeMacro{\glsquote}]
\cs{glsquote}\marg{text}
\end{definition}
which will produce \verb|"|\meta{text}\verb|"|. Alternatively,
you can use \verb|\string"| to write the double-quote character. 
This document assumes that the double quote character has not been
made active, so the examples just use \verb|"| for clarity.

If you want greater control over the \gls{xindy} style file than is
available through the \LaTeX\ commands provided by the
\styfmt{glossaries} package, you will need to edit the \gls*{xindy}
style file. In which case, you must use \ics{noist} to prevent the
style file from being overwritten by the \styfmt{glossaries}
package. For additional information about \gls*{xindy}, read the
\gls*{xindy} documentation. I'm sorry I can't provide any assistance
with writing \gls*{xindy} style files. If you need help, I recommend
you ask on the \gls*{xindy} mailing list
(\url{http://xindy.sourceforge.net/mailing-list.html}).

\section{Language and Encodings}
\label{sec:langenc}

When you use \gls{xindy}, you need to specify the language
and encoding used (unless you have written your own custom
\gls*{xindy} style file that defines the relevant alphabet
and sort rules). If you use \gls{makeglossaries},
this information is obtained from the document's auxiliary 
(\filetype{.aux}) file.  The \gls{makeglossaries} script attempts to 
find the root language given your document settings, but in the 
event that it gets it wrong or if \gls*{xindy} doesn't support 
that language, then you can specify the required language using:
\begin{definition}[\DescribeMacro{\GlsSetXdyLanguage}]
\cs{GlsSetXdyLanguage}\oarg{glossary type}\marg{language}
\end{definition}
where \meta{language} is the name of the language. The
optional argument can be used if you have multiple glossaries
in different languages. If \meta{glossary type} is omitted, it
will be applied to all glossaries, otherwise the language
setting will only be applied to the glossary given by
\meta{glossary type}.

If the \sty{inputenc} package is used, the encoding will be
obtained from the value of \ics{inputencodingname}. 
Alternatively, you can specify the encoding using:
\begin{definition}[\DescribeMacro{\GlsSetXdyCodePage}]
\cs{GlsSetXdyCodePage}\marg{code}
\end{definition}
where \meta{code} is the name of the encoding. For example:
\begin{verbatim}
\GlsSetXdyCodePage{utf8}
\end{verbatim}

Note that you can also specify the language and encoding using
the package option 
\verb|xindy={language=|\meta{lang}\verb|,codepage=|\meta{code}\verb|}|.
For example:
\begin{verbatim}
\usepackage[xindy={language=english,codepage=utf8}]{glossaries}
\end{verbatim}

If you write your own custom \gls{xindy} style file that 
includes the language settings, you need to set the language
to nothing:
\begin{verbatim}
\GlsSetXdyLanguage{}
\end{verbatim}
(and remember to use \ics{noist} to prevent the style file from
being overwritten).

\begin{important}
The commands \cs{GlsSetXdyLanguage} and \cs{GlsSetXdyCodePage}
have no effect if you don't use \gls{makeglossaries}. If
you call \gls{xindy} without \gls*{makeglossaries} you
need to remember to set the language and encoding using the
\texttt{-L} and \texttt{-C} switches.
\end{important}

\section{Locations and Number lists}
\label{sec:xindyloc}

If you use \pkgopt{xindy}, the \styfmt{glossaries} package needs to
know which counters you will be using in the \gls{numberlist} in order to correctly format the \gls{xindy} style
file. Counters specified using the \pkgopt{counter} package option
or the \meta{counter} option of \ics{newglossary} are
automatically taken care of, but if you plan to use a different
counter in the \gloskey[glslink]{counter} key for commands like 
\ics{glslink}, then you need to identify these counters \emph{before} 
\ics{makeglossaries} using:
\begin{definition}[\DescribeMacro{\GlsAddXdyCounters}]
\cs{GlsAddXdyCounters}\marg{counter list}
\end{definition}
where \meta{counter list} is a comma-separated list of counter names.

The most likely attributes used in the \gloskey[glslink]{format} 
key (\locfmt{textrm}, \locfmt{hyperrm} etc) are automatically added 
to the \gls{xindy} style file, but if you want to use another 
attribute, you need to add it using:
\begin{definition}[\DescribeMacro{\GlsAddXdyAttribute}]
\cs{GlsAddXdyAttribute}\marg{name}
\end{definition}
where \meta{name} is the name of the attribute, as used in
the \gloskey[glslink]{format} key. 

\begin{example}{Custom Font for Displaying a Location}{ex:hyperbfit}
Suppose I want a bold, italic, hyperlinked location. I first need to define a
command that will do this:
\begin{verbatim}
\newcommand*{\hyperbfit}[1]{\textit{\hyperbf{#1}}}
\end{verbatim}
but with \gls{xindy}, I also need to add this as an allowed
attribute:
\begin{verbatim}
\GlsAddXdyAttribute{hyperbfit}
\end{verbatim}
Now I can use it in the optional argument of commands like
\ics{gls}:
\begin{verbatim}
Here is a \gls[format=hyperbfit]{sample} entry.
\end{verbatim}
(where \texttt{sample} is the label of the required entry).
\end{example}

\begin{important}
Note that \cs{GlsAddXdyAttribute} has no effect if \ics{noist} is
used or if \ics{makeglossaries} is omitted.
\cs{GlsAddXdyAttribute} must be used before \ics{makeglossaries}.
Additionally, \cs{GlsAddXdyCounters} must come before
\cs{GlsAddXdyAttribute}.
\end{important}

If the location numbers don't get expanded to a simple Arabic or 
Roman number or a letter from a, \ldots, z or A, \ldots, Z, then
you need to add a location style in the appropriate format using
\begin{definition}[\DescribeMacro{\GlsAddXdyLocation}]
\cs{GlsAddXdyLocation}\oarg{prefix-location}\marg{name}\marg{definition}
\end{definition}
where \meta{name} is the name of the format and \meta{definition} is
the \gls{xindy} definition. The optional argument \meta{prefix-location}
is needed if \linebreak\cs{theH}\meta{counter} either isn't defined or is
different from \cs{the}\meta{counter}.

\begin{important}
Note that \cs{GlsAddXdyLocation} has no effect if \ics{noist} is
used or if \ics{makeglossaries} is omitted.
\cs{GlsAddXdyLocation} must be used before \ics{makeglossaries}.
\end{important}

\begin{example}{Custom Numbering System for Locations}{ex:customloc}
Suppose I decide to use a somewhat eccentric numbering
system for sections where I redefine \cs{thesection} as follows:
\begin{verbatim}
\renewcommand*{\thesection}{[\thechapter]\arabic{section}}
\end{verbatim}
If I haven't done "counter=section" in the package
option, I need to specify that the counter will be used as a
location number:
\begin{verbatim}
\GlsAddXdyCounters{section}
\end{verbatim}
Next I need to add the location style (\cs{thechapter} is assumed to
be the standard \verb"\arabic{chapter}"):
\begin{verbatim}
\GlsAddXdyLocation{section}{:sep "[" "arabic-numbers" :sep "]"
  "arabic-numbers"
}
\end{verbatim}
Note that if I have further decided to use the \sty{hyperref}
package and want to redefine \cs{theHsection} as:
\begin{verbatim}
\renewcommand*{\theHsection}{\thepart.\thesection}
\renewcommand*{\thepart}{\Roman{part}}
\end{verbatim}
then I need to modify the \cs{GlsAddXdyLocation} code above to:
\begin{verbatim}
\GlsAddXdyLocation["roman-numbers-uppercase"]{section}{:sep "[" 
  "arabic-numbers" :sep "]" "arabic-numbers"
}
\end{verbatim}
Since \ics{Roman} will result in an empty string if the counter is
zero, it's a good idea to add an extra location to catch this:
\begin{verbatim}
\GlsAddXdyLocation{zero.section}{:sep "[" 
  "arabic-numbers" :sep "]" "arabic-numbers"
}
\end{verbatim}
This example is illustrated in the sample file
\samplefile{xdy2}.
\end{example}

\begin{example}{Locations as Words not Digits}{ex:fmtcount}
Suppose I want the page numbers written as words
rather than digits and I~use the \sty{fmtcount} package to
do this. I~can redefine \ics{thepage} as follows:
\begin{verbatim}
\renewcommand*{\thepage}{\Numberstring{page}}
\end{verbatim}
This gets expanded to \verb|\protect \Numberstringnum |\marg{n}
where \meta{n} is the Arabic page number. This means that I~need to
define a new location that has that form:
\begin{verbatim}
\GlsAddXdyLocation{Numberstring}{:sep "\string\protect\space
  \string\Numberstringnum\space\glsopenbrace"
  "arabic-numbers" :sep "\glsclosebrace"}
\end{verbatim}
Note that it's necessary to use \cs{space} to indicate that 
spaces also appear in the format, since, unlike \TeX,
\gls{xindy} doesn't ignore spaces after control sequences.

Note that \cs{GlsAddXdyLocation}\marg{name}\marg{definition} will define 
commands in the form:
\begin{definition}
\cs{glsX}\meta{counter}"X"\meta{name}\marg{Hprefix}\marg{location}
\end{definition}
for each counter that has been identified either by the
\pkgopt{counter} package option, the \meta{counter} option for
\ics{newglossary} or in the argument of \ics{GlsAddXdyCounters}.

The first argument \meta{Hprefix} is only relevant when used with
the \sty{hyperref} package and indicates that \cs{the}\meta{Hcounter}
is given by \cs{Hprefix}"."\cs{the}\meta{counter}. The sample
file \samplefile{xdy}, which comes with the \styfmt{glossaries}
package, uses the default \ctr{page} counter for locations, and it
uses the default \ics{glsnumberformat} and a custom \cs{hyperbfit}
format. A new \gls{xindy} location called \texttt{Numberstring}, as
illustrated above, is defined to make the page numbers appear as
\qt{One}, \qt{Two}, etc. In order for the location numbers to
hyperlink to the relevant pages, I~need to redefine the necessary
\cs{glsX}\meta{counter}"X"\meta{format} commands:
\begin{verbatim}
\renewcommand{\glsXpageXglsnumberformat}[2]{%
 \linkpagenumber#2%
}

\renewcommand{\glsXpageXhyperbfit}[2]{%
 \textbf{\em\linkpagenumber#2}%
}

\newcommand{\linkpagenumber}[3]{\hyperlink{page.#3}{#1#2{#3}}}
\end{verbatim}
\end{example}

In the \gls{numberlist}, the locations are sorted according to
type. The default ordering is: \texttt{roman-page-numbers} (e.g.\
i), \texttt{arabic-page-numbers} (e.g.\ 1),
\texttt{arabic-section-numbers} (e.g.\ 1.1 if the compositor is a
full stop or 1-1 if the compositor is a hyphen\footnote{see
\ics{setCompositor} described in \sectionref{sec:setup}}),
\texttt{alpha-page-numbers} (e.g.\ a), \texttt{Roman-page-numbers}
(e.g.\ I), \texttt{Alpha-page-numbers} (e.g.\ A),
\texttt{Appendix-page-numbers} (e.g.\ A.1 if the Alpha compositor is
a full stop or A-1 if the Alpha compositor is a hyphen\footnote{see
\ics{setAlphaCompositor} described in
\sectionref{sec:setup}}), user defined location names (as
specified by \ics{GlsAddXdyLocation} in the order in which they were
defined), \texttt{see} (cross-referenced entries). This ordering can
be changed using:

\DescribeMacro{\GlsSetXdyLocationClassOrder}
\begin{definition}
\cs{GlsSetXdyLocationClassOrder}\marg{location names}
\end{definition}
where each location name is delimited by double quote marks and
separated by white space. For example:
\begin{verbatim}
\GlsSetXdyLocationClassOrder{
  "arabic-page-numbers"
  "arabic-section-numbers"
  "roman-page-numbers"
  "Roman-page-numbers"
  "alpha-page-numbers"
  "Alpha-page-numbers"
  "Appendix-page-numbers"
  "see"
}
\end{verbatim}

\begin{important}
Note that \cs{GlsSetXdyLocationClassOrder} has no effect if 
\ics{noist} is used or if \ics{makeglossaries} is omitted.
\cs{GlsSetXdyLocationClassOrder} must be used before 
\ics{makeglossaries}.
\end{important}

If a \gls{numberlist} consists of a sequence of consecutive 
numbers, the range will be concatenated. The 
number of consecutive locations that causes a range formation 
defaults to 2, but can be changed using:\newpage

\DescribeMacro{\GlsSetXdyMinRangeLength}
\begin{definition}
\cs{GlsSetXdyMinRangeLength}\marg{n}
\end{definition}
For example:
\begin{verbatim}
\GlsSetXdyMinRangeLength{3}
\end{verbatim}
The argument may also be the keyword \texttt{none}, to indicate that
there should be no range formations. See the \gls{xindy}
manual for further details on range formations.

\begin{important}
Note that \cs{GlsSetXdyMinRangeLength} has no effect if \ics{noist}
is used or if \ics{makeglossaries} is omitted.
\cs{GlsSetXdyMinRangeLength} must be used before 
\ics{makeglossaries}.
\end{important}

See \sectionref{sec:numberlists} for further details.

\section{Glossary Groups}
\label{sec:groups}

The glossary is divided into groups according to the first
letter of the sort key. The \styfmt{glossaries} package also adds
a number group by default, unless you suppress it in the
\pkgopt{xindy} package option. For example:
\begin{verbatim}
\usepackage[xindy={glsnumbers=false}]{glossaries}
\end{verbatim}
Any entry that doesn't go in one of the letter groups or the
number group is placed in the default group.

If you have a number group, the default behaviour is to locate
it before the \qt{A} letter group. If you are not using a
Roman alphabet, you can change this using:

\DescribeMacro{\GlsSetXdyFirstLetterAfterDigits}
\begin{definition}
\cs{GlsSetXdyFirstLetterAfterDigits}\marg{letter}
\end{definition}

\begin{important}
Note that \cs{GlsSetXdyFirstLetterAfterDigits} has no effect if 
\ics{noist} is used or if \ics{makeglossaries} is omitted.
\cs{GlsSetXdyFirstLetterAfterDigits} must be used before 
\ics{makeglossaries}.\par
\end{important}

\chapter{Defining New Glossaries}
\label{sec:newglossary}

A new glossary can be defined using:
\begin{definition}[\DescribeMacro{\newglossary}]
\cs{newglossary}\oarg{log-ext}\marg{name}\marg{in-ext}\marg{out-ext}\marg{title}\linebreak\oarg{counter}
\end{definition}
where \meta{name} is the label to assign to this glossary.
The arguments \meta{in-ext} and \meta{out-ext} specify the extensions to
give to the input and output files for that glossary, \meta{title}
is the default title for this new glossary and the final optional
argument \meta{counter} specifies which counter to use for the
associated \glspl{numberlist} (see also
\sectionref{sec:numberlists}). The first optional argument specifies
the extension for the \gls{makeindex} (\opt2) or \gls{xindy} (\opt3)
transcript file (this information is only used by
\gls{makeglossaries} which picks up the information from the
auxiliary file).  If you use \opt1, the \meta{log-ext},
\meta{in-ext} and \meta{out-ext} arguments are ignored.

\begin{important}
The glossary label \meta{name} must not contain any active
characters. It's generally best to stick with just characters that
have category code~11 (typically the non-extended \glspl{latinchar}).
\end{important}

There is also a starred version (new to v4.08):
\begin{definition}[\DescribeMacro{\newglossary*}]
\cs{newglossary*}\marg{name}\marg{title}\oarg{counter}
\end{definition}
which is equivalent to
\begin{alltt}
\cs{newglossary}[\meta{name}-glg]\marg{name}\{\meta{name}-gls\}\{\meta{name}-glo\}\ifpdf\linebreak\fi\marg{title}\oarg{counter}
\end{alltt}
or you can also use:
\begin{definition}[\DescribeMacro{\altnewglossary}]
\cs{altnewglossary}\marg{name}\marg{tag}\marg{title}\oarg{counter}
\end{definition}
which is equivalent to
\begin{alltt}
\cs{newglossary}[\meta{tag}-glg]\marg{name}\{\meta{tag}-gls\}\{\meta{tag}-glo\}\marg{title}\oarg{counter}
\end{alltt}

It may be that you have some terms or acronyms that are so common
that they don't need to be listed. In this case, you can define 
a~special type of glossary that doesn't create any associated files.
This is referred to as an \qt{ignored glossary} and it's ignored by
commands that iterate over all the glossaries, such as
\ics{printglossaries}. To define an ignored glossary, use
\begin{definition}[\DescribeMacro\newignoredglossary]
\cs{newignoredglossary}\marg{name}
\end{definition}
where \meta{name} is the name of the glossary (as above). This
glossary type will automatically be added to the
\pkgopt{nohypertypes} list, since there are no hypertargets for
the entries in an ignored glossary.
(The sample file \samplefile{-entryfmt} defines an ignored glossary.)

You can test if a glossary is an ignored one using:
\begin{definition}[\DescribeMacro\ifignoredglossary]
\cs{ifignoredglossary}\marg{name}\marg{true}\marg{false}
\end{definition}
This does \meta{true} if \meta{name} was defined as an ignored
glossary, otherwise it does \meta{false}.

Note that the main (default) glossary is automatically created as:
\begin{verbatim}
\newglossary{main}{gls}{glo}{\glossaryname}
\end{verbatim}
so it can be identified by the label \texttt{main} (unless the
\pkgopt{nomain} package option is used). Using the
\pkgopt{acronym} package option is equivalent to:
\begin{verbatim}
\newglossary[alg]{acronym}{acr}{acn}{\acronymname}
\end{verbatim}
so it can be identified by the label \texttt{acronym}. If you are
not sure whether the \pkgopt{acronym} option has been used, you
can identify the list of acronyms by the command 
\DescribeMacro{\acronymtype}\cs{acronymtype} which is set to
\texttt{acronym}, if the \pkgopt{acronym} option has been used,
otherwise it is set to \texttt{main}. Note that if you are using
the main glossary as your list of acronyms, you need to declare
it as a list of acronyms using the package option 
\pkgopt{acronymlists}.

The \pkgopt{symbols} package option creates a new glossary with the
label \texttt{symbols} using:
\begin{verbatim}
\newglossary[slg]{symbols}{sls}{slo}{\glssymbolsgroupname}
\end{verbatim}
The \pkgopt{numbers} package option creates a new glossary with
the label \texttt{numbers} using:
\begin{verbatim}
\newglossary[nlg]{numbers}{nls}{nlo}{\glsnumbersgroupname}
\end{verbatim}
The \pkgopt{index} package option creates a new glossary with
the label \texttt{index} using:
\begin{verbatim}
\newglossary[ilg]{index}{ind}{idx}{\indexname}
\end{verbatim}

\begin{important}
\optsand23: all glossaries must be defined before \ics{makeglossaries} to 
ensure that the relevant output files are opened.

See \sectionref{sec:fixednames} if you want to redefine \cs{glossaryname}, 
especially if you are using \sty{babel} or \sty{translator}.
(Similarly for \cs{glssymbolsgroupname} and
\cs{glsnumbersgroupname}.) If you want to redefine \ics{indexname},
just follow the advice in
\href{http://www.tex.ac.uk/cgi-bin/texfaq2html?label=fixnam}{How to
change LaTeX’s \qt{fixed names}}.
\end{important}

\chapter{Acronyms}
\label{sec:acronyms}

You may have noticed in \sectionref{sec:newglosentry} that when you
specify a new entry, you can specify alternate text to use when the
term is \glsd{firstuse} in the document. This provides a
useful means to define acronyms. For convenience, the
\styfmt{glossaries} package defines the command:
\begin{definition}[\DescribeMacro{\newacronym}]
\cs{newacronym}\oarg{key-val list}\marg{label}\marg{abbrv}\marg{long}
\end{definition}

This uses \ics{newglossaryentry} to create an entry with the given
label in the glossary given by \ics{acronymtype}. You can specify a
different glossary using the \gloskey{type} key within the optional
argument. The \cs{newacronym} command also uses the \gloskey{long},
\gloskey{longplural}, \gloskey{short} and \gloskey{shortplural} keys
in \cs{newglossaryentry} to store the long and abbreviated forms and
their plurals. 

\begin{important}
If you haven't identified the specified glossary type as a list of
acronyms (via the package option
\pkgopt{acronymlists} or the command \ics{DeclareAcronymList}, 
see \sectionref{sec:pkgopts-acronym}) \cs{newacronym} will add it to
the list and \emph{reset the display style} for that glossary via
\ics{defglsentryfmt}. If you have a mixture of acronyms and regular
entries within the same glossary, care is needed if you want to
change the display style: you must first identify that glossary as a
list of acronyms and then use \ics{defglsentryfmt} (not redefine
\ics{glsentryfmt}) before defining your entries.
\end{important}

The optional argument \marg{key-val list} allows you to specify keys
such as \gloskey{description} (when used with one of the styles that
require a description, described in
\sectionref{sec:setacronymstyle}) or you can
override plural forms of \meta{abbrv} or \meta{long} using the
\gloskey{shortplural} or \gloskey{longplural} keys.
For example:
\begin{verbatim}
\newacronym[longplural={diagonal matrices}]%
  {dm}{DM}{diagonal matrix}
\end{verbatim}
If the \firstuse\ uses the plural form, \verb|\glspl{dm}| will
display: diagonal matrices (DMs). If you want to use
the \gloskey{longplural} or \gloskey{shortplural} keys, I recommend
you use \ics{setacronymstyle} to set the display style rather than
using one of the pre-version 4.02 acronym styles.

Since \ics{newacronym} uses \ics{newglossaryentry}, you can use
commands like \ics{gls} and \ics{glsreset} as with any other
glossary entry.

\begin{important}
Since \cs{newacronym} sets \verb|type=\acronymtype|,
if you want to load a file containing acronym definitions using
\ics{loadglsentries}\oarg{type}\marg{filename}, the optional argument
\meta{type} will not have an effect unless you explicitly set the
type as \verb|type=\glsdefaulttype| in the optional argument to
\ics{newacronym}. See \sectionref{sec:loadglsentries}.
\end{important}

\begin{example}{Defining an Acronym}{ex:newacronym}
The following defines the acronym IDN:
\begin{verbatim}
\newacronym{idn}{IDN}{identification number}
\end{verbatim}
\verb|\gls{idn}| will produce \qt{identification number (IDN)} on
\firstuse\ and \qt{IDN} on subsequent uses. If you want to use one
of the smallcaps acronym styles, described in
\sectionref{sec:setacronymstyle}, you need to use lower case
characters for the shortened form:
\begin{verbatim}
\newacronym{idn}{idn}{identification number}
\end{verbatim}
Now \verb|\gls{idn}| will produce \qt{identification number
(\textsc{idn})} on \firstuse\ and \qt{\textsc{idn}} on subsequent
uses.
\end{example}

\begin{important}
The commands described below are similar to the \glstextlike\
commands in that they don't modify the \firstuseflag.
However, their display is governed by \ics{defentryfmt} with
\ics{glscustomtext} set as appropriate. All caveats that apply to
the \glstextlike\ commands also apply to the following commands.
\end{important}

The optional arguments are the same as those for the \glstextlike\
commands, and there are similar star and plus variants that switch
off or on the hyperlinks. As with the \glstextlike\ commands, the
\gls{linktext} is placed in the argument of \ics{glstextformat}.

\begin{definition}[\DescribeMacro{\acrshort}]
\cs{acrshort}\oarg{options}\marg{label}\oarg{insert}
\end{definition}
This sets the \gls{linktext} to the short form (within the argument
of \ics{acronymfont}) for the entry given by \meta{label}. The short
form is as supplied by the \gloskey{short} key, which
\ics{newacronym} implicitly sets.

There are also analogous upper case variants:
\begin{definition}[\DescribeMacro{\Acrshort}]
\cs{Acrshort}\oarg{options}\marg{label}\oarg{insert}
\end{definition}
\begin{definition}[\DescribeMacro{\ACRshort}]
\cs{ACRshort}\oarg{options}\marg{label}\oarg{insert}
\end{definition}
There are also plural versions:
\begin{definition}[\DescribeMacro\acrshortpl]
\cs{acrshortpl}\oarg{options}\marg{label}\oarg{insert}
\end{definition}
\begin{definition}[\DescribeMacro\Acrshortpl]
\cs{Acrshortpl}\oarg{options}\marg{label}\oarg{insert}
\end{definition}
\begin{definition}[\DescribeMacro\ACRshortpl]
\cs{ACRshortpl}\oarg{options}\marg{label}\oarg{insert}
\end{definition}
The short plural form is as supplied
by the \gloskey{shortplural} key, which \ics{newacronym} implicitly sets.

\begin{definition}[\DescribeMacro{\acrlong}]
\cs{acrlong}\oarg{options}\marg{label}\oarg{insert}
\end{definition}
This sets the \gls{linktext} to the long form for the entry given by 
\meta{label}. The long form is as supplied
by the \gloskey{long} key, which \ics{newacronym} implicitly sets.

There are also analogous upper case variants:
\begin{definition}[\DescribeMacro{\Acrlong}]
\cs{Acrlong}\oarg{options}\marg{label}\oarg{insert}
\end{definition}
\begin{definition}[\DescribeMacro{\ACRlong}]
\cs{ACRlong}\oarg{options}\marg{label}\oarg{insert}
\end{definition}
Again there are also plural versions:
\begin{definition}[\DescribeMacro{\acrlongpl}]
\cs{acrlongpl}\oarg{options}\marg{label}\oarg{insert}
\end{definition}
\begin{definition}[\DescribeMacro{\Acrlongpl}]
\cs{Acrlongpl}\oarg{options}\marg{label}\oarg{insert}
\end{definition}
\begin{definition}[\DescribeMacro{\ACRlongpl}]
\cs{ACRlongpl}\oarg{options}\marg{label}\oarg{insert}
\end{definition}
The long plural form is as supplied by the \gloskey{longplural} key,
which \ics{newacronym} implicitly sets.

The commands below display the full form of the acronym, but note
that this isn't necessarily the same as the form used on \firstuse.
These full-form commands are shortcuts that use the above commands,
rather than creating the \gls{linktext} from the complete full form.
These full-form commands have star and plus variants and optional
arguments that are passed to the above commands.

\begin{definition}[\DescribeMacro{\acrfull}]
\cs{acrfull}\oarg{options}\marg{label}\oarg{insert}
\end{definition}
This is a shortcut for
\begin{definition}[\DescribeMacro{\acrfullfmt}]
\cs{acrfullfmt}\marg{options}\marg{label}\marg{insert}
\end{definition}
which by default does
\begin{alltt}
\cs{acrfullformat}
 \{\cs{acrlong}\oarg{options}\marg{label}\marg{insert}\}
 \{\cs{acrshort}\oarg{options}\marg{label}\}
\end{alltt}
where
\begin{definition}[\DescribeMacro\acrfullformat]
\cs{acrfullformat}\marg{long}\marg{short}
\end{definition}
by default does \meta{long} (\meta{short}).
(For further details of these format commands see
\ifpdf section~\ref*{sec:code:acronym} in \fi the documented code,
\texttt{glossaries-code.pdf}.)

There are also analogous upper case variants:
\begin{definition}[\DescribeMacro{\Acrfull}]
\cs{Acrfull}\oarg{options}\marg{label}\oarg{insert}
\end{definition}
\begin{definition}[\DescribeMacro{\ACRfull}]
\cs{ACRfull}\oarg{options}\marg{label}\oarg{insert}
\end{definition}
and plural versions:
\begin{definition}[\DescribeMacro\acrfullpl]
\cs{acrfullpl}\oarg{options}\marg{label}\oarg{insert}
\end{definition}
\begin{definition}[\DescribeMacro\Acrfullpl]
\cs{Acrfullpl}\oarg{options}\marg{label}\oarg{insert}
\end{definition}
\begin{definition}[\DescribeMacro\ACRfullpl]
\cs{ACRfullpl}\oarg{options}\marg{label}\oarg{insert}
\end{definition}

If you find the above commands too cumbersome to write, you can use
the \pkgopt{shortcuts} package option to activate the shorter
command names listed in \tableref{tab:shortcuts}.

\begin{table}[htbp]
\caption{Synonyms provided by the package option \pkgoptfmt{shortcuts}}
\label{tab:shortcuts}
\vskip\baselineskip
\centering
\begin{tabular}{ll}
\bfseries Shortcut Command & \bfseries Equivalent Command\\
\ics{acs} & \ics{acrshort}\\
\ics{Acs} & \ics{Acrshort}\\
\ics{acsp} & \ics{acrshortpl}\\
\ics{Acsp} & \ics{Acrshortpl}\\
\ics{acl} & \ics{acrlong}\\
\ics{Acl} & \ics{Acrlong}\\
\ics{aclp} & \ics{acrlongpl}\\
\ics{Aclp} & \ics{Acrlongpl}\\
\ics{acf} & \ics{acrfull}\\
\ics{Acf} & \ics{Acrfull}\\
\ics{acfp} & \ics{acrfullpl}\\
\ics{Acfp} & \ics{Acrfullpl}\\
\ics{ac} & \ics{gls}\\
\ics{Ac} & \ics{Gls}\\
\ics{acp} & \ics{glspl}\\
\ics{Acp} & \ics{Glspl}
\end{tabular}
\end{table}

It is also possible to access the long and short forms without
adding information to the glossary using commands analogous to
\ics{glsentrytext} (described in \sectionref{sec:glsnolink}).

\begin{important}
The commands that convert the first letter to upper case come with
the same caveats as those for analogous commands like
\ics{Glsentrytext} (non-expandable, can't be used in PDF bookmarks,
care needs to be taken if the first letter is an accented character
etc). See \sectionref{sec:glsnolink}.
\end{important}

The long form can be accessed using:
\begin{definition}[\DescribeMacro{\glsentrylong}]
\cs{glsentrylong}\marg{label}
\end{definition}
or, with the first letter converted to upper case:
\begin{definition}[\DescribeMacro{\Glsentrylong}]
\cs{Glsentrylong}\marg{label}
\end{definition}
Plural forms:
\begin{definition}[\DescribeMacro{\glsentrylongpl}]
\cs{glsentrylongpl}\marg{label}
\end{definition}
\begin{definition}[\DescribeMacro{\Glsentrylongpl}]
\cs{Glsentrylongpl}\marg{label}
\end{definition}

Similarly, to access the short form:
\begin{definition}[\DescribeMacro{\glsentryshort}]
\cs{glsentryshort}\marg{label}
\end{definition}
or, with the first letter converted to upper case:
\begin{definition}[\DescribeMacro{\Glsentryshort}]
\cs{Glsentryshort}\marg{label}
\end{definition}
Plural forms:
\begin{definition}[\DescribeMacro{\glsentryshortpl}]
\cs{glsentryshortpl}\marg{label}
\end{definition}
\begin{definition}[\DescribeMacro{\Glsentryshortpl}]
\cs{Glsentryshortpl}\marg{label}
\end{definition}

And the full form can be obtained using:
\begin{definition}[\DescribeMacro{\glsentryfull}]
\cs{glsentryfull}\marg{label}
\end{definition}
\begin{definition}[\DescribeMacro{\Glsentryfull}]
\cs{Glsentryfull}\marg{label}
\end{definition}
\begin{definition}[\DescribeMacro{\glsentryfullpl}]
\cs{glsentryfullpl}\marg{label}
\end{definition}
\begin{definition}[\DescribeMacro{\Glsentryfullpl}]
\cs{Glsentryfullpl}\marg{label}
\end{definition}
These again use \ics{acrfullformat} by default, but
the format may be redefined by the acronym style.

\section{Changing the Acronym Style}
\label{sec:setacronymstyle}

It may be that the default style doesn't suit your requirements in
which case you can switch to another styles using
\begin{definition}[\DescribeMacro{\setacronymstyle}]
\cs{setacronymstyle}\marg{style name}
\end{definition}
where \meta{style name} is the name of the required style.

\begin{important}
You must use \cs{setacronymstyle} \emph{before} you define the
acronyms with \cs{newacronym}. If you have multiple glossaries
representing lists of acronyms, you must use \cs{setacronymstyle}
\emph{after} using \cs{DeclareAcronymList}.
\end{important}

Note that unlike the default behaviour of \cs{newacronym}, the
styles used via \cs{setacronymstyle} don't use the \gloskey{first}
or \gloskey{text} keys, but instead they use \ics{defglsentryfmt} to
set a~custom format that uses the \gloskey{long} and \gloskey{short}
keys (or their plural equivalents). This means that these styles
cope better with plurals that aren't formed by simply appending the
singular form with the letter \qt{s}. In fact, most of the predefined
styles use \ics{glsgenacfmt} and modify the definitions of commands
like \ics{genacrfullformat}.

Note that when you use \cs{setacronymstyle} the \gloskey{name} key
is set to
\begin{definition}[\DescribeMacro{\acronymentry}]
\cs{acronymentry}\marg{label}
\end{definition}
and the \gloskey{sort} key is set to
\begin{definition}[\DescribeMacro{\acronymsort}]
\cs{acronymsort}\marg{short}\marg{long}
\end{definition}
These commands are redefined by the acronym styles. However, you can
redefine them again after the style has been set but before you use
\cs{newacronym}. Protected expansion is performed on \cs{acronymsort}
when the entry is defined.

\subsection{Predefined Acronym Styles}
\label{sec:predefinedacrstyles}

The \styfmt{glossaries} package provides a~number of predefined
styles. These styles apply
\begin{definition}[\DescribeMacro{\firstacronymfont}]
\cs{firstacronymfont}\marg{text}
\end{definition}
to the short form on first use and
\begin{definition}[\DescribeMacro{\acronymfont}]
\cs{acronymfont}\marg{text}
\end{definition}
on subsequent use. The styles modify the definition of
\cs{acronymfont} as required, but \ics{firstacronymfont} is only set
once by the package when it's loaded. By default
\ics{firstacronymfont}\marg{text} is the same as
\ics{acronymfont}\marg{text}.
If you want the short form displayed differently on first use, you
can redefine \ics{firstacronymfont} independently of the acronym
style.

The predefined styles that contain \texttt{sc} in their name (for
example \acrstyle{long-sc-short}) redefine \cs{acronymfont} to use
\ics{textsc}, which means that the short form needs to be specified
in lower case. \ifpdf Remember that \verb|\textsc{abc}| produces
\textsc{abc} but \verb|\textsc{ABC}| produces \textsc{ABC}.\fi

\hypertarget{boldsc}{}%
\begin{important}%
Some fonts don't support bold smallcaps, so you may need to redefine 
\ics{glsnamefont} (see \sectionref{sec:printglossary}) to switch to
medium weight if you are using a glossary style that displays entry
names in bold and you have chosen an acronym style that uses
\ics{textsc}.
\end{important}

The predefined styles that contain \texttt{sm} in their name
(for example \acrstyle{long-sm-short}) redefine \cs{acronymfont} to
use \ics{textsmaller}.  

\hypertarget{smaller}{}%
\begin{important}
Note that the \styfmt{glossaries} package doesn't define or load any package that
defines \ics{textsmaller}. If you use one of the acronym styles that
set \ics{acronymfont} to \cs{textsmaller} you must
explicitly load the \sty{relsize} package or otherwise define
\cs{textsmaller}.
\end{important}

The remaining predefined styles redefine \cs{acronymfont}\marg{text}
to simply do its argument \meta{text}. 

\begin{important}
In most cases, the predefined styles adjust \ics{acrfull} and
\ics{glsentryfull} (and their plural and upper case variants) to
reflect the style. The only exceptions to this are the 
\acrstyle{dua} and \acrstyle{footnote} styles (and their variants).
\end{important}

The following styles are supplied by the \styfmt{glossaries}
package:
\begin{itemize}
\item \acrstyle{long-short}, \acrstyle{long-sc-short},
\acrstyle{long-sm-short}:

With these three styles, acronyms are displayed in the form
\begin{definition}
\meta{long} (\ics{firstacronymfont}\marg{short})
\end{definition}
on first use and
\begin{definition}
\ics{acronymfont}\marg{short}
\end{definition}
on subsequent use.
They also set \cs{acronymsort}\marg{short}\marg{long} to just
\meta{short}.  This means that the acronyms are sorted according to
their short form.  In addition, \cs{acronymentry}\marg{label} is set
to just the short form (enclosed in \cs{acronymfont}) and the
\gloskey{description} key is set to the long form.

\item \acrstyle{short-long}, \acrstyle{sc-short-long},
\acrstyle{sm-short-long}:

These three styles are analogous to the above three styles, except
the display order is swapped to
\begin{definition}
\ics{firstacronymfont}\marg{short} (\meta{long})
\end{definition}
on first use.

Note, however, that \ics{acronymsort} and \ics{acronymentry} are the
same as for the \meta{long} (\meta{short}) styles above, so the
acronyms are still sorted according to the short form.

\item \acrstyle{long-short-desc}, \acrstyle{long-sc-short-desc},
\acrstyle{long-sm-short-desc}:

These are like the \acrstyle{long-short}, \acrstyle{long-sc-short}
and \acrstyle{long-sm-short} styles described above, except that the
\gloskey{description} key must be supplied in the optional argument
of \ics{newacronym}. They also redefine \ics{acronymentry} to
\marg{long} (\cs{acronymfont}\marg{short}) and redefine
\ics{acronymsort}\marg{short}\marg{long} to just \meta{long}.
This means that the acronyms are sorted according to the long form,
and in the list of acronyms the name field has the long form
followed by the short form in parentheses. I~recommend you use
a~glossary style such as \glostyle{altlist} with these acronym
styles to allow for the long name field.

\item \acrstyle{short-long-desc}, \acrstyle{sc-short-long-desc},
\acrstyle{sm-short-long-desc}:

These styles are analogous to the above three styles, but the first
use display style is:
\begin{definition}
\cs{firstacronymfont}\marg{short} (\meta{long})
\end{definition}
The definitions of \cs{acronymsort} and \cs{acronymentry} are the
same as those for \acrstyle{long-short-desc} etc.

\item \acrstyle{dua}, \acrstyle{dua-desc}:

With these styles, the \glslike\ commands always display the long form 
regardless of whether the entry has been used or not. However, \ics{acrfull}
and \ics{glsentryfull} will display \meta{long} (\cs{acronymfont}\marg{short}). In the case
of \acrstyle{dua}, the \gloskey{name} and \gloskey{sort} keys are set to
the short form and the description is set to the long form. In the
case of \acrstyle{dua-desc}, the \gloskey{name} and \gloskey{sort}
keys are set to the long form and the description is supplied in the
optional argument of \ics{newacronym}.

\item \acrstyle{footnote}, \acrstyle{footnote-sc},
\acrstyle{footnote-sm}:

With these three styles, on first use the \glslike\ commands display:
\begin{definition}
\cs{firstacronymfont}\marg{short}\ics{footnote}\marg{long}
\end{definition}
However, \ics{acrfull} and \ics{glsentryfull} are set to 
\cs{acronymfont}\marg{short} (\meta{long}). On subsequent use the display is:
\begin{definition}
\cs{acronymfont}\marg{short}
\end{definition}
The \gloskey{sort} and \gloskey{name} keys are set to
the short form, and the \gloskey{description} is set to the long
form.

\begin{important}
In order to avoid nested hyperlinks on \firstuse\ the footnote
styles automatically implement \pkgopt[false]{hyperfirst} for the
acronym lists.
\end{important}

\item \acrstyle{footnote-desc}, \acrstyle{footnote-sc-desc},
\acrstyle{footnote-sm-desc}:

These three styles are similar to the previous three styles, but the
description has to be supplied in the optional argument of
\ics{newacronym}. The \gloskey{name} key is set to the long form
followed by the short form in parentheses and the \gloskey{sort} key
is set to the long form. This means that the acronyms will be sorted
according to the long form. In addition, since the \gloskey{name}
will typically be quite wide it's best to choose a glossary style
that can accommodate this, such as \glostyle{altlist}.

\end{itemize}

\begin{example}{Adapting a Predefined Acronym Style}{ex:acrstyle}
Suppose I~want to use the \acrstyle{footnote-sc-desc} style, but 
I~want the \gloskey{name} key set to the short form followed by the
long form in parentheses and the \gloskey{sort} key set to the short
form. Then I need to specify the \acrstyle{footnote-sc-desc} style:
\begin{verbatim}
\setacronymstyle{footnote-sc-desc}
\end{verbatim}
and then redefine \ics{acronymsort} and \ics{acronymentry}:
\begin{verbatim}
\renewcommand*{\acronymsort}[2]{#1}% sort by short form
\renewcommand*{\acronymentry}[1]{%
  \acronymfont{\glsentryshort{#1}}\space (\glsentrylong{#1})}%
\end{verbatim}
(I've used \cs{space} for extra clarity, but you can just use an
actual space instead.)

Since the default Computer Modern fonts don't support bold
smallcaps, I'm also going to redefine \ics{acronymfont} so that it
always switches to medium weight to ensure the smallcaps setting is
used:
\begin{verbatim}
\renewcommand*{\acronymfont}[1]{\textmd{\scshape #1}}
\end{verbatim}
This isn't necessary if you use a font that supports bold smallcaps.

The sample file \samplefile{FnAcrDesc} illustrates this
example.
\end{example}

\subsection{Defining A Custom Acronym Style}
\label{sec:customacronym}

You may find that the predefined acronyms styles that come with the
\styfmt{glossaries} package don't suit your requirements. In this
case you can define your own style using:
\begin{definition}[\DescribeMacro{\newacronymstyle}]
\cs{newacronymstyle}\marg{style name}\marg{display}\marg{definitions}
\end{definition}
where \meta{style name} is the name of the new style (avoid active
characters). The second argument, \meta{display}, is equivalent to
the mandatory argument of \ics{defglsentryfmt}. You can simply use
\ics{glsgenacfmt} or you can customize the display using commands
like \ics{ifglsused}, \ics{glsifplural} and \ics{glscapscase}.
(See \sectionref{sec:glsdisplay} for further details.)
If the style is likely to be used with a mixed glossary (that is entries in
that glossary are defined both with \ics{newacronym} and
\ics{newglossaryentry}) then you can test if the entry is an acronym
and use \ics{glsgenacfmt} if it is or \ics{glsgenentryfmt} if it
isn't. For example, the \acrstyle{long-short} style sets
\meta{display} as
\begin{verbatim}
\ifglshaslong{\glslabel}{\glsgenacfmt}{\glsgenentryfmt}%
\end{verbatim}
(You can use \ics{ifglshasshort} instead of \ics{ifglshaslong} to
test if the entry is an acronym if you prefer.)

The third argument, \meta{definitions}, can be used to redefine the
commands that affect the display style, such as \ics{acronymfont}
or, if \meta{display} uses \cs{glsgenacfmt}, \ics{genacrfullformat}
and its variants.

Note that \ics{setacronymstyle} redefines \ics{glsentryfull} and
\ics{acrfullfmt} to use \ics{genacrfullformat} (and similarly for
the plural and upper case variants). If this isn't appropriate for the
style (as in the case of styles like \acrstyle{footnote} and
\acrstyle{dua}) \cs{newacronymstyle} should redefine these commands
within \meta{definitions}.


Within \cs{newacronymstyle}'s \meta{definitions} argument you 
can also redefine
\begin{definition}[\DescribeMacro{\GenericAcronymFields}]
\cs{GenericAcronymFields}
\end{definition}
This is a list of additional fields to be set in \ics{newacronym}.
You can use the following token registers to access the entry label,
long form and short form: \DescribeMacro{\glslabeltok}\cs{glslabeltok}, 
\DescribeMacro{\glslongtok}\cs{glslongtok} and 
\DescribeMacro{\glsshorttok}\cs{glsshorttok}. As with all \TeX\
registers, you can access their values by preceding the register
with \ics{the}.  For example, the \acrstyle{long-short} style does:
\begin{verbatim}
\renewcommand*{\GenericAcronymFields}{%
   description={\the\glslongtok}}%
\end{verbatim}
which sets the \gloskey{description} field to the long form of the
acronym whereas the \acrstyle{long-short-desc} style does:
\begin{verbatim}
\renewcommand*{\GenericAcronymFields}{}%
\end{verbatim}
since the description needs to be specified by the user.

It may be that you want to define a new acronym style that's based
on an existing style. Within \meta{display} you can use
\par
\DescribeMacro{\GlsUseAcrEntryDispStyle}
\begin{definition}
\cs{GlsUseAcrEntryDispStyle}\marg{style name}
\end{definition}
to use the \meta{display} definition from the style given by
\meta{style name}. Within \meta{definitions} you can use
\begin{definition}[\DescribeMacro{\GlsUseAcrStyleDefs}]
\cs{GlsUseAcrStyleDefs}\marg{style name}
\end{definition}
to use the \meta{definitions} from the style given by \meta{style
name}. For example, the \acrstyle{long-sc-short} acronym style is
based on the \acrstyle{long-short} style with minor
modifications (remember to use \verb|##| instead of \verb|#| within
\meta{definitions}):
\begin{verbatim}
\newacronymstyle{long-sc-short}%
{% use the same display as "long-short"
  \GlsUseAcrEntryDispStyle{long-short}%
}%
{% use the same definitions as "long-short"
  \GlsUseAcrStyleDefs{long-short}%
  % Minor modifications:
  \renewcommand{\acronymfont}[1]{\textsc{##1}}%
  \renewcommand*{\acrpluralsuffix}{\glstextup{\glspluralsuffix}}%
}
\end{verbatim}
(\DescribeMacro{\glstextup}\cs{glstextup} is used to cancel the effect 
of \ics{textsc}. This defaults to \ics{textulc}, if defined,
otherwise \ics{textup}. For example, the plural of \textsc{svm}
should be rendered as \textsc{svm}s rather than \textsc{svms}.)

\begin{example}{Defining a Custom Acronym Style}{ex:customacrstyle}
Suppose I want my acronym on \firstuse\ to have the
short form in the text and the long form with the description in a
footnote. Suppose also that I want the short form to be put in small
caps in the main body of the document, but I want it in normal
capitals in the list of acronyms. In my list of acronyms, I want the
long form as the name with the short form in brackets followed by
the description. That is, in the text I want \ics{gls} on \gls{firstuse}
to display:
\begin{display}
\ics{textsc}\marg{abbrv}\cs{footnote}"{"\meta{long}: \meta{description}"}"
\end{display}
on subsequent use:
\begin{display}
\ics{textsc}\marg{abbrv}
\end{display}
and in the list of acronyms, each entry will be displayed in the
form:
\begin{display}
\meta{long} (\meta{short}) \meta{description}
\end{display}

Let's suppose it's possible that I may have a mixed glossary. I can
check this in the second argument of \ics{newacronymstyle} using:
\begin{verbatim}
\ifglshaslong{\glslabel}{\glsgenacfmt}{\glsgenentryfmt}%
\end{verbatim}
This will use \ics{glsgenentryfmt} if the entry isn't an acronym,
otherwise it will use \ics{glsgenacfmt}. The third argument
(\meta{definitions}) of
\ics{newacronymstyle} needs to redefine \ics{genacrfullformat} etc 
so that the \firstuse\ displays the short form in the text with the 
long form in a footnote followed by the description. This is done as
follows (remember to use \verb|##| instead of \verb|#|):
\begin{verbatim}
 % No case change, singular first use:
  \renewcommand*{\genacrfullformat}[2]{%
   \firstacronymfont{\glsentryshort{##1}}##2%
   \footnote{\glsentrylong{##1}: \glsentrydesc{##1}}%
  }%
 % First letter upper case, singular first use: 
  \renewcommand*{\Genacrfullformat}[2]{%
   \firstacronymfont{\Glsentryshort{##1}}##2%
   \footnote{\glsentrylong{##1}: \glsentrydesc{##1}}%
  }%
 % No case change, plural first use:
  \renewcommand*{\genplacrfullformat}[2]{%
   \firstacronymfont{\glsentryshortpl{##1}}##2%
   \footnote{\glsentrylongpl{##1}: \glsentrydesc{##1}}%
  }%
 % First letter upper case, plural first use:
  \renewcommand*{\Genplacrfullformat}[2]{%
   \firstacronymfont{\Glsentryshortpl{##1}}##2%
   \footnote{\glsentrylongpl{##1}: \glsentrydesc{##1}}%
  }%
\end{verbatim}
If you think it inappropriate for the short form to be capitalised
at the start of a sentence you can change the above to:
\begin{verbatim}
 % No case change, singular first use:
  \renewcommand*{\genacrfullformat}[2]{%
   \firstacronymfont{\glsentryshort{##1}}##2%
   \footnote{\glsentrylong{##1}: \glsentrydesc{##1}}%
  }%
 % No case change, plural first use:
  \renewcommand*{\genplacrfullformat}[2]{%
   \firstacronymfont{\glsentryshortpl{##1}}##2%
   \footnote{\glsentrylongpl{##1}: \glsentrydesc{##1}}%
  }%
 \let\Genacrfullformat\genacrfullformat
 \let\Genplacrfullformat\genplacrfullformat
\end{verbatim}
Another variation is to use \ics{Glsentrylong} and
\ics{Glsentrylongpl} in the footnote instead of \ics{glsentrylong} and
\ics{glsentrylongpl}.

Now let's suppose that commands such as \ics{glsentryfull} and
\ics{acrfull} shouldn't
use a~footnote, but instead use the format: \meta{long}
(\meta{short}).
This means that the style needs to redefine \cs{glsentryfull},
\ics{acrfullfmt} and their plural and upper case variants.

First, the non-linking commands: 
\begin{verbatim}
  \renewcommand*{\glsentryfull}[1]{%
    \glsentrylong{##1}\space
      (\acronymfont{\glsentryshort{##1}})%
  }%
  \renewcommand*{\Glsentryfull}[1]{%
    \Glsentrylong{##1}\space
      (\acronymfont{\glsentryshort{##1}})%
  }%
  \renewcommand*{\glsentryfullpl}[1]{%
    \glsentrylongpl{##1}\space
      (\acronymfont{\glsentryshortpl{##1}})%
  }%
  \renewcommand*{\Glsentryfullpl}[1]{%
    \Glsentrylongpl{##1}\space
      (\acronymfont{\glsentryshortpl{##1}})%
  }%
\end{verbatim}
Now for the linking commands:
\begin{verbatim}
  \renewcommand*{\acrfullfmt}[3]{%
    \glslink[##1]{##2}{%
     \glsentrylong{##2}##3\space
      (\acronymfont{\glsentryshort{##2}})%
    }%
  }%
  \renewcommand*{\Acrfullfmt}[3]{%
    \glslink[##1]{##2}{%
     \Glsentrylong{##2}##3\space
      (\acronymfont{\glsentryshort{##2}})%
    }%
  }%
  \renewcommand*{\ACRfullfmt}[3]{%
    \glslink[##1]{##2}{%
     \MakeTextUppercase{%
       \glsentrylong{##2}##3\space
         (\acronymfont{\glsentryshort{##2}})%
     }%
    }%
  }%
  \renewcommand*{\acrfullplfmt}[3]{%
    \glslink[##1]{##2}{%
     \glsentrylongpl{##2}##3\space
       (\acronymfont{\glsentryshortpl{##2}})%
    }%
  }%
  \renewcommand*{\Acrfullplfmt}[3]{%
    \glslink[##1]{##2}{%
     \Glsentrylongpl{##2}##3\space
       (\acronymfont{\glsentryshortpl{##2}})%
    }%
  }%
  \renewcommand*{\ACRfullplfmt}[3]{%
    \glslink[##1]{##2}{%
     \MakeTextUppercase{%
       \glsentrylongpl{##2}##3\space
         (\acronymfont{\glsentryshortpl{##2}})%
     }%
    }%
  }%
\end{verbatim}
(This may cause problems with long hyperlinks, in which case adjust
the definitions so that, for example, only the short form is inside
the argument of \ics{glslink}.)

The style also needs to redefine \ics{acronymsort} so that the
acronyms are sorted according to the long form:
\begin{verbatim}
  \renewcommand*{\acronymsort}[2]{##2}%
\end{verbatim}
If you prefer them to be sorted according to the short form you can
change the above to:
\begin{verbatim}
  \renewcommand*{\acronymsort}[2]{##1}%
\end{verbatim}
The acronym font needs to be set to \ics{textsc} and the plural
suffix adjusted so that the \qt{s} suffix in the plural short form
doesn't get converted to smallcaps:
\begin{verbatim}
  \renewcommand*{\acronymfont}[1]{\textsc{##1}}%
  \renewcommand*{\acrpluralsuffix}{\glstextup{\glspluralsuffix}}%
\end{verbatim}
There are a number of ways of dealing with the format in the list of
acronyms. The simplest way is to redefine \ics{acronymentry} to the
long form followed by the upper case short form in parentheses:
\begin{verbatim}
 \renewcommand*{\acronymentry}[1]{%
   \Glsentrylong{##1}\space
     (\MakeTextUppercase{\glsentryshort{##1}})}%
\end{verbatim}
(I've used \ics{Glsentrylong} instead of \ics{glsentrylong} to
capitalise the name in the glossary.)

An alternative approach is to set \ics{acronymentry} to just the
long form and redefine \ics{GenericAcronymFields} to set the
\gloskey{symbol} key to the short form and use a glossary style that
displays the symbol in parentheses after the \gloskey{name} (such as
the \glostyle{tree} style) like this:
\begin{verbatim}
 \renewcommand*{\acronymentry}[1]{\Glsentrylong{##1}}%
 \renewcommand*{\GenericAcronymFields}{%
    symbol={\protect\MakeTextUppercase{\the\glsshorttok}}}%
\end{verbatim}
I'm going to use the first approach and set
\ics{GenericAcronymFields} to do nothing:
\begin{verbatim}
 \renewcommand*{\GenericAcronymFields}{}%
\end{verbatim}

Finally, this style needs to switch off hyperlinks on first use to
avoid nested links:
\begin{verbatim}
 \glshyperfirstfalse
\end{verbatim}
Putting this all together:
\begin{verbatim}
\newacronymstyle{custom-fn}% new style name
{%
  \ifglshaslong{\glslabel}{\glsgenacfmt}{\glsgenentryfmt}%
}%
{%
 \renewcommand*{\GenericAcronymFields}{}%
 \glshyperfirstfalse
  \renewcommand*{\genacrfullformat}[2]{%
   \firstacronymfont{\glsentryshort{##1}}##2%
   \footnote{\glsentrylong{##1}: \glsentrydesc{##1}}%
  }%
  \renewcommand*{\Genacrfullformat}[2]{%
   \firstacronymfont{\Glsentryshort{##1}}##2%
   \footnote{\glsentrylong{##1}: \glsentrydesc{##1}}%
  }%
  \renewcommand*{\genplacrfullformat}[2]{%
   \firstacronymfont{\glsentryshortpl{##1}}##2%
   \footnote{\glsentrylongpl{##1}: \glsentrydesc{##1}}%
  }%
  \renewcommand*{\Genplacrfullformat}[2]{%
   \firstacronymfont{\Glsentryshortpl{##1}}##2%
   \footnote{\glsentrylongpl{##1}: \glsentrydesc{##1}}%
  }%
  \renewcommand*{\glsentryfull}[1]{%
    \glsentrylong{##1}\space
      (\acronymfont{\glsentryshort{##1}})%
  }%
  \renewcommand*{\Glsentryfull}[1]{%
    \Glsentrylong{##1}\space
      (\acronymfont{\glsentryshort{##1}})%
  }%
  \renewcommand*{\glsentryfullpl}[1]{%
    \glsentrylongpl{##1}\space
      (\acronymfont{\glsentryshortpl{##1}})%
  }%
  \renewcommand*{\Glsentryfullpl}[1]{%
    \Glsentrylongpl{##1}\space
      (\acronymfont{\glsentryshortpl{##1}})%
  }%
  \renewcommand*{\acrfullfmt}[3]{%
    \glslink[##1]{##2}{%
     \glsentrylong{##2}##3\space
      (\acronymfont{\glsentryshort{##2}})%
    }%
  }%
  \renewcommand*{\Acrfullfmt}[3]{%
    \glslink[##1]{##2}{%
     \Glsentrylong{##2}##3\space
      (\acronymfont{\glsentryshort{##2}})%
    }%
  }%
  \renewcommand*{\ACRfullfmt}[3]{%
    \glslink[##1]{##2}{%
     \MakeTextUppercase{%
       \glsentrylong{##2}##3\space
         (\acronymfont{\glsentryshort{##2}})%
     }%
    }%
  }%
  \renewcommand*{\acrfullplfmt}[3]{%
    \glslink[##1]{##2}{%
     \glsentrylongpl{##2}##3\space
       (\acronymfont{\glsentryshortpl{##2}})%
    }%
  }%
  \renewcommand*{\Acrfullplfmt}[3]{%
    \glslink[##1]{##2}{%
     \Glsentrylongpl{##2}##3\space
       (\acronymfont{\glsentryshortpl{##2}})%
    }%
  }%
  \renewcommand*{\ACRfullplfmt}[3]{%
    \glslink[##1]{##2}{%
     \MakeTextUppercase{%
       \glsentrylongpl{##2}##3\space
         (\acronymfont{\glsentryshortpl{##2}})%
     }%
    }%
  }%
  \renewcommand*{\acronymfont}[1]{\textsc{##1}}%
  \renewcommand*{\acrpluralsuffix}{\glstextup{\glspluralsuffix}}%
  \renewcommand*{\acronymsort}[2]{##2}%
  \renewcommand*{\acronymentry}[1]{%
   \Glsentrylong{##1}\space
     (\MakeTextUppercase{\glsentryshort{##1}})}%
}
\end{verbatim}

Now I need to specify that I want to use this new style:
\begin{verbatim}
\setacronymstyle{custom-fn}
\end{verbatim}
I also need to use a glossary style that suits this acronym style,
for example \glostyle{altlist}:
\begin{verbatim}
\setglossarystyle{altlist}
\end{verbatim}

Once the acronym style has been set, I can define my acronyms:
\begin{verbatim}
\newacronym[description={set of tags for use in 
developing hypertext documents}]{html}{html}{Hyper 
Text Markup Language}

\newacronym[description={language used to describe the 
layout of a document written in a markup language}]{css}
{css}{Cascading Style Sheet}
\end{verbatim}

The sample file \samplefile{-custom-acronym} illustrates this
example.
\end{example}

\section{Displaying the List of Acronyms}
\label{sec:disploa}

The list of acronyms is just like any other type of glossary and can
be displayed on its own using:
\begin{description}
\itempar{\opt1:}
\begin{alltt}
\ics{printnoidxglossary}[type=\ics{acronymtype}]
\end{alltt}

\itempar{\optsand23:}
\begin{alltt}
\ics{printglossary}[type=\ics{acronymtype}]
\end{alltt}

(If you use the
\pkgopt{acronym} package option you can also use 
\begin{alltt}
\ics{printacronyms}\oarg{options}
\end{alltt}
as a synonym for
\begin{alltt}
\ics{printglossary}[type=\ics{acronymtype},\meta{options}]
\end{alltt}
See \sectionref{sec:pkgopts-acronym}.)

\end{description}
Alternatively the list of acronyms can be displayed with all the other 
glossaries using:
\begin{description}
\item[\opt1:] \ics{printnoidxglossaries}

\item[\optsand23:] \ics{printglossaries}
\end{description}

However, care must be taken to choose a glossary style that's
appropriate to your acronym style.
Alternatively, you can define your own custom style (see 
\sectionref{sec:newglossarystyle} for further details).

\section{Upgrading From the glossary Package}
\label{sec:oldacronym}

Users of the obsolete \sty{glossary} package may recall that
the syntax used to define new acronyms has changed with the
replacement \styfmt{glossaries} package. In addition, the old
\sty{glossary} package created the command 
\cs{}\meta{acr-name} when defining the acronym \meta{acr-name}.

In order to facilitate migrating from the old package to the new
one, the \styfmt{glossaries} package\footnote{as from version 1.18} 
provides the command:
\begin{definition}[\DescribeMacro{\oldacronym}]
\cs{oldacronym}\oarg{label}\marg{abbrv}\marg{long}\marg{key-val list}
\end{definition}
This uses the same syntax as the \sty{glossary} package's
method of defining acronyms. It is equivalent to:
\begin{display}
\ics{newacronym}\oarg{key-val list}\marg{label}\marg{abbrv}\marg{long}
\end{display}
In addition, \ics{oldacronym} also defines the commands
\cs{}\meta{label}, which is equivalent to \ics{gls}\marg{label},
and \cs{}\meta{label}\texttt{*}, which is equivalent to
\ics{Gls}\marg{label}. If \meta{label} is omitted, \meta{abbrv}
is used. Since commands names must consist only of alphabetical
characters, \meta{label} must also only consist of alphabetical
characters. Note that \cs{}\meta{label} doesn't allow you to use
the first optional argument of \ics{gls} or \ics{Gls} --- you will
need to explicitly use \ics{gls} or \ics{Gls} to change the
settings.

\begin{important}
Recall that, in general, \LaTeX\ ignores spaces following command 
names consisting of alphabetical characters. This is also true for 
\cs{}\meta{label} unless you additionally load the
\sty{xspace} package, but be aware that there are some issues with
using \sty{xspace}.\footnotemark
\end{important}
\footnotetext{See David Carlisle's explanation in
\url{http://tex.stackexchange.com/questions/86565/drawbacks-of-xspace}}

The \styfmt{glossaries} package doesn't load the \sty{xspace} package
since there are both advantages and disadvantages to using
\ics{xspace} in \cs{}\meta{label}. If you don't use the 
\sty{xspace} package you need to explicitly force a space using
\verb*|\ | (backslash space) however you can follow \cs{}\meta{label}
with additional text in square brackets (the final optional
argument to \ics{gls}). If you use the \sty{xspace} package
you don't need to escape the spaces but you can't use
the optional argument to insert text (you will have to explicitly
use \ics{gls}).

To illustrate this, suppose I define the acronym \qt{abc} as
follows:
\begin{verbatim}
\oldacronym{abc}{example acronym}{}
\end{verbatim}
This will create the command \cs{abc} and its starred version
\cs{abc*}. \Tableref{tab:xspace} illustrates the effect of
\cs{abc} (on subsequent use) according to whether or not the
\sty{xspace} package has been loaded. As can be seen from the
final row in the table, the \sty{xspace} package prevents the
optional argument from being recognised.

\begin{table}[htbp]
\caption[The effect of using xspace]{The effect of using 
\styfmt{xspace} with \cs{oldacronym}}
\label{tab:xspace}
\vskip\baselineskip
\centering
\begin{tabular}{lll}
\bfseries Code & \bfseries With \styfmt{xspace} &
\bfseries Without \styfmt{xspace}\\
\verb|\abc.| & abc. & abc.\\
\verb|\abc xyz| & abc xyz & abcxyz\\
\verb|\abc\ xyz| & abc xyz & abc xyz\\
\verb|\abc* xyz| & Abc xyz & Abc xyz\\
\verb|\abc['s] xyz| & abc ['s] xyz & abc's xyz
\end{tabular}
\par
\end{table}

\chapter{Unsetting and Resetting Entry Flags}
\label{sec:glsunset}

When using the \glslike\ commands it is
possible that you may want to use the value given by the
\gloskey{first} key, even though you have already \glslink{firstuse}{used} the glossary
entry. Conversely, you may want to use the value given by the
\gloskey{text} key, even though you haven't used the glossary entry.
The former can be achieved by one of the following commands:
\begin{definition}[\DescribeMacro{\glsreset}]
\cs{glsreset}\marg{label}
\end{definition}
\begin{definition}[\DescribeMacro{\glslocalreset}]
\cs{glslocalreset}\marg{label}
\end{definition}
while the latter can be achieved by one of the following commands:
\begin{definition}[\DescribeMacro{\glsunset}]
\cs{glsunset}\marg{label}
\end{definition}
\begin{definition}[\DescribeMacro{\glslocalunset}]
\cs{glslocalunset}\marg{label}
\end{definition}
You can also reset or unset all entries for a given glossary or list of
glossaries using:
\begin{definition}[\DescribeMacro{\glsresetall}]
\cs{glsresetall}\oarg{glossary list}
\end{definition}
\begin{definition}[\DescribeMacro{\glslocalresetall}]
\cs{glslocalresetall}\oarg{glossary list}
\end{definition}
\begin{definition}[\DescribeMacro{\glsunsetall}]
\cs{glsunsetall}\oarg{glossary list}
\end{definition}
\begin{definition}[\DescribeMacro{\glslocalunsetall}]
\cs{glslocalunsetall}\oarg{glossary list}
\end{definition}
where \meta{glossary list} is a comma-separated list of 
glossary labels. If omitted, all defined glossaries are assumed
(except for the ignored ones).
For example, to reset all entries in the main glossary and the
list of acronyms:
\begin{verbatim}
\glsresetall[main,acronym]
\end{verbatim}

You can determine whether an entry's \gls{firstuseflag} is set using:
\begin{definition}[\DescribeMacro{\ifglsused}]
\cs{ifglsused}\marg{label}\marg{true part}\marg{false part}
\end{definition}
where \meta{label} is the label of the required entry. If the
entry has been used, \meta{true part} will be done, otherwise
\meta{false part} will be done.

\begin{important}
Be careful when using \glslike\ commands within an
environment or command argument that gets processed multiple times
as it can cause unwanted side-effects when the first use displayed
text is different from subsequent use.
\end{important}

For example, the \env{frame} environment in \cls{beamer} processes
its argument for each overlay. This means that the \firstuseflag\
will be unset on the first overlay and subsequent overlays will use
the non-first use form.

Consider the following example:
\begin{verbatim}
\documentclass{beamer}

\usepackage{glossaries}

\newacronym{svm}{SVM}{support vector machine}

\begin{document}

\begin{frame}
 \frametitle{Frame 1}

 \begin{itemize}
  \item<+-> \gls{svm}
  \item<+-> Stuff.
 \end{itemize}
\end{frame}

\end{document}
\end{verbatim}

On the first overlay, \verb|\gls{svm}| produces \qt{support vector
machine (SVM)} and then unsets the \firstuseflag. When the second
overlay is processed, \verb|\gls{svm}| now produces \qt{SVM}, which
is unlikely to be the desired effect. I~don't know anyway around
this and I can only offer two suggestions.

Firstly, unset all acronyms at the start of the document and
explicitly use \ics{acrfull} when you want the full version to be
displayed:
\begin{verbatim}
\documentclass{beamer}

\usepackage{glossaries}

\newacronym{svm}{SVM}{support vector machine}

\glsunsetall

\begin{document}
\begin{frame}
 \frametitle{Frame 1}

 \begin{itemize}
  \item<+-> \acrfull{svm}
  \item<+-> Stuff.
 \end{itemize}
\end{frame}
\end{document}
\end{verbatim}

Secondly, explicitly reset each acronym on first use:
\begin{verbatim}
\begin{frame}
 \frametitle{Frame 1}

 \begin{itemize}
  \item<+-> \glsreset{svm}\gls{svm}
  \item<+-> Stuff.
 \end{itemize}
\end{frame}
\end{verbatim}
These are non-optimal, but the \cls{beamer} class is too complex for
me to provide a programmatic solution. Other potentially problematic
environments are some \env{tabular}-like environments (but not
\env{tabular} itself) that process the contents in order to work out
the column widths and then reprocess the contents to do the actual
typesetting.

The \sty{amsmath} environments, such as \env{align}, also process
their contents multiple times, but the \styfmt{glossaries} package now
checks for this.

\chapter{Glossary Styles}
\label{sec:styles}

Glossaries vary from lists that simply contain a symbol with a terse
description to lists of terms or phrases with lengthy descriptions.
Some glossaries may have terms with associated symbols. Some may
have hierarchical entries.  There is therefore no single style that
fits every type of glossary. The \styfmt{glossaries} package comes
with a number of pre-defined glossary styles, described in
\sectionref{sec:predefinedstyles}. You can choose one of these that
best suits your type of glossary or, if none of them suit your
document, you can defined your own style (see
\sectionref{sec:newglossarystyle}).

The glossary style can be set using the \gloskey[printglossary]{style} key in the optional
argument to \ics{printnoidxglossary} (\opt1) or \ics{printglossary}
(\optsand23) or using the command:
\begin{definition}[\DescribeMacro{\setglossarystyle}]
\cs{setglossarystyle}\marg{style-name}
\end{definition}
(before the glossary is displayed).

Some of the predefined glossary styles may also be set using the \pkgopt{style} 
package option, it depends if the package in which they are defined
is automatically loaded by the \styfmt{glossaries} package.

You can use the lorum ipsum dummy entries provided in the
\texttt{example-glossaries-*.tex} files (described in
\sectionref{sec:lipsum}) to test the different styles.

\section{Predefined Styles}
\label{sec:predefinedstyles}

The predefined styles can
accommodate numbered level~0 (main) and level~1 entries. See the
package options \pkgopt{entrycounter}, \pkgopt{counterwithin} and
\pkgopt{subentrycounter} described in
\sectionref{sec:pkgopts-printglos}. There is a summary
of available styles in \tableref{tab:styles}. 

\begin{table}[htbp]
\caption[Glossary Styles]{Glossary Styles. An asterisk in the style 
name indicates anything that matches that doesn't match any
previously listed style (e.g.\ \texttt{long3col*}
matches \glostyle{long3col}, \glostyle{long3colheader}, 
\glostyle{long3colborder} and \glostyle{long3colheaderborder}).
A maximum level of 0 indicates a flat glossary (sub-entries
are displayed in the same way as main entries). Where the maximum
level is given as --- there is no limit, but note that 
\gls{makeindex} (\opt2) imposes a limit of 2 sub-levels. If the
homograph column is checked, then the name is not displayed for
sub-entries. If the symbol column is checked, then the symbol will
be displayed.}
\label{tab:styles}
\vskip\baselineskip
\centering
\begin{tabular}{lccc}
\bfseries Style & \bfseries Maximum Level &
\bfseries Homograph & \bfseries Symbol\\
\ttfamily listdotted & 0 & & \\
\ttfamily sublistdotted & 1 & & \\
\ttfamily list* & 1 & \tick & \\
\ttfamily altlist* & 1 & \tick & \\
\ttfamily long*3col* & 1 & \tick & \\
\ttfamily long4col* & 1 & \tick & \tick \\
\ttfamily altlong*4col* & 1 & \tick & \tick \\
\ttfamily long* & 1 & \tick & \\
\ttfamily super*3col* & 1 & \tick & \\
\ttfamily super4col* & 1 & \tick & \tick \\
\ttfamily altsuper*4col* & 1 & \tick & \tick \\
\ttfamily super* & 1 & \tick & \\
\ttfamily *index* & 2 & & \tick\\
\ttfamily treenoname* & --- & \tick & \tick\\
\ttfamily *tree* & --- & & \tick\\
\ttfamily *alttree* & --- & & \tick\\
\ttfamily inline & 1 & \tick &  
\end{tabular}
\par
\end{table}

The tabular-like styles that allow multi-line descriptions and page
lists use the length \DescribeMacro{\glsdescwidth}\cs{glsdescwidth}
to set the width of the description column and the length
\DescribeMacro{\glspagelistwidth}\cs{glspagelistwidth} to set the
width of the page list column.\footnote{These lengths will not be
available if you use both the \pkgopt{nolong} and \pkgopt{nosuper}
package options or if you use the \pkgopt{nostyles} package option
unless you explicitly load the relevant package.}
These will need to be changed using \cs{setlength} if the glossary
is too wide. Note that the \glostyle{long4col} and
\glostyle{super4col} styles (and their header and border variations)
don't use these lengths as they are designed for single line
entries. Instead you should use the analogous \glostyle{altlong4col}
and \glostyle{altsuper4col} styles.  If you want to
explicitly create a line-break within a multi-line description in
a tabular-like style it's better to use \ics{newline} instead of
\verb|\\|.

Note that if you use the \gloskey[printglossary]{style} key in the
optional argument to \ics{printnoidxglossary} (\opt1) or
\ics{printglossary} (\optsand23), it will override any 
previous style settings for the given glossary, so if, for example,
you do
\begin{verbatim}
\renewcommand*{\glsgroupskip}{}
\printglossary[style=long]
\end{verbatim}
then the new definition of \ics{glsgroupskip} will not have an affect
for this glossary, as \cs{glsgroupskip} is redefined by 
\verb|style=long|. Likewise, \ics{setglossarystyle} will also
override any previous style definitions, so, again
\begin{verbatim}
\renewcommand*{\glsgroupskip}{}
\setglossarystyle{long}
\end{verbatim}
will reset \cs{glsgroupskip} back to its default definition for the
named glossary style (\glostyle{long} in this case). If you want to 
modify the styles, either use \ics{newglossarystyle} (described
in the next section) or make the modifications after 
\ics{setglossarystyle}, e.g.:
\begin{verbatim}
\setglossarystyle{long}
\renewcommand*{\glsgroupskip}{}
\end{verbatim}
As from version 3.03, you can now use the package option
\pkgopt{nogroupskip} to suppress the gap between groups for the
default styles instead of redefining \cs{glsgroupskip}.

All the styles except for the three- and four-column styles and the
\glostyle{listdotted} style use the command
\begin{definition}[\DescribeMacro{\glspostdescription}]
\cs{glspostdescription}
\end{definition}
after the description. This simply displays a full stop by default.
To eliminate this full stop (or replace it with something else, say,
a comma) you will need to redefine \cs{glspostdescription} before
the glossary is displayed. Alternatively, you can suppress it for a
given entry by placing \ics{nopostdesc} in the entry's description.

As from version 3.03 you can now use the package option
\pkgopt{nopostdot} to suppress this full stop.

\subsection{List Styles}
\label{sec:liststyles}

The styles described in this section are all defined in the package
\sty{glossary-list}. Since they all use the \env{description}
environment, they are governed by the same parameters as that
environment. These styles all ignore the entry's symbol.  Note that
these styles will automatically be available unless you use the
\pkgopt{nolist} or \pkgopt{nostyles} package options.

\begin{description}
\item[list] The \glostyle{list} style uses the \env{description}
environment. The entry name is placed in the optional argument of
the \ics{item} command (so it will usually appear in bold by default). The
description follows, and then the associated \gls{numberlist} for 
that entry. The symbol is ignored.  If the entry has child entries,
the description and number list follows (but not the name) for each
child entry.  Groups are separated using \ics{indexspace}.

\item[listgroup] The \glostyle{listgroup} style is like 
\glostyle{list} but the glossary groups have headings.

\item[listhypergroup] The \glostyle{listhypergroup} style is like
\glostyle{listgroup} but has a navigation line at the start of the
glossary with links to each group that is present in the glossary.
This requires an additional run through \LaTeX\ to ensure the group
information is up to date. In the navigation line, each group is
separated by
\begin{definition}[\DescribeMacro{\glshypernavsep}]
\cs{glshypernavsep}
\end{definition}
which defaults to a vertical bar with a space on either side. For
example, to simply have a space separating each group, do:
\begin{verbatim}
\renewcommand*{\glshypernavsep}{\space}
\end{verbatim}

Note that the hyper-navigation line is now (as from version 1.14) 
set inside the optional argument to \ics{item} instead of after it
to prevent a spurious space at the start. This can be changed
by redefining \ics{glossaryheader}, but note that this needs to
be done \emph{after} the glossary style has been set.

\item[altlist] The \glostyle{altlist} style is like \glostyle{list}
but the description starts on the line following the name. (As
with the \glostyle{list} style, the symbol is ignored.) Each child
entry starts a new line, but as with the \glostyle{list} style,
the name associated with each child entry is ignored.

\item[altlistgroup] The \glostyle{altlistgroup} style is like 
\glostyle{altlist} but the glossary groups have headings.

\item[altlisthypergroup] The \glostyle{altlisthypergroup} style is like 
\glostyle{altlistgroup} but has a set of links to the glossary 
groups. The navigation line is the same as that for 
\glostyle{listhypergroup}, described above.

\item[listdotted] This style uses the \env{description}
environment.\footnote{This style was supplied by Axel~Menzel.} Each
entry starts with \verb|\item[]|, followed by the name followed by a
dotted line, followed by the description. Note that this style
ignores both the \gls{numberlist} and the symbol. The length
\begin{definition}[\DescribeMacro{\glslistdottedwidth}]
\cs{glslistdottedwidth}
\end{definition}
governs where the description should start. This is a flat style, so
child entries are formatted in the same way as the parent entries.

\item[sublistdotted] This is a variation on the \glostyle{listdotted}
style designed for hierarchical glossaries. The main entries
have just the name displayed. The sub entries are displayed in
the same manner as \glostyle{listdotted}.
\end{description}


\subsection{Longtable Styles}
\label{sec:longstyles}

The styles described in this section are all defined in the package
\sty{glossary-long}. Since they all use the \env{longtable}
environment, they are governed by the same parameters as that
environment.  Note that these styles will automatically be available
unless you use the \pkgopt{nolong} or \pkgopt{nostyles} package
options.  These styles fully justify the description and page list
columns.  If you want ragged right formatting instead, use the
analogous styles described in \sectionref{sec:longraggedstyles}.


\begin{description}
\item[long] The \glostyle{long} style uses the \env{longtable}
environment (defined by the \sty{longtable} package). It has two
columns: the first column contains the entry's name and the second
column contains the description followed by the \gls{numberlist}.
The entry's symbol is ignored.
Sub groups are separated with a blank row. The width of the
first column is governed by the widest entry in that column. The
width of the second column is governed by the length
\ics{glsdescwidth}. Child entries have a similar format to the
parent entries except that their name is suppressed.

\item[longborder] The \glostyle{longborder} style is like
\glostyle{long} but has horizontal and vertical lines around it.

\item[longheader] The \glostyle{longheader} style is like
\glostyle{long} but has a header row.

\item[longheaderborder] The \glostyle{longheaderborder} style is
like \glostyle{longheader} but has horizontal and vertical lines
around it.

\item[long3col] The \glostyle{long3col} style is like
\glostyle{long} but has three columns. The first column contains
the entry's name, the second column contains the description
and the third column contains the \gls{numberlist}.
The entry's symbol is ignored. The width of the
first column is governed by the widest entry in that column, the
width of the second column is governed by the length
\ics{glsdescwidth}, and the width of the third column is governed by the
length \ics{glspagelistwidth}.

\item[long3colborder] The \glostyle{long3colborder} style is like
the \glostyle{long3col} style but has horizontal and vertical
lines around it.

\item[long3colheader] The \glostyle{long3colheader} style is like
\glostyle{long3col} but has a header row.

\item[long3colheaderborder] The \glostyle{long3colheaderborder} style is
like \glostyle{long3colheader} but has horizontal and vertical lines
around it.

\item[long4col] The \glostyle{long4col} style is like 
\glostyle{long3col} but has an additional column in which the
entry's associated symbol appears. This style is used for brief
single line descriptions. The column widths are governed by the
widest entry in the given column. Use \glostyle{altlong4col} for 
multi-line descriptions.

\item[long4colborder] The \glostyle{long4colborder} style is like
the \glostyle{long4col} style but has horizontal and vertical
lines around it.

\item[long4colheader] The \glostyle{long4colheader} style is like
\glostyle{long4col} but has a header row.

\item[long4colheaderborder] The \glostyle{long4colheaderborder} style is
like \glostyle{long4colheader} but has horizontal and vertical lines
around it.

\item[altlong4col] The \glostyle{altlong4col} style is like
\glostyle{long4col} but allows multi-line descriptions and page
lists.  The width of the description column is governed by the
length \ics{glsdescwidth} and the width of the page list column is
governed by the length \linebreak\ics{glspagelistwidth}. The widths of the
name and symbol columns are governed by the widest entry in the
given column.

\item[altlong4colborder] The \glostyle{altlong4colborder} style is like
the \glostyle{long4colborder} but allows multi-line descriptions and
page lists.

\item[altlong4colheader] The \glostyle{altlong4colheader} style is like
\glostyle{long4colheader} but allows multi-line descriptions and
page lists.

\item[altlong4colheaderborder] The \glostyle{altlong4colheaderborder} 
style is like \linebreak\glostyle{long4colheaderborder} but allows multi-line
descriptions and page lists.
\end{description}


\subsection{Longtable Styles (Ragged Right)}
\label{sec:longraggedstyles}

The styles described in this section are all defined in the package
\sty{glossary-longragged}. These styles are analogous to those
defined in \sty{glossary-long} but the multiline columns are left
justified instead of fully justified. Since these styles all use the
\env{longtable} environment, they are governed by the same
parameters as that environment. The \sty{glossary-longragged}
package additionally requires the \sty{array} package.  Note that
these styles will only be available if you explicitly load
\sty{glossary-longragged}:
\begin{verbatim}
\usepackage{glossaries}
\usepackage{glossary-longragged}
\end{verbatim}
Note that you can't set these styles using the \pkgopt{style}
package option since the styles aren't defined until after
the \styfmt{glossaries} package has been loaded.

\begin{description}
\item[longragged] The \glostyle{longragged} style has two
columns: the first column contains the entry's name and the second
column contains the (left-justified) description followed by the 
\gls{numberlist}.  The entry's symbol is ignored.
Sub groups are separated with a blank row. The width of the
first column is governed by the widest entry in that column. The
width of the second column is governed by the length
\ics{glsdescwidth}. Child entries have a similar format to the
parent entries except that their name is suppressed.

\item[longraggedborder] The \glostyle{longraggedborder} style is like
\glostyle{longragged} but has horizontal and vertical lines around it.

\item[longraggedheader] The \glostyle{longraggedheader} style is like
\glostyle{longragged} but has a header row.

\item[longraggedheaderborder] The \glostyle{longraggedheaderborder} 
style is like \glostyle{longraggedheader} but has horizontal and 
vertical lines around it.

\item[longragged3col] The \glostyle{longragged3col} style is like
\glostyle{longragged} but has three columns. The first column
contains the entry's name, the second column contains the (left
justified) description and the third column contains the (left
justified) \gls{numberlist}.  The entry's symbol is ignored. The
width of the first column is governed by the widest entry in that
column, the width of the second column is governed by the length
\ics{glsdescwidth}, and the width of the third column is governed by
the length \ics{glspagelistwidth}.

\item[longragged3colborder] The \glostyle{longragged3colborder}
style is like the \glostyle{longragged3col} style but has horizontal
and vertical lines around it.

\item[longragged3colheader] The \glostyle{longragged3colheader} style is like
\glostyle{longragged3col} but has a header row.

\item[longragged3colheaderborder] The \glostyle{longragged3colheaderborder} style is
like \glostyle{longragged3colheader} but has horizontal and vertical lines
around it.

\item[altlongragged4col] The \glostyle{altlongragged4col} style is
like \glostyle{longragged3col} but has an additional column in which
the entry's associated symbol appears. The width of the description
column is governed by the length \ics{glsdescwidth} and the width of
the page list column is governed by the length
\ics{glspagelistwidth}. The widths of the name and symbol columns
are governed by the widest entry in the given column.

\item[altlongragged4colborder] The
\glostyle{altlongragged4colborder} style is like the
\glostyle{altlongragged4col} but has horizontal and vertical lines
around it.

\item[altlongragged4colheader] The
\glostyle{altlongragged4colheader} style is like
\glostyle{altlongragged4col} but has a header row.

\item[altlongragged4colheaderborder] The
\glostyle{altlongragged4colheaderborder} style is like
\glostyle{altlongragged4colheader} but has horizontal and vertical
lines around it.

\end{description}


\subsection{Supertabular Styles}
\label{sec:superstyles}

The styles described in this section are all defined in the package
\sty{glossary-super}. Since they all use the \env{supertabular}
environment, they are governed by the same parameters as that
environment.  Note that these styles will automatically be available
unless you use the \pkgopt{nosuper} or \pkgopt{nostyles} package
options.  In general, the \env{longtable} environment is better,
but there are some circumstances where it is better to use
\env{supertabular}.\footnote{e.g.\ with the \sty{flowfram}
package.} These styles fully justify the description and page list
columns.  If you want ragged right formatting instead, use the
analogous styles described in \sectionref{sec:superraggedstyles}.

\begin{description}
\item[super] The \glostyle{super} style uses the \env{supertabular}
environment (defined by the \sty{supertabular} package). It has two
columns: the first column contains the entry's name and the second
column contains the description followed by the \gls{numberlist}.
The entry's symbol is ignored.
Sub groups are separated with a blank row. The width of the
first column is governed by the widest entry in that column. The
width of the second column is governed by the length
\ics{glsdescwidth}. Child entries have a similar format to the
parent entries except that their name is suppressed.

\item[superborder] The \glostyle{superborder} style is like
\glostyle{super} but has horizontal and vertical lines around it.

\item[superheader] The \glostyle{superheader} style is like
\glostyle{super} but has a header row.

\item[superheaderborder] The \glostyle{superheaderborder} style is
like \glostyle{superheader} but has horizontal and vertical lines
around it.

\item[super3col] The \glostyle{super3col} style is like
\glostyle{super} but has three columns. The first column contains
the entry's name, the second column contains the description
and the third column contains the \gls{numberlist}. The
entry's symbol is ignored. The width of the
first column is governed by the widest entry in that column. The
width of the second column is governed by the length
\ics{glsdescwidth}. The width of the third column is governed by the
length \ics{glspagelistwidth}.

\item[super3colborder] The \glostyle{super3colborder} style is like
the \glostyle{super3col} style but has horizontal and vertical
lines around it.

\item[super3colheader] The \glostyle{super3colheader} style is like
\glostyle{super3col} but has a header row.

\item[super3colheaderborder] The \glostyle{super3colheaderborder} style
is like the \linebreak\glostyle{super3colheader} style but has horizontal and vertical
lines around it.

\item[super4col] The \glostyle{super4col} style is like 
\glostyle{super3col} but has an additional column in which the
entry's associated symbol appears. This style is designed for
entries with brief single line descriptions. The column widths are governed by the
widest entry in the given column. Use \glostyle{altsuper4col}
for multi-line descriptions.

\item[super4colborder] The \glostyle{super4colborder} style is like
the \glostyle{super4col} style but has horizontal and vertical
lines around it.

\item[super4colheader] The \glostyle{super4colheader} style is like
\glostyle{super4col} but has a header row.

\item[super4colheaderborder] The \glostyle{super4colheaderborder} style
is like the \linebreak\glostyle{super4colheader} style but has horizontal and vertical
lines around it.

\item[altsuper4col] The \glostyle{altsuper4col} style is like 
\glostyle{super4col} but allows multi-line descriptions and page
lists.
The width of the description column is governed by the length
\ics{glsdescwidth} and the width of the page list column is
governed by the length \ics{glspagelistwidth}. The width of the name 
and symbol columns is governed by the widest entry in the 
given column.

\item[altsuper4colborder] The \glostyle{altsuper4colborder} style is like
the \glostyle{super4colborder} style but allows multi-line descriptions
and page lists.

\item[altsuper4colheader] The \glostyle{altsuper4colheader} style is like
\glostyle{super4colheader} but allows multi-line descriptions and
page lists.

\item[altsuper4colheaderborder] The \glostyle{altsuper4colheaderborder}
style is like \glostyle{super4colheaderborder} but allows multi-line
descriptions and page lists.
\end{description}


\subsection{Supertabular Styles (Ragged Right)}
\label{sec:superraggedstyles}

The styles described in this section are all defined in the package
\sty{glossary-superragged}. These styles are analogous to those
defined in \sty{glossary-super} but the multiline columns are left
justified instead of fully justified. Since these styles all use the
\env{supertabular} environment, they are governed by the same
parameters as that environment. The \sty{glossary-superragged}
package additionally requires the \sty{array} package.  Note that
these styles will only be available if you explicitly load
\sty{glossary-superragged}:
\begin{verbatim}
\usepackage{glossaries}
\usepackage{glossary-superragged}
\end{verbatim}
Note that you can't set these styles using the \pkgopt{style}
package option since the styles aren't defined until after
the \styfmt{glossaries} package has been loaded.

\begin{description}
\item[superragged] The \glostyle{superragged} style uses the
\env{supertabular} environment (defined by the
\sty{supertabular} package). It has two columns: the first column
contains the entry's name and the second column contains the (left
justified) description followed by the \gls{numberlist}.  The
entry's symbol is ignored.  Sub groups are separated with a blank
row. The width of the first column is governed by the widest entry
in that column. The width of the second column is governed by the
length \ics{glsdescwidth}. Child entries have a similar format to
the parent entries except that their name is suppressed.

\item[superraggedborder] The \glostyle{superraggedborder} style is
like \glostyle{superragged} but has horizontal and vertical lines
around it.

\item[superraggedheader] The \glostyle{superraggedheader} style is
like \glostyle{superragged} but has a header row.

\item[superraggedheaderborder] The
\glostyle{superraggedheaderborder} style is like
\glostyle{superraggedheader} but has horizontal and vertical lines
around it.

\item[superragged3col] The \glostyle{superragged3col} style is like
\glostyle{superragged} but has three columns. The first column
contains the entry's name, the second column contains the (left
justified) description and the third column contains the (left
justified) \gls{numberlist}. The entry's symbol is ignored. The
width of the first column is governed by the widest entry in that
column. The width of the second column is governed by the length
\ics{glsdescwidth}. The width of the third column is governed by the
length \ics{glspagelistwidth}.

\item[superragged3colborder] The \glostyle{superragged3colborder}
style is like the \glostyle{superragged3col} style but has
horizontal and vertical lines around it.

\item[superragged3colheader] The \glostyle{superragged3colheader}
style is like \glostyle{superragged3col} but has a header row.

\item[superragged3colheaderborder] The
\glostyle{superragged3colheaderborder} style is like
\glostyle{superragged3colheader} but has horizontal and vertical
lines around it.

\item[altsuperragged4col] The \glostyle{altsuperragged4col} style is
like \glostyle{superragged3col} but has an additional column in
which the entry's associated symbol appears. The column widths for
the name and symbol column are governed by the widest entry in the
given column.

\item[altsuperragged4colborder] The
\glostyle{altsuperragged4colborder} style is like the
\glostyle{altsuperragged4col} style but has horizontal and vertical
lines around it.

\item[altsuperragged4colheader] The
\glostyle{altsuperragged4colheader} style is like
\glostyle{altsuperragged4col} but has a header row.

\item[altsuperragged4colheaderborder] The
\glostyle{altsuperragged4colheaderborder} style is like
\glostyle{altsuperragged4colheader} but has horizontal and vertical
lines around it.

\end{description}

\subsection{Tree-Like Styles}
\label{sec:treestyles}

The styles described in this section are all defined in the package
\sty{glossary-tree}. These styles are designed for hierarchical
glossaries but can also be used with glossaries that don't have
sub-entries. These styles will display the entry's symbol if it
exists. Note that these styles will automatically be available
unless you use the \pkgopt{notree} or \pkgopt{nostyles} package
options. These styles all format the entry name using:
\begin{definition}[\DescribeMacro\glstreenamefmt]
\cs{glstreenamefmt}\marg{name}
\end{definition}
This defaults to \cs{textbf}\marg{name}, but note that \meta{name}
includes \ics{glsnamefont} so the bold setting in \cs{glstreenamefont}
may be counteracted by another font change in \cs{glsnamefont} (or
in \cs{acronymfont}). The tree-like styles that also display the
header use \cs{glstreenamefmt} to format the heading.

\begin{description}
\item[index] The \glostyle{index} style is similar to the way
indices are usually formatted in that it has a hierarchical
structure up to three levels (the main level plus two sub-levels).
The name is typeset in bold, and if the symbol is present it is set
in parentheses after the name and before the description.
Sub-entries are indented and also include the name, the symbol
in brackets (if present) and the description.  Groups are separated
using \ics{indexspace}.

\item[indexgroup] The \glostyle{indexgroup} style is similar to
the \glostyle{index} style except that each group has a heading.

\item[indexhypergroup] The \glostyle{indexhypergroup} style is like 
\glostyle{indexgroup} but has a set of links to the glossary 
groups. The navigation line is the same as that for 
\glostyle{listhypergroup}, described above.

\item[tree] The \glostyle{tree} style is similar to the
\glostyle{index} style except that it can have arbitrary levels.
(Note that \gls{makeindex} is limited to three levels, so
you will need to use \gls{xindy} if you want more than 
three levels.) Each sub-level is indented by 
\DescribeMacro{\glstreeindent}\cs{glstreeindent}. Note that the
name, symbol (if present) and description are placed in the
same paragraph block. If you want the name to be apart from the
description, use the \glostyle{alttree} style instead. (See below.)

\item[treegroup] The \glostyle{treegroup} style is similar to
the \glostyle{tree} style except that each group has a heading.

\item[treehypergroup] The \glostyle{treehypergroup} style is like 
\glostyle{treegroup} but has a set of links to the glossary 
groups. The navigation line is the same as that for 
\glostyle{listhypergroup}, described above.

\item[treenoname] The \glostyle{treenoname} style is like
the \glostyle{tree} style except that the name for each sub-entry
is ignored.

\item[treenonamegroup] The \glostyle{treenonamegroup} style is 
similar to the \glostyle{treenoname} style except that each group 
has a heading.

\item[treenonamehypergroup] The \glostyle{treenonamehypergroup} 
style is like \glostyle{treenonamegroup} but has a set of links to 
the glossary groups. The navigation line is the same as that for 
\glostyle{listhypergroup}, described above.

\item[alttree] The \glostyle{alttree} style is similar to the
\glostyle{tree} style except that the indentation for each level
is determined by the width of the text specified by
\begin{definition}[\DescribeMacro{\glssetwidest}]
\cs{glssetwidest}\oarg{level}\marg{text}
\end{definition}
The optional argument \meta{level} indicates the level, where
0 indicates the top-most level, 1 indicates the first level 
sub-entries, etc. If \cs{glssetwidest} hasn't been used for a 
given sub-level, the level~0 widest text is used instead. If
\meta{level} is omitted, 0 is assumed.

For each level, the name is placed to the left of the paragraph
block containing the symbol (optional) and the description. If the
symbol is present, it is placed in parentheses before the
description.

\item[alttreegroup] The \glostyle{alttreegroup} is like the
\glostyle{alttree} style except that each group has a heading.

\item[alttreehypergroup] The \glostyle{alttreehypergroup} style is 
like \glostyle{alttreegroup} but has a set of links to the glossary 
groups. The navigation line is the same as that for 
\glostyle{listhypergroup}, described above.

\end{description}

\subsection{Multicols Style}
\label{sec:mcolstyles}

The \sty{glossary-mcols} package provides tree-like styles that are
in the \env{multicols} environment (defined by the \sty{multicol}
package). The style names are as their analogous tree styles (as
defined in \sectionref{sec:treestyles}) but are prefixed with
\qt{mcol}.
For example, the \glostyle{mcolindex} style is essentially the
\glostyle{index} style but put in a \env{multicols} environment.
For the complete list, see \tableref{tab:mcols}.

\begin{important}
Note that \sty{glossary-mcols} is not loaded by \styfmt{glossaries}. If
you want to use any of the multicol styles in that package you need
to load it explicitly with \cs{usepackage} and set the required glossary
style using \ics{setglossarystyle}.
\end{important}

The default number of columns is 2, but can be changed by redefining
\begin{definition}[\DescribeMacro{\glsmcols}]
\cs{glsmcols}
\end{definition}
to the required number. For example, for a three column glossary:
\begin{verbatim}
\usepackage{glossary-mcols}
\renewcommand*{\glsmcols}{3}
\setglossarystyle{mcolindex}
\end{verbatim}

\begin{table}[htbp]
\caption{Multicolumn Styles}
\label{tab:mcols}
\centering
\begin{tabular}{ll}
\bfseries
\sty{glossary-mcols} Style &
\bfseries
Analogous Tree Style\\
\glostyle{mcolindex} & \glostyle{index}\\
\glostyle{mcolindexgroup} & \glostyle{indexgroup}\\
\glostyle{mcolindexhypergroup} & \glostyle{indexhypergroup}\\
\glostyle{mcoltree} & \glostyle{tree}\\
\glostyle{mcoltreegroup} & \glostyle{treegroup}\\
\glostyle{mcoltreehypergroup} & \glostyle{treehypergroup}\\
\glostyle{mcoltreenoname} & \glostyle{treenoname}\\
\glostyle{mcoltreenonamegroup} & \glostyle{treenonamegroup}\\
\glostyle{mcoltreenonamehypergroup} & \glostyle{treenonamehypergroup}\\
\glostyle{mcolalttree} & \glostyle{alttree}\\
\glostyle{mcolalttreegroup} & \glostyle{alttreegroup}\\
\glostyle{mcolalttreehypergroup} & \glostyle{alttreehypergroup}
\end{tabular}
\end{table}

\subsection{In-Line Style}
\label{sec:inline}

This section covers the \sty{glossary-inline} package that supplies
the \glostyle{inline} style. This is a style that is designed for
in-line use (as opposed to block styles, such as lists or tables).
This style doesn't display the \gls{numberlist}.

You will most likely need to redefine \ics{glossarysection} with
this style. For example, suppose you are required to have your
glossaries and list of acronyms in a footnote, you can do:
\begin{verbatim}
 \usepackage{glossary-inline}

 \renewcommand*{\glossarysection}[2][]{\textbf{#1}: }
 \setglossarystyle{inline}
\end{verbatim}

\begin{important}
Note that you need to include \sty{glossary-inline} with
\cs{usepackage} as it's not automatically included by the
\styfmt{glossaries} package and then set the style using 
\ics{setglossarystyle}.
\end{important}

Where you need to include your glossaries as a footnote you can do:
\begin{verbatim}
\footnote{\printglossaries}
\end{verbatim}

The \glostyle{inline} style is governed by the following:
\begin{definition}[\DescribeMacro{\glsinlineseparator}]
\ics{glsinlineseparator}
\end{definition}
This defaults to \qt{\texttt{\glsinlineseparator}} and is used between
main (i.e.\ level~0) entries.
\begin{definition}[\DescribeMacro{\glsinlinesubseparator}]
\ics{glsinlinesubseparator}
\end{definition}
This defaults to \qt{\texttt{\glsinlinesubseparator}} and is used between
sub-entries.
\begin{definition}[\DescribeMacro{\glsinlineparentchildseparator}]
\ics{glsinlineparentchildseparator}
\end{definition}
This defaults to \qt{\texttt{\glsinlineparentchildseparator}} and is used between
a parent main entry and its first sub-entry.
\begin{definition}[\DescribeMacro{\glspostinline}]
\ics{glspostinline}
\end{definition}
This defaults to \qt{\texttt{\glsinlineseparator}} and is used at the end
of the glossary.

\section{Defining your own glossary style}
\label{sec:newglossarystyle}

If the predefined styles don't fit your requirements, you can
define your own style using:
\begin{definition}[\DescribeMacro{\newglossarystyle}]
\cs{newglossarystyle}\marg{name}\marg{definitions}
\end{definition}
where \meta{name} is the name of the new glossary style (to be
used in \cs{setglossarystyle}). The second argument \meta{definitions}
needs to redefine all of the following:

\begin{definition}[\DescribeEnv{theglossary}]
\env{theglossary}
\end{definition}
This environment defines how the main body of the glossary should
be typeset. Note that this does not include the section heading,
the glossary preamble (defined by \ics{glossarypreamble}) or the
glossary postamble (defined by \ics{glossarypostamble}). For example,
the \glostyle{list} style uses the \env{description} environment,
so the \env{theglossary} environment is simply redefined to begin
and end the \env{description} environment.

\begin{definition}[\DescribeMacro{\glossaryheader}]
\cs{glossaryheader}
\end{definition}
This macro indicates what to do at the start of the main body
of the glossary. Note that this is not the same as 
\ics{glossarypreamble}, which should not be affected by changes in
the glossary style. The \glostyle{list} glossary style redefines
\ics{glossaryheader} to do nothing, whereas the \glostyle{longheader}
glossary style redefines \cs{glossaryheader} to do a header row.

\begin{definition}[\DescribeMacro{\glsgroupheading}]
\cs{glsgroupheading}\marg{label}
\end{definition}
This macro indicates
what to do at the start of each logical block within the main body
of the glossary. If you use \gls{makeindex} the glossary is
sub-divided into a maximum of twenty-eight logical blocks that are
determined by the first character of the \gloskey{sort} key (or
\gloskey{name} key if the \gloskey{sort} key is omitted). The
sub-divisions are in the following order: symbols, numbers, A,
\ldots, Z\@. If you use \gls{xindy}, the sub-divisions depend on
the language settings.

Note that the argument to \cs{glsgroupheading}
is a label \emph{not} the group title. The group title can be obtained
via
\begin{definition}[\DescribeMacro{\glsgetgrouptitle}]
\cs{glsgetgrouptitle}\marg{label}
\end{definition}
This obtains the title as follows: if \meta{label} consists of a
single non-active character or \meta{label} is equal to
\texttt{glssymbols} or \texttt{glsnumbers} and
\cs{}\meta{label}\texttt{groupname} exists, this is taken to be the 
title, otherwise the title is just \meta{label}. (The \qt{symbols}
group has the label \texttt{glssymbols}, so the command 
\ics{glssymbolsgroupname} is used, and the \qt{numbers} group has the
label \texttt{glsnumbers}, so the command \ics{glsnumbersgrouptitle}
is used.) If you are using \gls{xindy}, \meta{label} may be an
active character (for example \o), in which case the title will be
set to just \meta{label}. You can redefine \cs{glsgetgrouptitle} if
this is unsuitable for your document.

A navigation hypertarget can be created using
\begin{definition}[\DescribeMacro{\glsnavhypertarget}]
\cs{glsnavhypertarget}\marg{label}\marg{text}
\end{definition}
For further details about \cs{glsnavhypertarget}, see
\ifpdf section~\ref*{sec:code:hypernav} in \fi the documented code 
(\texttt{glossaries-code.pdf}).

Most of the predefined glossary styles redefine \cs{glsgroupheading}
to simply ignore its argument.  The \glostyle{listhypergroup} style
redefines \cs{glsgroupheading} as follows:
\begin{verbatim}
\renewcommand*{\glsgroupheading}[1]{%
\item[\glsnavhypertarget{##1}{\glsgetgrouptitle{##1}}]}
\end{verbatim}
See also \cs{glsgroupskip} below. (Note that command definitions within 
\ics{newglossarystyle} must use \verb|##1| instead of \verb|#1| etc.)

\begin{definition}[\DescribeMacro{\glsgroupskip}]
\cs{glsgroupskip}
\end{definition}
This macro determines what to do after one logical group but before
the header for the next logical group. The \glostyle{list} glossary
style simply redefines \cs{glsgroupskip} to be \ics{indexspace},
whereas the tabular-like styles redefine \cs{glsgroupskip} to
produce a blank row.

As from version 3.03, the package option \pkgopt{nogroupskip} can be
used to suppress this default gap for the predefined styles.

\begin{definition}[\DescribeMacro{\glossentry}]
\cs{glossentry}\marg{label}\marg{number list}
\end{definition}
This macro indicates what to do for each level~0 glossary entry.
The entry label is given by \meta{label} and the associated
\gls{numberlist} is given by \meta{number list}. You can redefine
\ics{glossentry} to use commands like \ics{glossentryname}\marg{label},
\ics{glossentrydesc}\marg{label} and \ics{glossentrysymbol}\marg{label} 
to display the name, description and symbol fields, or to access
other fields, use commands like \ics{glsentryuseri}\marg{label}.
(See \sectionref{sec:glsnolink} for further details.) You can also
use the following commands:

\begin{definition}[\DescribeMacro{\glsentryitem}]
\cs{glsentryitem}\marg{label}
\end{definition}
This macro will increment and display the associated counter for the
main (level~0) entries if the \pkgopt{entrycounter} or
\pkgopt{counterwithin} package options have been used. This macro
is typically called by \cs{glossentry} before \cs{glstarget}. 
The format of the counter is controlled by the macro
\begin{definition}[\DescribeMacro{\glsentrycounterlabel}]
\cs{glsentrycounterlabel}
\end{definition}

Each time you use a glossary entry it creates a hyperlink (if
hyperlinks are enabled) to the relevant line in the glossary.
Your new glossary style must therefore redefine
\cs{glossentry} to set the appropriate target. This
is done using
\begin{definition}[\DescribeMacro{\glstarget}]
\cs{glstarget}\marg{label}\marg{text}
\end{definition}
where \meta{label} is the entry's label. Note that you
don't need to worry about whether the \sty{hyperref} package has
been loaded, as \cs{glstarget} won't create a target if
\cs{hypertarget} hasn't been defined.

For example, the \glostyle{list} style defines \cs{glossentry} as follows:
\begin{verbatim}
  \renewcommand*{\glossentry}[2]{%
    \item[\glsentryitem{##1}%
          \glstarget{##1}{\glossentryname{##1}}]
       \glossentrydesc{##1}\glspostdescription\space ##2}
\end{verbatim}

Note also that \meta{number list} will always be of the form
\begin{definition}
\ics{glossaryentrynumbers}\{\cs{relax}\\
\ics{setentrycounter}\oarg{Hprefix}\marg{counter name}\meta{format cmd}\marg{number(s)}\}
\end{definition}
where \meta{number(s)}
may contain \ics{delimN} (to delimit individual numbers) and/or
\ics{delimR} (to indicate a range of numbers). There may be 
multiple occurrences of 
\ics{setentrycounter}\oarg{Hprefix}\marg{counter name}\meta{format cmd}\marg{number(s)}, but note
that the entire number list is enclosed within the argument
of \linebreak\ics{glossaryentrynumbers}. The user can redefine this to change
the way the entire number list is formatted, regardless of 
the glossary style. However the most common use of
\ics{glossaryentrynumbers} is to provide a means of suppressing the
number list altogether. (In fact, the \pkgopt{nonumberlist} option 
redefines \ics{glossaryentrynumbers} to ignore its argument.)
Therefore, when you define a new glossary style, you don't need
to worry about whether the user has specified the 
\pkgopt{nonumberlist} package option.

\begin{definition}[\DescribeMacro{\subglossentry}]
\cs{subglossentry}\marg{level}\marg{label}\marg{number list}
\end{definition}
This is used to display sub-entries.
The first argument, \meta{level}, indicates the sub-entry level.
This must be an integer from 1 (first sub-level) onwards. The
remaining arguments are analogous to those for
\ics{glossentry} described above.

\begin{definition}[\DescribeMacro{\glssubentryitem}]
\cs{glssubentryitem}\marg{label}
\end{definition}
This macro will increment and display the associated counter for the
level~1 entries if the \pkgopt{subentrycounter} package option has
been used. This macro is typically called by \cs{subglossentry}
before \ics{glstarget}. The format of the counter is controlled by
the macro
\begin{definition}[\DescribeMacro{\glssubentrycounterlabel}]
\cs{glssubentrycounterlabel}
\end{definition}

Note that \ics{printglossary} (which \ics{printglossaries} calls)
sets
\begin{definition}[\DescribeMacro{\currentglossary}]
\cs{currentglossary}
\end{definition}
to the current glossary label, so it's possible to create a glossary
style that varies according to the glossary type.

For further details of these commands, see \ifpdf
section~\ref*{sec:code:printglos} \fi \qt{Displaying the glossary}
in the documented code (\texttt{glossaries-code.pdf}).

\begin{example}{Creating a completely new style}{sec:exnewstyle}

If you want a completely new style, you will need to redefine all
of the commands and the environment listed above. 

For example, suppose you want each entry to start with a bullet point.
This means that the glossary should be placed in the \env{itemize}
environment, so \env{theglossary} should start and end that
environment. Let's also suppose that you don't want anything between
the glossary groups (so \ics{glsgroupheading} and \ics{glsgroupskip}
should do nothing) and suppose you don't want anything to appear
immediately after \verb|\begin{theglossary}| (so \ics{glossaryheader}
should do nothing). In addition, let's suppose the symbol should
appear in brackets after the name, followed by the description and
last of all the \gls{numberlist} should appear within square brackets
at the end. Then you can create this new glossary style, called, say,
\texttt{mylist}, as follows:
\begin{verbatim}
 \newglossarystyle{mylist}{%
 % put the glossary in the itemize environment:
 \renewenvironment{theglossary}%
   {\begin{itemize}}{\end{itemize}}%
 % have nothing after \begin{theglossary}:
 \renewcommand*{\glossaryheader}{}%
 % have nothing between glossary groups:
 \renewcommand*{\glsgroupheading}[1]{}%
 \renewcommand*{\glsgroupskip}{}%
 % set how each entry should appear:
 \renewcommand*{\glossentry}[2]{%
 \item % bullet point
 \glstarget{##1}{\glossentryname{##1}}% the entry name
 \space (\glossentrysymbol{##1})% the symbol in brackets
 \space \glossentrydesc{##1}% the description
 \space [##2]% the number list in square brackets
 }%
 % set how sub-entries appear:
 \renewcommand*{\subglossentry}[3]{%
   \glossentry{##2}{##3}}%
 }
\end{verbatim}
Note that this style creates a flat glossary, where sub-entries
are displayed in exactly the same way as the top level entries.
It also hasn't used \ics{glsentryitem} or \ics{glssubentryitem} so
it won't be affected by the \pkgopt{entrycounter},
\pkgopt{counterwithin} or \pkgopt{subentrycounter} package options.

Variations:
\begin{itemize}
\item You might want the entry name to be capitalised, in
which case use \ics{Glossentryname} instead of \ics{glossentryname}.

\item You might want to check if the symbol hasn't been set and omit
the parentheses if the symbol is absent. In this case you can use
\ics{ifglshassymbol} (see \sectionref{sec:utilities}):
\begin{verbatim}
 \renewcommand*{\glossentry}[2]{%
 \item % bullet point
 \glstarget{##1}{\glossentryname{##1}}% the entry name
 \ifglshassymbol{##1}% check if symbol exists
 {%
   \space (\glossentrysymbol{##1})% the symbol in brackets
 }%
 {}% no symbol so do nothing
 \space \glossentrydesc{##1}% the description
 \space [##2]% the number list in square brackets
 }%
\end{verbatim}
\end{itemize}
\end{example}

\begin{example}{Creating a new glossary style based on an
existing style}{sec:exadaptstyle}

If you want to define a new style that is a slightly modified
version of an existing style, you can use \ics{setglossarystyle}
within the second argument of \ics{newglossarystyle} followed by
whatever alterations you require. For example, suppose you want 
a style like the \glostyle{list} style but you don't want the extra
vertical space created by \ics{indexspace} between groups, then you
can create a new glossary style called, say, \texttt{mylist} as
follows:
\begin{verbatim}
\newglossarystyle{mylist}{%
\setglossarystyle{list}% base this style on the list style
\renewcommand{\glsgroupskip}{}% make nothing happen 
                              % between groups
}
\end{verbatim} 
(In this case, you can actually achieve the same effect using the
\glostyle{list} style in combination with the package option
\pkgopt{nogroupskip}.)
\end{example}

\begin{example}{Example: creating a glossary style that uses the
\texorpdfstring{\gloskey{user1}}{user1}, \ldots, 
\texorpdfstring{\gloskey{user6}}{user6} keys}{sec:exuserstyle}

Suppose each entry not only has an associated symbol,
but also units (stored in \gloskey{user1}) and dimension
(stored in \gloskey{user2}). Then you can define a glossary style
that displays each entry in a \env{longtable} as follows:
\begin{verbatim}
\newglossarystyle{long6col}{%
 % put the glossary in a longtable environment:
 \renewenvironment{theglossary}%
  {\begin{longtable}{lp{\glsdescwidth}cccp{\glspagelistwidth}}}%
  {\end{longtable}}%
 % Set the table's header:
 \renewcommand*{\glossaryheader}{%
  \bfseries Term & \bfseries Description & \bfseries Symbol &
  \bfseries Units & \bfseries Dimensions & \bfseries Page List
  \\\endhead}%
 % No heading between groups:
  \renewcommand*{\glsgroupheading}[1]{}%
 % Main (level 0) entries displayed in a row optionally numbered:
  \renewcommand*{\glossentry}[2]{%
    \glsentryitem{##1}% Entry number if required
    \glstarget{##1}{\glossentryname{##1}}% Name
    & \glossentrydesc{##1}% Description
    & \glossentrysymbol{##1}% Symbol
    & \glsentryuseri{##1}% Units
    & \glsentryuserii{##1}% Dimensions
    & ##2% Page list
    \tabularnewline % end of row
  }%
 % Similarly for sub-entries (no sub-entry numbers):
 \renewcommand*{\subglossentry}[3]{%
    % ignoring first argument (sub-level)
    \glstarget{##2}{\glossentryname{##2}}% Name
    & \glossentrydesc{##2}% Description
    & \glossentrysymbol{##2}% Symbol
    & \glsentryuseri{##2}% Units
    & \glsentryuserii{##2}% Dimensions
    & ##3% Page list
    \tabularnewline % end of row
  }%
 % Nothing between groups:
 \renewcommand*{\glsgroupskip}{}%
}
\end{verbatim}
\end{example}

\chapter{Utilities}
\label{sec:utilities}

This section describes some utility commands. Additional commands
can be found in the documented code (glossaries-code.pdf).

\begin{important}
Some of the commands described here take a comma-separated list as
an argument. As with \LaTeX's \cs{@for} command, make sure your list
doesn't have any unwanted spaces in it as they don't get stripped.
\end{important}

\begin{definition}[\DescribeMacro{\forallglossaries}]
\cs{forallglossaries}\oarg{glossary list}\marg{cs}\marg{body}
\end{definition}
This iterates through \meta{glossary list}, a comma-separated list
of glossary labels (as supplied when the glossary was defined).
At each iteration \meta{cs} (which must be a control sequence) is
set to the glossary label for the current iteration and \meta{body}
is performed. If \meta{glossary list} is omitted, the default is to
iterate over all glossaries (except the ignored ones).

\begin{definition}[\DescribeMacro{\forallacronyms}]
\cs{forallacronyms}\marg{cs}\marg{body}
\end{definition}
This is like \cs{forallglossaries} but only iterates over the
lists of acronyms (that have previously been declared using
\ics{DeclareAcronymList} or the \pkgopt{acronymlists} package
option). This command doesn't have an optional argument. If you want
to explicitly say which lists to iterate over, just use the optional
argument of \cs{forallglossaries}.

\begin{definition}[\DescribeMacro{\forglsentries}]
\cs{forglsentries}\oarg{glossary label}\marg{cs}\marg{body}
\end{definition}
This iterates through all entries in the glossary given by
\meta{glossary label}. At each iteration \meta{cs} (which must be a
control sequence) is set to the entry label for the current
iteration and \meta{body} is performed. If \meta{glossary label} is
omitted, \ics{glsdefaulttype} (usually the main glossary) is used.

\begin{definition}[\DescribeMacro{\forallglsentries}]
\cs{forallglsentries}\oarg{glossary list}\marg{cs}\marg{body}
\end{definition}
This is like \cs{forglsentries} but for each glossary in
\meta{glossary list} (a comma-separated list of glossary labels). If
\meta{glossary list} is omitted, the default is the list of all
defined glossaries (except the ignored ones). At each iteration 
\meta{cs} is set to the entry label and \meta{body} is performed. 
(The current glossary label can be obtained using 
\cs{glsentrytype}\marg{cs} within \meta{body}.)

\begin{definition}[\DescribeMacro{\ifglossaryexists}]
\cs{ifglossaryexists}{\meta{label}}{\meta{true part}}{\meta{false
part}}
\end{definition}
This checks if the glossary given by \meta{label} exists. If it
does \meta{true part} is performed, otherwise \meta{false part}.

\begin{definition}[\DescribeMacro{\ifglsentryexists}]
\cs{ifglsentryexists}{\meta{label}}{\meta{true part}}{\meta{false
part}}
\end{definition}
This checks if the glossary entry given by \meta{label} exists. If it
does \meta{true part} is performed, otherwise \meta{false part}.
(Note that \cs{ifglsentryexists} will always be true after the
containing glossary has been displayed via \ics{printglossary}
or \ics{printglossaries} even if the entry is explicitly defined
later in the document. This is because the entry has to be defined
before it can be displayed in the glossary, see \sectionref{sec:techissues} for
further details.)

\begin{definition}[\DescribeMacro{\glsdoifexists}]
\cs{glsdoifexists}\marg{label}\marg{code}
\end{definition}
Does \meta{code} if the entry given by \meta{label} exists. If it
doesn't exist, an error is generated. (This command uses
\cs{ifglsentryexists}.)

\begin{definition}[\DescribeMacro{\glsdoifnoexists}]
\cs{glsdoifnoexists}\marg{label}\marg{code}
\end{definition}
Does the reverse of \cs{glsdoifexists}. (This command uses
\cs{ifglsentryexists}.)

\begin{definition}[\DescribeMacro{\glsdoifexistsorwarn}]
\cs{glsdoifexistsorwarn}\marg{label}\marg{code}
\end{definition}
As \cs{glsdoifexists} but issues a warning rather than an error if
the entry doesn't exist.

\begin{definition}[\DescribeMacro{\ifglsused}]
\cs{ifglsused}{\meta{label}}{\meta{true part}}{\meta{false
part}}
\end{definition}
See \sectionref{sec:glsunset}.

\begin{definition}[\DescribeMacro{\ifglshaschildren}]
\cs{ifglshaschildren}{\meta{label}}{\meta{true part}}{\meta{false
part}}
\end{definition}
This checks if the glossary entry given by \meta{label} has any
sub-entries. If it does, \meta{true part} is performed, otherwise
\meta{false part}.

\begin{definition}[\DescribeMacro{\ifglshasparent}]
\cs{ifglshasparent}{\meta{label}}{\meta{true part}}{\meta{false
part}}
\end{definition}
This checks if the glossary entry given by \meta{label} has a parent
entry. If it does, \meta{true part} is performed, otherwise
\meta{false part}.

\begin{definition}[\DescribeMacro{\ifglshassymbol}]
\cs{ifglshassymbol}\marg{label}\marg{true part}\marg{false part}
\end{definition}
This checks if the glossary entry given by \meta{label} has had the
\gloskey{symbol} field set. If it has, \meta{true part} is performed,
otherwise \meta{false part}.

\begin{definition}[\DescribeMacro{\ifglshaslong}]
\cs{ifglshaslong}\marg{label}\marg{true part}\marg{false part}
\end{definition}
This checks if the glossary entry given by \meta{label} has had the
\gloskey{long} field set. If it has, \meta{true part} is performed,
otherwise \meta{false part}. This should be true for any entry that
has been defined via \ics{newacronym}.
There is no check for the existence of \meta{label}.

\begin{definition}[\DescribeMacro{\ifglshasshort}]
\cs{ifglshasshort}\marg{label}\marg{true part}\marg{false part}
\end{definition}
This checks if the glossary entry given by \meta{label} has had the
\gloskey{short} field set. If it has, \meta{true part} is performed,
otherwise \meta{false part}. This should be true for any entry that
has been defined via \ics{newacronym}.
There is no check for the existence of \meta{label}.

\begin{definition}[\DescribeMacro{\ifglshasdesc}]
\cs{ifglshasdesc}\marg{label}\marg{true part}\marg{false part}
\end{definition}
This checks if the \gloskey{description} field is non-empty for 
the entry given by \meta{label}. If it has, \meta{true part} is 
performed, otherwise \meta{false part}. Compare with:
\begin{definition}[\DescribeMacro{\ifglsdescsuppressed}]
\cs{ifglsdescsuppressed}\marg{label}\marg{true part}\marg{false part}
\end{definition}
This checks if the \gloskey{description} field has been set to just
\ics{nopostdesc} for the entry given by \meta{label}. If it has, 
\meta{true part} is performed, otherwise \meta{false part}.
There is no check for the existence of \meta{label}.

For all other fields you can use:
\begin{definition}
\cs{ifglshasfield}\marg{field}\marg{label}\marg{true part}\marg{false
part}
\end{definition}
This checks if the field given by \meta{field} for the entry
identified by \meta{label} is empty. If it is, \meta{true part} is
performed, otherwise \meta{false part}. If the field supplied is 
unrecognised \meta{false part}
is performed and a warning is issued. Unlike the above commands,
such as \cs{ifglshasshort}, an error occurs if the entry is
undefined.

\chapter{Prefixes or Determiners}\label{sec:prefix}

The \sty{glossaries-prefix} package provides 
additional keys that can be used as prefixes. For example, if you
want to specify determiners (such
as \qt{a}, \qt{an} or \qt{the}). The \styfmt{glossaries-prefix}
package automatically loads the \styfmt{glossaries} package and has
the same package options.

The extra keys for \ics{newglossaryentry} are as follows:
\begin{description}
\item[\gloskey{prefix}] The prefix associated with the
\gloskey{text} key. This defaults to nothing.

\item[\gloskey{prefixplural}] The prefix associated with the
\gloskey{plural} key. This defaults to nothing.

\item[\gloskey{prefixfirst}] The prefix associated with the
\gloskey{first} key. If omitted, this defaults to the value of the
\gloskey{prefix} key.

\item[\gloskey{prefixfirstplural}] The prefix associated with the
\gloskey{firstplural} key. If omitted, this defaults to the value of
the \gloskey{prefixplural} key.
\end{description}

\begin{example}{Defining Determiners}{ex:determiners}
Here's the start of my example document:
\begin{verbatim}
documentclass{article}

\usepackage[colorlinks]{hyperref}
\usepackage[toc,acronym]{glossaries-prefix}
\end{verbatim}
Note that I've simply replaced \styfmt{glossaries} from previous
sample documents with \styfmt{glossaries-prefix}. Now for a sample
definition\footnote{Single letter words, such as \qt{a} and \qt{I}
should typically not appear at the end of a line, hence the
non-breakable space after \qt{a} in the \gloskey{prefix} field.}:
\begin{verbatim}
\newglossaryentry{sample}{name={sample},%
  description={an example},%
  prefix={a~},%
  prefixplural={the\space}%
}
\end{verbatim}
Note that I've had to explicitly insert a~space after the prefix.
This allows for the possibility of prefixes that shouldn't have a~space, 
such as:
\begin{verbatim}
\newglossaryentry{oeil}{name={oeil},
  plural={yeux},
  description={eye},
  prefix={l'},
  prefixplural={les\space}}
\end{verbatim}
Where a space is required at the end of the prefix, you must use 
a~spacing command, such as \cs{space}, \verb*|\ | (backslash space) or
\verb|~| due to the automatic spacing trimming performed in
\meta{key}=\meta{value} options.

The prefixes can also be used with acronyms. For example:
\begin{verbatim}
\newacronym
 [%
   prefix={an\space},prefixfirst={a~}%
 ]{svm}{SVM}{support vector machine}
\end{verbatim}
\end{example}

The \sty{glossaries-prefix} package provides convenient commands to
use these prefixes with commands such as \ics{gls}. Note that the
prefix is not considered part of the \gls{linktext}, so it's not
included in the hyperlink (where hyperlinks are enabled). The
options and any star or plus modifier are passed on to the \glslike\
command. (See \sectionref{sec:glslink} for further details.)

\begin{definition}[\DescribeMacro\pgls]
\cs{pgls}\oarg{options}\marg{label}\oarg{insert}
\end{definition}
This is inserts the value of the \gloskey{prefix} key (or
\gloskey{prefixfirst} key, on \gls{firstuse}) in front of
\cs{gls}\oarg{options}\marg{label}\oarg{insert}.

\begin{definition}[\DescribeMacro\Pgls]
\cs{Pgls}\oarg{options}\marg{label}\oarg{insert}
\end{definition}
If the \gloskey{prefix} key (or \gloskey{prefixfirst}, on
\gls*{firstuse}) has been set, this displays the value of that key
with the first letter converted to upper case followed by
\cs{gls}\oarg{options}\marg{label}\oarg{insert}. If that key hasn't
been set, this is equivalent to
\ics{Gls}\oarg{options}\marg{label}\oarg{insert}.

\begin{definition}[\DescribeMacro\PGLS]
\cs{PGLS}\oarg{options}\marg{label}\oarg{insert}
\end{definition}
As \cs{pgls} but converts the prefix to upper case and uses
\ics{GLS} instead of \cs{gls}.

\begin{definition}[\DescribeMacro\pglspl]
\cs{pglspl}\oarg{options}\marg{label}\oarg{insert}
\end{definition}
This is inserts the value of the \gloskey{prefixplural} key (or
\gloskey{prefixfirstplural} key, on \gls{firstuse}) in front of
\cs{glspl}\oarg{options}\marg{label}\oarg{insert}.

\begin{definition}[\DescribeMacro\Pglspl]
\cs{Pglspl}\oarg{options}\marg{label}\oarg{insert}
\end{definition}
If the \gloskey{prefixplural} key (or \gloskey{prefixfirstplural}, on
\gls*{firstuse}) has been set, this displays the value of that key
with the first letter converted to upper case followed by
\cs{glspl}\oarg{options}\marg{label}\oarg{insert}. If that key hasn't
been set, this is equivalent to
\ics{Glspl}\oarg{options}\marg{label}\oarg{insert}.

\begin{definition}[\DescribeMacro\PGLSpl]
\cs{PGLSpl}\oarg{options}\marg{label}\oarg{insert}
\end{definition}
As \cs{pglspl} but converts the prefix to upper case and uses
\ics{GLSpl} instead of \cs{glspl}.

\begin{example}{Using Prefixes}{ex:prefixes}
Continuing from Example~\ref{ex:determiners}, now that I've defined
my entries, I~can use them in the text via the above commands:
\begin{verbatim}
First use: \pgls{svm}. Next use: \pgls{svm}.
Singular: \pgls{sample}, \pgls{oeil}.
Plural: \pglspl{sample}, \pglspl{oeil}.
\end{verbatim}
which produces:
\begin{quote}
First use: a~support vector machine (SVM). Next
use: an SVM.
Singular: a~sample, l'oeil. Plural: the samples, les yeux.
\end{quote}
For a complete document, see \samplefile{-prefix}.
\end{example}

This package also provides the commands described below, none of
which perform any check to determine the entry's existence.

\begin{definition}[\DescribeMacro\ifglshasprefix]
\cs{ifglshasprefix}\marg{label}\marg{true part}\marg{false part}
\end{definition}
Does \meta{true part} if the entry identified by \meta{label} has a
non-empty value for the \gloskey{prefix} key.

This package also provides the following commands:
\begin{definition}[\DescribeMacro\ifglshasprefixplural]
\cs{ifglshasprefixplural}\marg{label}\marg{true part}\marg{false part}
\end{definition}
Does \meta{true part} if the entry identified by \meta{label} has a
non-empty value for the \gloskey{prefixplural} key.

\begin{definition}[\DescribeMacro\ifglshasprefixfirst]
\cs{ifglshasprefixfirst}\marg{label}\marg{true part}\marg{false part}
\end{definition}
Does \meta{true part} if the entry identified by \meta{label} has a
non-empty value for the \gloskey{prefixfirst} key.

\DescribeMacro\ifglshasprefixfirstplural
\begin{definition}
\cs{ifglshasprefixfirstplural}\marg{label}\marg{true part}\marg{false part}
\end{definition}
Does \meta{true part} if the entry identified by \meta{label} has a
non-empty value for the \gloskey{prefixfirstplural} key.

\begin{definition}[\DescribeMacro\glsentryprefix]
\cs{glsentryprefix}\marg{label}
\end{definition}
Displays the value of the \gloskey{prefix} key for the entry given
by \meta{label}.

\begin{definition}[\DescribeMacro\glsentryprefixfirst]
\cs{glsentryprefixfirst}\marg{label}
\end{definition}
Displays the value of the \gloskey{prefixfirst} key for the entry given
by \meta{label}.

\begin{definition}[\DescribeMacro\glsentryprefixplural]
\cs{glsentryprefixplural}\marg{label}
\end{definition}
Displays the value of the \gloskey{prefixplural} key for the entry given
by \meta{label}. (No check is performed to determine if the entry
exists.)

\DescribeMacro\glsentryprefixfirstplural
\begin{definition}
\cs{glsentryprefixfirstplural}\marg{label}
\end{definition}
Displays the value of the \gloskey{prefixfirstplural} key for the entry given
by \meta{label}. (No check is performed to determine if the entry
exists.)

There are also variants that convert the first letter to upper
case\footnote{The earlier caveats about initial non-Latin characters
apply.}:
\begin{definition}[\DescribeMacro\Glsentryprefix]
\cs{Glsentryprefix}\marg{label}
\end{definition}
\begin{definition}[\DescribeMacro\Glsentryprefixfirst]
\cs{Glsentryprefixfirst}\marg{label}
\end{definition}
\begin{definition}[\DescribeMacro\Glsentryprefixplural]
\cs{Glsentryprefixplural}\marg{label}
\end{definition}

\DescribeMacro\Glsentryprefixfirstplural
\begin{definition}
\cs{Glsentryprefixfirstplural}\marg{label}
\end{definition}

\begin{important}
As with analogous commands such as \ics{Glsentrytext}, these
commands aren't expandable so can't be used in PDF bookmarks.
\end{important}

\begin{example}{Adding Determiner to Glossary Style}{ex:plist}
You can use the above commands to define a new glossary style that
uses the determiner. For example, the following style is a slight
modification of the \glostyle{list} style that inserts the prefix
before the name:
\begin{verbatim}
\newglossarystyle{plist}{%
  \setglossarystyle{list}%
  \renewcommand*{\glossentry}[2]{%
    \item[\glsentryitem{##1}%
          \Glsentryprefix{##1}%
          \glstarget{##1}{\glossentryname{##1}}]
       \glossentrydesc{##1}\glspostdescription\space ##2}%
}
\end{verbatim}
\end{example}

\chapter{Accessibility Support}\label{sec:accsupp}

Limited accessibility support is provided by the accompanying
\sty{glossaries-accsupp} package, but note that this package is
experimental and it requires the \sty{accsupp} package 
which is also listed as experimental. This package defines
additional keys that may be used when defining glossary entries.
The keys are as follows:
\begin{description}
\item[{\gloskey{access}}] The replacement text corresponding to
the \gloskey{name} key.

\item[{\gloskey{textaccess}}] The replacement text corresponding
to the \gloskey{text} key.

\item[{\gloskey{firstaccess}}] The replacement text corresponding
to the \gloskey{first} key.

\item[{\gloskey{pluralaccess}}] The replacement text corresponding
to the \gloskey{plural} key.

\item[{\gloskey{firstpluralaccess}}] The replacement text corresponding
to the \gloskey{firstplural} key.

\item[{\gloskey{symbolaccess}}] The replacement text corresponding
to the \gloskey{symbol} key.

\item[{\gloskey{symbolpluralaccess}}] The replacement text corresponding
to the \gloskey{symbolplural} key.

\item[{\gloskey{descriptionaccess}}] The replacement text corresponding
to the \gloskey{description} key.

\item[{\gloskey{descriptionpluralaccess}}] The replacement text corresponding
to the \gloskey{descriptionplural} key.

\item[{\gloskey{longaccess}}] The replacement text corresponding to
the \gloskey{long} key (used by \ics{newacronym}).

\item[{\gloskey{shortaccess}}] The replacement text corresponding to
the \gloskey{short} key (used by \ics{newacronym}).

\item[{\gloskey{longpluralaccess}}] The replacement text corresponding to
the \gloskey{longplural} key (used by \ics{newacronym}).

\item[{\gloskey{shortpluralaccess}}] The replacement text corresponding to
the \gloskey{shortplural} key (used by \ics{newacronym}).

\end{description}
For example:
\begin{verbatim}
\newglossaryentry{tex}{name={\TeX},description={Document 
preparation language},access={TeX}}
\end{verbatim}
Now \verb|\gls{tex}| will be equivalent to
\begin{verbatim}
\BeginAccSupp{ActualText=TeX}\TeX\EndAccSupp{}
\end{verbatim}
The sample file \samplefile{accsupp} illustrates the
\sty{glossaries-accsupp} package.

See \ifpdf section~\ref*{sec:code:accsupp} in \fi the documented code
(\texttt{glossaries-code.pdf}) for further details. It is recommended
that you also read the \sty{accsupp} documentation. 

\chapter{Troubleshooting}
\label{sec:trouble}

The \styfmt{glossaries} package comes with a minimal file called 
\texttt{minimalgls.tex} which can be used for testing. This
should be located in the \texttt{samples} subdirectory (folder)
of the \styfmt{glossaries} documentation directory. The location
varies according to your operating system and \TeX\ installation.
For example, on my Linux partition it can be found in
\texttt{\slash usr\slash local\slash texlive\slash 2013\slash
texmf-dist\slash doc\slash latex\slash glossaries/}. 
Further information on debugging \LaTeX\ code is available at
\url{http://www.dickimaw-books.com/latex/minexample/}.

If you have any problems, please first consult the 
\urlfootref{http://www.dickimaw-books.com/faqs/glossariesfaq.html}{\styfmt{glossaries} FAQ}. If that
doesn't help, try posting your query to somewhere like the
comp.text.tex newsgroup, the 
\urlfootref{http://www.latex-community.org/}{\LaTeX\ Community Forum} or 
\urlfootref{http://tex.stackexchange.com/}{\TeX\ on StackExchange}. 
Bug reports can be submitted via 
\urlfootref{http://www.dickimaw-books.com/bug-report.html}{my package bug report form}. 

\PrintIndex

\end{document}
